% This file was generated by downloading the book in pdf form, and then by
% running a pdf->latex converter.
%
% The book pdf was downloaded from: https://kheavan.files.wordpress.com/2011/10/mathematical-olympiads-1997-1998-problems-solutions-from-around-the-world-maa-problem-book-225p-b002kypabi.pdf
%
% The book contains several national olympiads from 1997 (problems + solutions),
% and from 1998 (problems only).
%
% The pdf->latex conversion was done via the software inftyreader (http://www.inftyreader.org/).
%
% Due to a software restriction, only the first 100 of 225 pages were converted.
% I believe that only 100 pages get converted at a time. I will rerun for the
% other pages and commit separately in the future.
%
% The software does an admirable job, but there are some immediately noticeable
% issues in the latex, some of which can likely be fixed with some clever
% automated post-processing. A sampling of issues I immediately observed:
%
% - Line breaks and section headers are not always accurately converted.
%
% - Page numbers at the bottom of each page were directly inserted into the text.
%
% - Automated numbering constructs are never used (such as \enumerate, \ref, \label, etc.).
%
% - Sometimes \it is used instead of math-mode, seemingly randomly.
%
% - I see "mod" written as: $\mathrm{m}\mathrm{o}\mathrm{d}$. Similarly for
%   other acronyms like gcd.
%
% - Sometimes I see swaths of text written in math-mode with \mathrm.

\documentclass[a4paper,10pt]{article}
\usepackage{latexsym}
\usepackage{amsmath}
\usepackage{amssymb}
\usepackage{bm}
\usepackage{graphicx}
\usepackage{wrapfig}
\usepackage{fancybox}
\pagestyle{plain}

\begin{document}

\includegraphics[width=43.48mm,height=4.44mm]{./mathematical-olympiads-1997-1998-problems-solutions-from-around-the-world-maa-problem-book-225p-b002kypabi_images/image001.eps}

\includegraphics[width=33.15mm,height=5.63mm]{./mathematical-olympiads-1997-1998-problems-solutions-from-around-the-world-maa-problem-book-225p-b002kypabi_images/image002.eps}

\includegraphics[width=62.70mm,height=3.73mm]{./mathematical-olympiads-1997-1998-problems-solutions-from-around-the-world-maa-problem-book-225p-b002kypabi_images/image003.eps}

\includegraphics[width=64.60mm,height=3.81mm]{./mathematical-olympiads-1997-1998-problems-solutions-from-around-the-world-maa-problem-book-225p-b002kypabi_images/image004.eps}

Preface

This book is a continuation of {\it Mathematical Olympiads} 1996-1997: {\it Olym}{\it piad Problems from Around the World}, published by the American Mathematics Competitions. It contains solutions to the problems from 34 national and regional contests featured in the earlier book, together with selected problems (without solutions) from national and regional contests given during 1998.

This collection is intended as practice for the serious student who wishes to improve his or her performance on the USAMO. Some of the problems are comparable to the USAMO in that they came from national contests. Others are harder, as some countries first have a national olympiad, and later one or more exams to select a team for the IMO. And some problems come from regional international contests (�mini-IMOs�). Different nations have different mathematical cultures, so you will find some of these problems extremely hard and some rather easy. We have tried to present a wide variety of problems, especially from those countries that have often done well at the IMO.

Each contest has its own time limit. We have not furnished this information, because we have not always included complete exams. As a rule of thumb, most contests allow a time limit ranging between one-half to one full hour per problem.

Thanks to the following students of the 1998 and 1999 Mathematical Olympiad Summer Programs for their help in preparing and proofreading solutions: David Arthur, Reid Barton, Gabriel Carroll, Chi-Bong Chan, Lawrence Detlor, Daniel Katz, George Lee, Po-Shen Loh, Yogesh More, Oaz Nir, David Speyer, Paul Valiant, Melanie Wood. Without their efforts, this work would not have been possible. Thanks also to Alexander Soifer for correcting an early draft of the manuscript.

The problems in this publication are copyrighted. Requests for reproduction permissions should be directed to:

Dr. Walter Mientka

Secretary, IMO Advisory Board 1740 Vine Street

Lincoln, NE 68588-0658, USA.

Contents

1 1997 National Contests: Solutions 3

1.1 Austria . . . . . . . . . .

1.2 Bulgaria . . . . . . . . . .

1.3 Canada . . . . . . . . . .

1.4 China . . . . . . . . . . .

1.5 Colombia . . . . . . . . .

1.6 Czech and Slovak Republics 1.7 France . . . . . . . . . . .

1.8 Germany . . . . . . . . .

1.9 Greece . . . . . . . . . . .

1.10 Hungary . . . . . . . . . .

1.11 Iran . . . . . . . . . . . .

1.12 Ireland . . . . . . . . . . .

1.13 Italy . . . . . . . . . . . .

1.14 Japan . . . . . . . . . . .

1.15 Korea . . . . . . . . . . .

1.16 Poland . . . . . . . . . . .

1.17 Romania . . . . . . . . . .

1.18 Russia . . . . . . . . . . .

1.19 South Africa . . . . . . .

1.20 Spain . . . . . . . . . . .

1.21 Taiwan . . . . . . . . . . .

1.22 Turkey . . . . . . . . . . .

1.23 Ukraine . . . . . . . . . .

1.24 United Kingdom . . . . .

1.25 United States of America 1.26 Vietnam . . . . . . . . . .

2

3 7 24 27 31 34 38 40 44 47 52 55 59 62 65 73 78 86 . . . . . . . . . . . . . . . . . . 105

. . . . . . . . . . . . . . . . . . 108

. . . . . . . . . . . . . . . . . . 111

. . . . . . . . . . . . . . . . . . 118

. . . . . . . . . . . . . . . . . . 121

. . . . . . . . . . . . . . . . . . 127

. . . . . . . . . . . . . . . . . . 130

. . . . . . . . . . . . . . . . . . 136

1997 Regional Contests: Solutions 141

2.1 Asian Pacific Mathematics Olympiad . . . . . . . . . . . . . 141

2.2 Austrian-Polish Mathematical Competition . . . . . . . . . 145

2.3 Czech-Slovak Match . . . . . . . . . . . . . . . . . . . . . . 149

2.4 Hungary-Israel Mathematics Competition . . . . . . . . . . 153

2.5 Iberoamerican Mathematical Olympiad . . . . . . . . . . . 156

2.6 Nordic Mathematical Contest . . . . . . . . . . . . . . . . . 161

2.7 Rio Plata Mathematical Olympiad . . . . . . . . . . . . . . 163

2.8 St. Petersburg City Mathematical Olympiad (Russia) . . . 166

3 1998 National Contests: Problems

180

3.1 Bulgaria . . . . . . . . . .

3.2 Canada . . . . . . . . . .

3.3 China . . . . . . . . . . .

3.4 Czech and Slovak Republics 3.5 Hungary . . . . . . . . . .

3.6 India . . . . . . . . . . . .

3.7 Iran . . . . . . . . . . . .

3.8 Ireland . . . . . . . . . . .

3.9 Japan . . . . . . . . . . .

3.10 Korea . . . . . . . . . . .

3.11 Poland . . . . . . . . . . .

3.12 Romania . . . . . . . . . .

3.13 Russia . . . . . . . . . . .

3.14 Taiwan . . . . . . . . . . .

3.15 Turkey . . . . . . . . . . .

3.16 United Kingdom . . . . .

3.17 United States of America 3.18 Vietnam . . . . . . . . . .

4

. . . . . . . . . . . . . . . . . . 180

. . . . . . . . . . . . . . . . . . 183

. . . . . . . . . . . . . . . . . . 184

. . . . . . . . . . . . . . . . . . 185

. . . . . . . . . . . . . . . . . . 186

. . . . . . . . . . . . . . . . . . 188

. . . . . . . . . . . . . . . . . . 190

. . . . . . . . . . . . . . . . . . 193

. . . . . . . . . . . . . . . . . . 195

. . . . . . . . . . . . . . . . . . 196

. . . . . . . . . . . . . . . . . . 197

. . . . . . . . . . . . . . . . . . 198

. . . . . . . . . . . . . . . . . . 200

. . . . . . . . . . . . . . . . . . 207

. . . . . . . . . . . . . . . . . . 208

. . . . . . . . . . . . . . . . . . 209

. . . . . . . . . . . . . . . . . . 211

. . . . . . . . . . . . . . . . . . 212

1998 Regional Contests: Problems 213

4.1 Asian Pacific Mathematics Olympiad . . . . . . . . . . . . . 213

4.2 Austrian-Polish Mathematics Competition . . . . . . . . . . 214

4.3 Balkan Mathematical Olympiad . . . . . . . . . . . . . . . . 216

4.4 Czech-Slovak Match . . . . . . . . . . . . . . . . . . . . . . 217

4.5 Iberoamerican Olympiad . . . . . . . . . . . . . . . . . . . . 218

4.6 Nordic Mathematical Contest . . . . . . . . . . . . . . . . . 219

4.7 St. Petersburg City Mathematical Olympiad (Russia) . . . 220

1 1997 National Contests: Solutions

1.1 Austria

1. Solve the system for $x, y$ real:
$$
(x-1)(y^{2}+6)\ =\ y(x^{2}+1)
$$
$$
(y-1)(x^{2}+6)\ =\ x(y^{2}+1)\ .
$$
Solution: We begin by adding the two given equations together. After simplifying the resulting equation and completing the square, we arrive at the following equation:
\begin{center}
$(x-5/2)^{2}+(y-5/2)^{2}=1/2$.   (1)
\end{center}
We can also subtract the two equations; subtracting the second given equation from the first and grouping, we have:
$$
xy(y-x)+6(x-y)+(x+y)(x-y)\ =\ xy(x-y)+(y-x)
$$
$$
(x-y)(-xy+6+(x+y)\ - xy\ +1)\ =\ 0
$$
$$
(x-y)(x+y-2xy+7)\ =\ 0
$$
Thus, either $x-y=0$ or $x+y-2xy+7=0$. The only ways to have $x-y=0$ are with $x=y=2$ or $x=y=3$ (found by solving equation (1) with the substitution $x=y$).

Now, all solutions to the original system where $x\neq y$ will be solutions to $x+y-2xy+7=0$. This equation is equivalent to the following equation (derived by rearranging terms and factoring).
\begin{center}
$(x-1/2)(y-1/2)=15/4$.   (2)
\end{center}
Let us see if we can solve equations (1) and (2) simultaneously. Let $a=x-5/2$ and $b=y-5/2$. Then, equation (1) is equivalent to:
\begin{center}
$a^{2}+b^{2}=1/2$   (3)
\end{center}
and equation (2) is equivalent to:

$(a+2)(b+2)=15/4\Rightarrow ab+2(a+b)=-1/4\Rightarrow 2ab+4(a+b)=-1/2.$ (4)

2

Adding equation (4) to equation (3), we find:
\begin{center}
$(a+b)^{2}+4(a+b)=0\Rightarrow a+b=0,\ -4$   (5)
\end{center}
Subtracting equation (4) from equation (3), we find:
\begin{center}
$(a-b)^{2}-4(a+b)=1$.   (6)
\end{center}
But now we see that if $a+b=-4$, then equation (6) will be false; thus, $a+b=0$. Substituting this into equation (6), we obtain:
\begin{center}
$(a-b)^{2}=1\Rightarrow a-b=\pm 1$   (7)
\end{center}
Since we know that $a+b=0$ from equation (5), we now can find all ordered pairs $(a,\ b)$ with the help of equation (7). They are (-1/2, 1/2) and $(1/2,\ -1/2)$ . Therefore, our only solutions $(x,\ y)$ are (2, 2), (3, 3), (2, 3), and (3, 2).

2. Consider the sequence of positive integers which satisfies $a_{n}=a_{n-1}^{2}+ a_{n-2}^{2}+a_{n-3}^{2}$ for all $n\geq 3$. Prove that if $a_{k}=1997$ then $k\leq 3.$

Solution: We proceed indirectly; assume that for some $k>3, a_{k}=$ 1997. Then , each of the four numbers $a_{k-1}, a_{k-2}, a_{k-3},$ and $a_{k-4}$ must exist. Let $w=a_{k-1}, x=a_{k-2}, y=a_{k-3}$, and $z=a_{k-4}$. Now, by the given condition, $1997=w^{2}+x^{2}+y^{2}$. Thus, $w\leq\sqrt{1997}<45$, and since $w$ is a positive integer, $w\leq 44$. But then $x^{2}+y^{2}\geq 1997-44^{2}=61.$

Now, $w=x^{2}+y^{2}+z^{2}$. Since $x^{2}+y^{2}\geq 61$ and $z^{2}\geq 0, x^{2}+y^{2}+ z^{2}\geq 61$. But $w\leq 44$. Therefore, we have a contradiction and our assumption was incorrect.

If $a_{k}=1997$, then $k\leq 3.$

3. Let $k$ be a positive integer. The sequence $a_{n}$ is defined by $a_{1}=1$, and $a_{n}$ is the {\it n}-th positive integer greater than $a_{n-1}$ which is congruent to $n$ modulo $k$. Find $a_{n}$ in closed form.

Solution: We have $a_{n}=n(2+(n-1)k)/2$. If $k=2$, then $a_{n}=n^{2}.$ First, observe that $a_{1}\equiv 1 (\mathrm{m}\mathrm{o}\mathrm{d}\ k)$ . Thus, for all $n, a_{n}\equiv n (\mathrm{m}\mathrm{o}\mathrm{d}\ k)$ , and the first positive integer greater than $a_{n-1}$ which is congruent to $n$ modulo $k$ must be $a_{n-1}+1.$

3

The {\it n}-th positive integer greater than $a_{n-1}$ that is congruent to $n$ modulo $k$ is simply $(n-1)k$ more than the first positive integer greater than $a_{n-1}$ which satisfies that condition. Therefore, $a_{n}= a_{n-1}+1+(n-1)k$. Solving this recursion gives the above answer.

4. Given a parallelogram {\it ABCD}, inscribe in the angle $\angle BAD$ a circle that lies entirely inside the parallelogram. Similarly, inscribe a circle in the angle $\angle BCD$ that lies entirely inside the parallelogram and such that the two circles are tangent. Find the locus of the tangency point of the circles, as the two circles vary.

Solution: Let $K_{1}$ be the largest circle inscribable in $\angle BAD$ such that it is completely inside the parallelogram. It intersects the line $AC$ in two points; let the point farther from $A$ be $P_{1}$. Similarly, let $K_{2}$ be the largest circle inscribable in $\angle BCD$ such that it is completely inside the parallelogram. It intersects the line $AC$ in two points; let the point farther from $C$ be $P_{2}$. then the locus is the intersection of the segments $AP_{1}$ and $CP_{2}.$

We begin by proving that the tangency point must lie on line $AC.$ Let $I_{1}$ be the center of the circle inscribed in $\angle BAD$. Let $I_{2}$ be the center of the circle inscribed in $\angle BCD$. Let $X$ represent the tangency point of the circles.

Since circles $I_{1}$ and $I_{2}$ are inscribed in angles, these centers must lie on the respective angle bisectors. Then, since $AI_{1}$ and $CI_{2}$ are bisectors of opposite angles in a parallelogram, they are parallel; therefore, since $I_{1}I_{2}$ is a transversal, $\angle AI_{1}X=\angle CI_{2}X.$

Let $T_{1}$ be the foot of the perpendicular from $I_{1}$ to $AB$. Similarly, let $T_{2}$ be the foot of the perpendicular from $I_{2}$ to $CD$. Observe that $I_{1}T_{1}/AI_{1}=\sin\angle I_{1}AB=\sin\angle I_{2}CD=I_{2}T_{2}/CI_{2}$. But $I_{1}X=I_{1}T_{1}$ and $I_{2}X=I_{2}T_{2}$. Thus, $I_{1}X/AI_{1}=I_{2}X/CI_{2}.$

Therefore, triangles $CI_{2}X$ and $AI_{1}X$ are similar, and vertical angles $\angle I_{1}XA$ and $\angle I_{2}XC$ are equal. Since these vertical angles are equal, the points $A, X$, and $C$ must be collinear.

The tangency point, $X$, thus lies on diagonal $AC$, which was what we wanted.

Now that we know that $X$ will always lie on $AC$, we will prove that any point on our locus can be a tangency point. For any $X$ on our

4

locus, we can let circle $I_{1}$ be the smaller circle through $X$, tangent to the sides of $\angle BAD.$

It will definitely fall inside the parallelogram because $X$ is between $A$ and $P_{1}$. Similarly, we can draw a circle tangent to circle $I_{1}$ and to the sides of $\angle BCD$; from our proof above, we know that it must be tangent to circle $I_{1}$ at $X$. Again, it will definitely fall in the parallelogram because $X$ is between $C$ and $P_{2}.$

Thus, any point on our locus will work for $X$. To prove that any other point will not work, observe that any other point would either not be on line $AC$ or would not allow one of the circles $I_{1}$ or $I_{2}$ to be contained inside the parallelogram.

Therefore, our locus is indeed the intersection of segments $AP_{1}$ and $CP_{2}.$

1.2 Bulgaria

1. Find all real numbers $m$ such that the equation
$$
(x^{2}-2mx-4(m^{2}+1))(x^{2}-4x-2m(m^{2}+1))=0
$$
has exactly three different roots.

Solution: Answer: $m=3$. Proof: By setting the two factors on the left side equal to $0$ we obtain two polynomial equations, at least one of which must be true for some $x$ in order for $x$ to be a root of our original equation. These equations can be rewritten as $(x-m)^{2}=5m^{2}+4$ and $(x-2)^{2}=2(m^{3}+m+2)$ . We have three ways that the original equation can have just three distinct roots: either the first equation has a double root, the second equation has a double root, or there is one common root of the two equations.The first case is out, however, because this would imply $5m^{2}+4=0$ which is not possible for real $m.$

In the second case, we must have $2(m^{3}+m+2)=0;m^{3}+m+2$ factors as $(m+1)(m^{2}-m+2)$ and the second factor is always positive for real $m$. So we would have to have $m=-1$ for this to occur. Then the only root of our second equation is $x=2$, and our first equation becomes $(x+1)^{2}=9$, i.e. $x=2, -4$. But this means our original equation had only 2 and -$4$ as roots, contrary to intention.

In our third case let $r$ be the common root, so $x-r$ is a factor of both $x^{2}-2mx-4(m^{2}+1)$ and $x^{2}-4x-2m(m^{2}+1)$ . Subtracting, we get that $x-r$ is a factor of $(2m-4)x-(2m^{3}-4m^{2}+2m-4)$ , i.e. $(2m-4)r=(2m-4)(m^{2}+1)$ . So $m=2$ or $r=m^{2}+1$. In the former case, however, both our second-degree equations become $(x-2)^{2}= 24$ and so again we have only two distinct roots. So we must have $r=m^{2}+1$ and then substitution into $(r-2)^{2}=2(m^{3}+m+2)$ gives $(m^{2}-1)^{2}=2(m^{3}+m+2)$ , which can be rewritten and factored as $(m+1)(m-3)(m^{2}+1)=0$. So $m=-1$ or 3; the first case has already been shown to be spurious, so we can only have $m=3.$ Indeed, our equations become $(x-3)^{3}=49$ and $(x-2)^{2}=64$ so $x=-6, -4,10$, and indeed we have 3 roots.

2. Let {\it ABC} be an equilateral triangle with area 7 and let $M, N$ be points on sides {\it AB}, $AC$, respectively, such that $AN=BM$. Denote

6

by $O$ the intersection of $BN$ and $CM$. Assume that triangle {\it BOC} has area 2.

(a) Prove that $MB/AB$ equals either 1/3 or 2/3.

(b) Find $\angle AOB.$

Solution:

(a) Let $L$ be on $BC$ with $CL=AN,$ and let the intersections of $CM$ and $AL, AL$ and $BN$ be $P, Q$, respectively. A 120-degree rotation about the center of {\it ABC} takes $A$ to $B, B$ to $C, C$ to $A$; this same rotation then also takes $M$ to $L, L$ to $N, N$ to $M$, and also $O$ to $P, P$ to $Q, Q$ to $O$. Thus {\it OPQ} and {\it MLN} are equilateral triangles concentric with {\it ABC}. It follows that $\angle BOC=\pi-\angle NOC=2\pi/3$, so $O$ lies on the reflection of the circumcircle of {\it ABC} through $BC$. There are most two points $O$ on this circle and inside of triangle {\it ABC} such that the ratio of the distances to $BC$ from $O$ and from $A$ -- i.e. the ratio of the areas of triangles {\it OBC} and {\it ABC}--can be 2/7; so once we show that $MB/AB=1/3$ or 2/3 gives such positions of $O$ it will follow that there are no other such ratios (no two points $M$ can give the same $O$, since it is easily seen that as $M$ moves along {\it AB}, $O$ varies monotonically along its locus). If $MB/AB= 1/3$ then $AN/AC=1/3$, and Menelaus' theorem in triangle {\it ABN} and line $CM$ gives $BO/ON=3/4$ so $[BOC]/[BNC]= BO/BN=3/7$. Then $[BOC]/[ABC]=(3/7)(CN/CA)= 2/7$ as desired. Similarly if $MB/AB=2/3$ the theorem gives us $BO/BN=6$, so $[BOC]/[BNC]=BO/BN=6/7$ and
$$
[BOC]/[ABC]=(6/7)(CN/AC)=2/7.
$$
(b) If $MB/AB=1/3$ then {\it MONA} is a cyclic quadrilateral since $\angle A=\pi/3$ and $\angle O=\pi-(\angle POQ)=2\pi/3$. Thus $\angle AOB= \angle AOM+\angle MOB=\angle ANM+\angle POQ=\angle ANM+\pi/3$. But $MB/AB=1/3$ and $AN/AC=1/3$ easily give that $N$ is the projection of $M$ onto $AC$, so $\angle ANM=\pi/2$ and $\angle AOB=$
$$
5\pi/6.
$$
If $MB/AB=2/3$ then {\it MONA} is a cyclic quadrilateral as before, so that $\angle AOB=\angle AOM+\angle MOB=\angle ANM+\angle POQ.$ But {\it AMN} is again a right triangle, now with right angle at $M,$ and $\angle MAN=\pi/3$ so $\angle ANM=\pi/6$, so $\angle AOB=\pi/2.$

7

3. Let $f(x)=x^{2}-2ax-a^{2}-3/4$. Find all values of $a$ such that $|f(x)|\leq 1$ for all $x\in[0,1].$

Solution: Answer: $-1/2\leq a\leq\sqrt{2}/4$. Proof: The graph of $f(x)$ is a parabola with an absolute minimum (i.e., the leading coefficient is positive), and its vertex is $(a,\ f(a))$ . Since $f(0)=-a^{2}-3/4$, we obtain that $|a|\leq 1/2$ if we want $f(0)\geq-1$. Now suppose $a\leq 0$; then our parabola is strictly increasing between $x=0$ and $x=1$ so it suffices to check $f(1)\leq 1$. But we have $ 1/2\leq a+1\leq 1,1/4\leq (a+1)^{2}\leq 1,1/4\leq 5/4-(a+1)^{2}\leq 1$. Since $5/4-(a+1)^{2}=f(1)$ , we have indeed that $f$ meets the conditions for $-1/2\leq a\leq 0$. For $a>0, f$ decreases for $0\leq x\leq a$ and increases for $a\leq x\leq 1$. So we must check that the minimum value $f(a)$ is in our range, and that $f(1)$ is in our range. This latter we get from $1<(a+1)^{2}\leq 9/4$ (since $a\leq 1/2$) and so $f(x)=-1\leq 5/4-(a+1)^{2}<1/4$. On the other hand, $f(a)=-2a^{2}-3/4$, so we must have $a\leq\sqrt{2}/4$ for $f(a)\geq-1$. Conversely, by bounding $f(0), f(a), f(1)$ we have shown that $f$ meets the conditions for $0<a\leq\sqrt{2}/4.$

4. Let $I$ and $G$ be the incenter and centroid, respectively, ofa triangle {\it ABC} with sides $AB=c, BC=a, CA=b.$

(a) Prove that the area of triangle {\it CIG} equals $|a-b|r/6$, where $r$ is the inradius of {\it ABC}.

(b) If $a=c+1$ and $b=c-1$, prove that the lines $IG$ and $AB$ are parallel, and find the length of the segment $IG.$

Solution:

(a) Assume WLOG $a>b$. Let $CM$ be a median and $CF$ be the bisector of angle $C$; let $S$ be the area of triangle {\it ABC}. Also let $BE$ be the bisector of angle $B$; by Menelaus' theorem on line $BE$ and triangle {\it ACF} we get $(CE/EA)(AB/BF)(FI/IC)= 1$. Applying the Angle Bisector Theorem twice in triangle {\it ABC} we can rewrite this as $(a/c)((a+b)/a)(FI/IC)=1$, or $IC/FI=(a+b)/c$, or $IC/CF=(a+b)/(a+b+c)$ . Now also by the Angle Bisector Theorem we have $BF=ac/(a+b)$ ; since $BM=c/2$ and $a>b$ then $MF=(a-b)c/2(a+b)$ . So comparing triangles {\it CMF} and {\it ABC}, noting that the altitudes

8

to side $MF$ (respectively {\it AB}) are equal, we have $[CMF]/S= (a-b)/2(a+b)$ . Similarly using altitudes from $M$ in triangles {\it CMI} and {\it CMF} (and using the ratio $IC/CF$ found earlier), we have $[CMI]/S=(a-b)/2(a+b+c)$ ; and using altitudes from $I$ in triangles {\it CGI} and {\it CMI} gives (since $CG/CM=2/3$) $[CGI]/S=(a-b)/3(a+b+c)$ . Finally $S=(a+b+c)r/2$ leads

to [{\it CGI}] $=(a-b)r/6.$

(b) As noted earlier, $IC/CF=(a+b)/(a+b+c)=2/3= CG/CM$ in the given case. But $C, G, M$ are collinear, as are $C, I, F$, giving the desired parallelism (since line $MF=$ line {\it AB}). We found earlier $MF=(a-b)c/2(a+b)=1/2$, so
$$
GI=(2/3)(MF)=1/3.
$$
5. Let $n\geq 4$ be an even integer and $A$ a subset of $\{$1, 2, . . . , $n\}$. Consider the sums $e_{1}x_{1}+e_{2}x_{2}+e_{3}x_{3}$ such that:
$$
\bullet x_{1},\ x_{2},\ x_{3}\in A;
$$
$$
\bullet\ e_{1},\ e_{2},\ e_{3}\in\{-1,0,1\};
$$
$\bullet$ at least one of $e_{1}, e_{2}, e_{3}$ is nonzero;

$\bullet$ if $x_{i}=x_{j}$, then $e_{i}e_{j}\neq-1.$

The set $A$ is {\it free} if all such sums are not divisible by $n.$

(a) Find a free set of cardinality $\lfloor n/4\rfloor.$

(b) Prove that any set of cardinality $\lfloor n/4\rfloor+1$ is not free.

Solution:

(a) We show that the set $A=\{1,3,5,\ 2\lfloor n/4\rfloor-1\}$ is free. Any combination $e_{1}x_{1}+e_{2}x_{2}+e_{3}x_{3}$ with zero or two $e_{i}$'s equal to $0$ has an odd value and so is not divisible by $n$; otherwise, we have one $e_{i}$ equal to $0$, so we have either a difference of two distinct elements of $A$, which has absolute value less than $ 2\lfloor n/4\rfloor$ and cannot be $0$, so it is not divisible by $n$, or a sum (or negative sum) of two elements, in which case the absolute value must range between 2 and $4\lfloor n/4\rfloor-2<n$ and so again is not divisible by $n.$

(b) Suppose $A$ is a free set; we will show $|A|\leq\lfloor n/4\rfloor$. For any $k, k$ and $n-k$ cannot both be in $A$ since their sum is $n$; likewise, $n$ and $n/2$ cannot be in $A$. If we change any element $k$ of $A$ to $n-k$ then we can verify that the set of all combinations $\displaystyle \sum e_{i}x_{i}$ taken $\mathrm{m}\mathrm{o}\mathrm{d}\ n$ is invariant, since we can simply flip the sign of any $e_{i}$ associated with the element $k$ in any combination. Hence we may assume that $A$ is a subset of $B=\{1,2,\ n/2-1\}.$ Let $d$ be the smallest element of $A$. We group all the elements of $B$ greater than $d$ into �packages� of at most $2d$ elements, starting with the largest; i.e. we put the numbers from $n/2- 2d$ to $n/2-1$ into one package, then put the numbers from $n/2-4d$ to $n/2-2d-1$ into another, and so forth, until we hit $d+1$ and at that point we terminate the packaging process. All our packages, except possibly the last, have $2d$ elements; so let $p+1$ be the number of packages and let $r$ be the number of elements in the last package (assume $p\geq 0$, since otherwise we have no packages and $d=n/2-1$ so our desired conclusion holds because $|A|=1$). The number of elements in $B$ is then $2dp+r+d$, so $n=4dp+2d+2r+2$. Note that no two elements of $A$ can differ by $d$, since otherwise $A$ is not free. Also the only element of $A$ not in a package is $d$, since it is the smallest element and all higher elements of $B$ are in packages.

Now do a case analysis on $r$. If $r<d$ then each complete package has at most $d$ elements in common with $A$, since the elements of any such package can be partitioned into disjoint pairs each with difference $d$. Thus $|A|\leq 1+dp+r$ and $ 4|A|\leq 4dp+4r+4\leq n$ (since $r+1\leq d$) so our conclusion holds. If $r=d$ then each complete package has at most $d$ elements in common with $A$, and also the last package (of $d$ elements) has at most $d-1$ elements in common with $A$ for the following reason: its highest element is $2d$, but $2d$ is not in $A$ since $d+d-2d=0$. So $|A|\leq d(p+1), 4|A|<n$ and our conclusion holds. If $r>d$ then we can form $r-d$ pairs in the last package each of difference $d$, so each contains at most 1 element of $A$, and then there are $2d-r$ remaining elements in this package. So this package contains at most $d$ elements, and the total number of elements in $A$ is at least $d(p+1)+1$, so $4|A|\leq n$ and our conclusion again holds.

10

6. Find the least natural number $a$ for which the equation
$$
\cos^{2}\pi(a-x)-2\cos\pi(a-x)+\cos\frac{3\pi x}{2a}\cos(\frac{\pi x}{2a}+\frac{\pi}{3})+2=0
$$
has a real root.

Solution: The smallest such $a$ is 6. The equation holds if $a= 6, x=8$. To prove minimality, write the equation as
$$
(\cos\pi(a-x)-1)^{2}+(\cos(3\pi x/2a)\cos(\pi x/2a+\pi/3)+1)=0;
$$
since both terms on the left side are nonnegative, equality can only hold if both are $0$. From $\cos\pi(a-x)-1=0$ we get that $x$ is an integer congruent to $a(\mathrm{m}\mathrm{o}\mathrm{d}\ 2)$ . From the second term we see that each cosine involved must be $-1$ or 1 for the whole term to be $0$; if $\cos(\pi x/2a+\pi/3)=1$ then $\pi x/2a+\pi/3=2k\pi$ for some integer $k,$ and multiplying through by $ 6a/\pi$ gives $3x\equiv-2a (\mathrm{m}\mathrm{o}\mathrm{d}\ 12a)$ , while if the cosine is $-1$ then $\pi x/2a+\pi/3=(2k+1)\pi$ and multiplying by $ 6a/\pi$ gives $3x\equiv 4a (\mathrm{m}\mathrm{o}\mathrm{d}\ 12a)$ . In both cases we have $3x$ divisible by 2, so $x$ is divisible by 2 and hence so is $a$. Also our two cases give $-2a$ and $4a$, respectively, are divisible by 3, so $a$ is divisible by 3. We conclude that $6|a$ and so our solution is minimal.

7. Let {\it ABCD} be a trapezoid $(AB||CD)$ and choose $F$ on the segment $AB$ such that $DF=CF$. Let $E$ be the intersection of $AC$ and $BD,$ and let $O_{1}, O_{2}$ be the circumcenters of {\it ADF, BCF}. Prove that the lines $EF$ and $O_{1}O_{2}$ are perpendicular.

Solution: Project each of points $A, B, F$ orthogonally onto $CD$ to obtain $A', B', F'$; then $F'$ is the midpoint of $CD$. Also let the circumcircles of {\it AFD, BFC} intersect line $CD$ again at $M, N$ respectively; then {\it AFMD, BFNC} are isosceles trapezoids and $F'M= DA', NF'=B'C$. Let $x=DA', y=A'F'=AF, z=F'B'=FB, w=B'C$, using signed distances throughout $(x<0$ if $D$ is between $A'$ and $F'$, etc.), so we have $x+y=z+w$; call this value $S$, so $DC=2S$. Also let line $FE$ meet $DC$ at $G$; since a homothety about $E$ with (negative) ratio $CD/AB$ takes triangle {\it ABE} into {\it CDE} it also takes $F$ into $G$, so $DG/GC=FB/AF=F'B'/A'B'=z/y$ and we easily get $DG=2zS/(y+z), GC=2yS/(y+z)$ . Now

11

$NF'=w, DF'=S$ implies $DN=z$ and so $DN/DG=(y+z)/2S.$ Similarly $F'M=x, F'C=S$ so $MC=y$ and $MC/GC=(y+z)/2S$ also. So $DN/DG=MC/GC, NG/DG=GM/GC$ and $NG\cdot GC= DG\cdot GM$. Since $NC$ and $DM$ are the respective chords of the circumcircles of {\it BFC} and {\it ADC} that contain point $G$ we conclude that $G$ has equal powers with respect to these two circles, i.e. it is on the radical axis. $F$ is also on the axis since it is an intersection point of the circles, so the line {\it FGE} is the radical axis, which is perpendicular to the line $O_{1}O_{2}$ connecting the centers of the circles.

8. Find all natural numbers $n$ for which a convex $n$-gon can be divided into triangles by diagonals with disjoint interiors, such that each vertex of the $n$-gon is the endpoint of an even number of the diagonals.

Solution: We claim that $3|n$ is a necessary and sufficient condition. To prove sufficiency, we use induction of step 3. Certainly for $n=3$ we have the trivial dissection (no diagonals drawn). If $n>3$ and $3|n$ then let $A_{1}, A_{2}$, . . . , $A_{n}$ be the vertices of an $n$-gon in counterclockwise order; then draw the diagonals $A_{1}A_{n-3}, A_{n-3}A_{n-1}, A_{n-1}A_{1}$; these three diagonals divide our polygon into three triangles and an $(n-3)$-gon $A_{1}A_{2}$ . . . $A_{n-3}$. By the inductive hypothesis the latter can be dissected into triangles with evenly many diagonals at each vertex, so we obtain the desired dissection of our $n$-gon, since each vertex from $A_{2}$ through $A_{n-4}$ has the same number of diagonals in the $n$-gon as in the $(n-3)$-gon (an even number), $A_{1}$ and $A_{n-3}$ each have two diagonals more than in the $(n-3)$-gon, while $A_{n-1}$ has 2 diagonals and $A_{n}$ and $A_{n-2}$ have $0$ each.

To show necessity, suppose we have such a decomposition of a polygon with vertices $A_{1}, A_{2}$, . . . , $A_{n}$ in counterclockwise order, and for convenience assume labels are $\mathrm{m}\mathrm{o}\mathrm{d}\ n$. Call a diagonal $A_{i}A_{j}$ in our dissection a �right diagonal� from $A_{i}$ if no point $A_{i+2}, A_{i+3}$, . . . , $A_{j-1}$ is joined to $A_{i}$ (we can omit $A_{i+1}$ from our list since it is joined by an edge). Clearly every point from which at least one diagonal emanates has a unique right diagonal. Also we have an important lemma: if $A_{i}A_{j}$ is a right diagonal from $A_{i}$, then within the polygon $A_{i}A_{i+1}$ . . . $A_{j}$, each vertex belongs to an even number of diagonals.

Proof: Each vertex from any of the points $A_{i+1}$, . . . , $A_{j-1}$ belongs to an even number of diagonals of the $n$-gon, but since the diagonals

12

of the $n$-gon are nonintersecting these diagonals must lie within our smaller polygon, so we have an even number of such diagonals for each of these points. By hypothesis, $A_{i}$ is not connected via a diagonal to any other point of this polygon, so we have $0$ diagonals from $A_{i}$, an even number. Finally evenly many diagonals inside this polygon stem from $A_{j}$, since otherwise we would have an odd number of total endpoints of all diagonals.

Now we can show $3|n$ by strong induction on $n$. If $n=1$ or 2, then there is clearly no decomposition, while if $n=3$ we have $3|n$. For $n>3$ choose a vertex $A_{i_{1}}$ with some diagonal emanating from it, and let $A_{i_{1}}A_{i_{2}}$ be the right diagonal from $A_{i_{1}}$. By the lemma there are evenly many diagonals from $A_{i_{2}}$ with their other endpoints in $\{A_{i_{1}+1},\ A_{i_{1}+2},\ .\ .\ .\ ,\ A_{i_{2}-1}\}$, and one diagonal $A_{i_{1}}A_{i_{2}}$, so there must be at least one other diagonal from $A_{i_{2}}$ (since the total number of diagonals there is even). This implies $A_{i_{1}}A_{i_{2}}$ is not the right diagonal from $A_{i_{2}}$ , so choose the right diagonal $A_{i_{2}}A_{i_{3}}$. Along the same lines we can choose the right diagonal $A_{i_{3}}A_{i_{4}}$ from $A_{i_{3}}$, with $A_{i_{2}}$ and $A_{i_{4}}$ distinct, then continue with $A_{i_{4}}A_{i_{5}}$ as the right diagonal from $A_{i_{4}},$ etc. Since the diagonals of the $n$-gon are nonintersecting this process must terminate with some $A_{i_{k+1}}=A_{i_{1}}$. Now examine each of the polygons $A_{i_{x}}A_{i_{x}+1}A_{i_{x}+2}$ . . . $A_{i_{x+1}}$ , $x=1,2$, . . . , $k$ (indices $x$ are taken $\mathrm{m}\mathrm{o}\mathrm{d}\ k$). By the lemma each of these polygons is divided into triangles by nonintersecting diagonals with evenly many diagonals at each vertex, so by the inductive hypothesis the number of vertices of each such polygon is divisible by 3. Also consider the polygon $A_{i_{1}}A_{i_{2}}$ . . . $A_{i_{k}}$ . We claim that in this polygon, each vertex belongs to an even number of diagonals. Indeed, from $A_{i_{x}}$ we have an even number of diagonals to points in $A_{i_{x-1}+1}, A_{i_{x-1}+2}$, . . . , $A_{i_{x}-1}$, plus the two diagonals $A_{i_{x-1}}A_{i_{x}}$ and $A_{i_{x}}A_{i_{x+1}}$. This leaves an even number of diagonals from $A_{i_{x}}$ to other points; since $A_{i_{x}}$ was chosen as the endpoint of a right diagonal we have no diagonals lead to points in $A_{i_{x}+1}$, . . . , $A_{i_{x+1}-1}$, so it follows from the nonintersecting criterion that all remaining diagonals must lead to points $A_{i_{y}}$ for some $y.$ Thus we have an even number of diagonals from $A_{i_{x}}$ to points $A_{i_{y}}$ for some fixed $x$; it follows from the induction hypothesis that $3|k.$ So, if we count each vertex of each polygon $A_{i_{x}}A_{i_{x}+1}A_{i_{x}+2}$ . . . $A_{i_{x+1}}$ once and then subtract the vertices of $A_{i_{1}}A_{i_{2}}$ . . . $A_{i_{k}}$, each vertex of our $n$-gon is counted exactly once, but from the above we have been adding and subtracting multiples of 3. Thus we have $3|n.$

13

9. For any real number $b$, let $f(b)$ denote the maximum of the function $|\displaystyle \sin x+\frac{2}{3+\sin x}+b|$

over all $x\in \mathbb{R}$. Find the minimum of $f(b)$ over all $b\in \mathbb{R}.$

Solution: The minimum value is 3/4. Let $y=3+\sin x$; note $y\in[2,4]$ and assumes all values therein. Also let $g(y)=y+2/y$; this function is increasing on [2, 4], so $g(2)\leq g(y)\leq g(4)$ . Thus $3\leq g(y)\leq 9/2$, and both extreme values are attained. It now follows that the minimum of $f(b)=\displaystyle \max(|g(y)+b-3|)$ is 3/4, which is attained by $b=-3/4$; for if $b>-3/4$ then choose $x=\pi/2$ so $y=4$ and then $g(y)+b-3>3/4$, while if $b<-3/4$ then choose $x=-\pi/2$ so $y=2$ and $g(y)+b-3=-3/4$; on the other hand, our range for $g(y)$ guarantees $-3/4\leq g(y)+b-3\leq 3/4$ for $b=-3/4.$

10. Let {\it ABCD} be a convex quadrilateral such that $\angle DAB=\angle ABC= \angle BCD$. Let $H$ and $O$ denote the orthocenter and circumcenter of the triangle {\it ABC}. Prove that $H, O, D$ are collinear.

Solution: Let $M$ be the midpoint of $B$ and $N$ the midpoint of $BC$. Let $E=AB\cap CD$ and $F=BC\cap AD$. Then {\it EBC} and {\it FAB} are isosceles triangles, so $EN\cap FM=0$. Thus applying Pappus's theorem to hexagon {\it MCENAF}, we find that $G, O, D$ are collinear, so $D$ lies on the Euler line of {\it ABC} and $H, O, D$ are collinear.

11. For any natural number $n\geq 3$, let $m(n)$ denote the maximum number of points lying within or on the boundary of a regular $n$-gon of side length 1 such that the distance between any two of the points is greater than 1. Find all $n$ such that $m(n)=n-1.$

Solution: The desired $n$ are 4, 5, 6. We can easily show that $m(3)=1$, e.g. dissect an equilateral triangle {\it ABC} into 4 congruent triangles and then for two points $P, Q$ there is some corner triangle inside which neither lies; if we assume this corner is at $A$ then the circle with diameter $BC$ contains the other three small triangles and so contains $P$ and $Q;BC=1$ so $PQ\leq 1$. This method will be useful later; call it a lemma.

14

On the other hand, $m(n)\geq n-1$ for $n\geq 4$ as the following process indicates. Let the vertices of our $n$-gon be $A_{1}, A_{2}$, . . . , $A_{n}$. Take $P_{1}=A_{1}$. Take $P_{2}$ on the segment $A_{2}A_{3}$ at an extremely small distance $d_{2}$ from $A_{2}$; then $P_{2}P_{1}>1$, as can be shown rigorously, e.g. using the Law of Cosines in triangle $P_{1}A_{2}P_{2}$ and the fact that the cosine of the angle at $A_{2}$ is nonnegative (since $n\geq 4$). Moreover $P_{2}$ is on a side of the $n$-gon other than $A_{3}A_{4}$, and it is easy to see that as long as $n\geq 4$, the circle of radius 1 centered at $A_{4}$ intersects no side of the $n$-gon not terminating at $A_{4}$, so $P_{2}A_{4}>1$ while clearly $P_{2}A_{3}<1$. So by continuity there is a point $P_{3}$ on the side $A_{3}A_{4}$ with $P_{2}P_{3}=1$. Now slide $P_{3}$ by a small distance $d_{3}$ on $A_{3}A_{4}$ towards $A_{4}$; another trigonometric argument can easily show that then $P_{2}P_{3}>1.$ Continuing in this manner, obtain $P_{4}$ on $A_{4}A_{5}$ with $P_{3}P_{4}=1$ and slide $P_{4}$ by distance $d_{4}$ so that now $P_{3}P_{4}>1$, etc. Continue doing this until all points $P_{i}$ have been defined; distances $P_{i}P_{i+1}$ are now greater than by construction, $P_{n-1}P_{1}>1$ because $P_{1}=A_{1}$ while $P_{n-1}$ is in the interior of the side $A_{n-1}A_{n}$; and all other $P_{i}P_{j}$ are greater than 1 because it is easy to see that the distance between any two points of nonadjacent sides of the $n$-gon is at least 1 with equality possible only when (among other conditions) $P_{i}, P_{j}$ are endpoints of their respective sides, and in our construction this never occurs for distinct $i,j$. So our construction succeeds. Moreover, as all the distances $d_{i}$ tend to $0$ each $P_{i}$ tends toward $A_{i}$, so it follows that the maximum of the distances $A_{i}P_{i}$ can be made as small as desired by choosing $d_{i}$ sufficiently small. On the other hand, when $n>6$ the center $O$ of the $n$-gon is at a distance greater than 1 from each vertex, so if the $P_{i}$ are sufficiently close to the $A_{i}$ then we will also have $OP_{i}>1$ for each $i$. Thus we can add the point $O$ to our set, showing that $m(n)\geq n$ for $n>6.$

It now remains to show that we cannot have more than $n-1$ points at mutual distances greater than 1 for $n=4,5,6$. As before let the vertices of the polygon be $A_{1}$, etc. and the center $O$; suppose we have $n$ points $P_{1}$, . . . , $P_{n}$ with $P_{i}P_{j}>1$ for $i$ not equal to $j$. Since $n\leq 6$ it follows that the circumradius of the polygon is not greater than 1, so certainly no $P_{i}$ can be equal to $O$. Let the ray from $O$ through $P_{i}$ intersect the polygon at $Q_{i}$ and assume WLOG our numbering is such that $Q_{1}, Q_{2}$, . . . , $Q_{n}$ occur in that order around the polygon, in the same orientation as the vertices were numbered. Let $Q_{1}$ be on the side $A_{k}A_{k+1}$. A rotation by angle $2\pi/n$ brings $A_{k}$ into $A_{k+1}$; let it

15

also bring $Q_{1}$ into $Q_{1}'$, so triangles $Q_{1}Q_{1}'O$ and $A_{k}A_{k+1}O$ are similar. We claim $P_{2}$ cannot lie inside or on the boundary of quadrilateral $OQ_{1}A_{k+1}Q_{1}'$. To see this, note that $P_{1}Q_{1}A_{k+1}$ and $P_{1}A_{k+1}Q_{1}'$ are triangles with an acute angle at $P_{1}$, so the maximum distance from $P_{1}$ to any point on or inside either of these triangles is attained when that point is some vertex; however $P_{1}Q_{1}\leq OQ_{1}\leq 1$, and $P_{1}A_{k+1}\leq O_{1}A_{k+1}\leq 1$ (e.g. by a trigonometric argument similar to that mentioned earlier), and as for $P_{1}Q_{1}'$, it is subsumed in the following case: we can show that $P_{1}P\leq 1$ for any $P$ on or inside $OQ_{1}Q_{1}'$, because $n\leq 6$ implies that $\angle Q_{1}OQ_{1}'=2\pi/n\geq\pi/3$, and so we can erect an equilateral triangle on $Q_{1}Q_{1}'$ which contains $O$, and the side of this triangle is less than $A_{k}A_{k+1}=1$ (by similar triangles $OA_{k}A_{k+1}$ and $OQ_{1}Q_{1}'$) so we can apply the lemma now to show that two points inside this triangle are at a distance at most 1. The result of all this is that $P_{2}$ is not inside the quadrilateral $OQ_{1}A_{k+1}Q_{1}'$, so that $\angle P_{1}OP_{2}=\angle Q_{1}OP_{2}>2\pi/n$. On the other hand, the label $P_{1}$ is not germane to this argument; we can show in the same way that $\angle P_{i}OP_{i+1}>2\pi/n$ for any $i$ (where $P_{n+1}=P_{1}$). But then adding these $n$ inequalities gives $ 2\pi>2\pi$, a contradiction, so our points $P_{i}$ cannot all exist. Thus $m(n)\leq n-1$ for $n=4,5,6$, completing the proof.

12. Find all natural numbers $a, b, c$ such that the roots of the equations
$$
x^{2}-2ax+b\ =\ 0
$$
$$
x^{2}-2bx+c\ =\ 0
$$
$$
x^{2}-2cx+a\ =\ 0
$$
are natural numbers.

Solution: We have that $a^{2}-b, b^{2}-c, c^{2}-a$ are perfect squares. Since $a^{2}-b\leq(a-1)^{2}$, we have $b\geq 2a-1$; likewise $c\geq 2b-1,  a\geq 2c-1$. Putting these together gives $a\geq 8a-7$, or $a\leq 1$. Thus $(a,\ b,\ c)=(1,1,1)$ is the only solution.

13. Given a cyclic convex quadrilateral {\it ABCD}, let $F$ be the intersection of $AC$ and $BD$, and $E$ the intersection of $AD$ and $BC$. Let $M, N$ be the midpoints of {\it AB}, $CD$. Prove that
$$
\frac{MN}{EF}=\frac{1}{2}|\frac{AB}{CD}-\frac{CD}{AB}|\ .
$$
16

Solution: Since {\it ABCD} is a cyclic quadrilateral, $AB$ and $CD$ are antiparallel with respect to the point $E$, so a reflection through the bisector of $\angle AEB$ followed by a homothety about $E$ with ratio $AB/CD$ takes $C, D$ into $A, B$ respectively. Let $G$ be the image of $F$ under this transformation. Similarly, reflection through the bisector of $\angle AEB$ followed by homothety about $E$ with ratio $CD/AB$ takes $A, B$ into $C, D$; let $H$ be the image of $F$ under this transformation. $G, H$ both lie on the reflection of line $EF$ across the bisector of $\angle AEB$, so $GH=|EG-EH|=EF|AB/CD-CD/AB|.$ On the other hand, the fact that {\it ABCD} is cyclic implies (e.g. by power of a point) that triangles {\it ABF} and {\it DCF} are similar with ratio $AB/CD$. But by virtue of the way the points $A, B, G$ were shown to be obtainable from $C, D, F$, we have that {\it BAG} is also similar to {\it DCF} with ratio $AB/CD$, so {\it ABF} and {\it BAG} are congruent. Hence $AG=BF, AF=BG$ and {\it AGBF} is a parallelogram. So the midpoints of the diagonals of {\it AGBF} coincide, i.e. $M$ is the midpoint of $GF$. Analogously (using the parallelogram {\it CHDF}) we can show that $N$ is the midpoint of $HF$. But then $MN$ is the image of $GH$ under a homothety about $F$ with ratio 1/2, so $MN=GH/2=(EF/2)|AB/CD-CD/AB|$ which is what we wanted to prove.

14. Prove that the equation
$$
x^{2}+y^{2}+z^{2}+3(x+y+z)+5=0
$$
has no solutions in rational numbers.

Solution: Let $u=2x+3, v=2y+3, w=2z+3$. Then the given equation is equivalent to
$$
u^{2}+v^{2}+w^{2}=7.
$$
It is equivalent to ask that the equation
$$
x^{2}+y^{2}+z^{2}=7w^{2}
$$
has no nonzero solutions in integers; assume on the contrary that $(x,\ y,\ z,\ w)$ is a nonzero solution with $|w|+|x|+|y|+|z|$ minimal.

17

Modulo 4, we have $x^{2}+y^{2}+z^{2}\equiv 7w^{2}$, but every perfect square is congruent to $0$ or 1 modulo 4. Thus we must have $x, y, z, w$ even, and $(x/2,\ y/2,\ z/2,\ w/2)$ is a smaller solution, contradiction.

15. Find all continuous functions $f$ : $\mathbb{R}\rightarrow \mathbb{R}$ such that for all $x\in \mathbb{R},$
$$
f(x)=f(x^{2}+\frac{1}{4})\ .
$$
Solution: Put $g(x)=x^{2}+1/4$. Note that if $-1/2\leq x\leq 1/2$, then $x\leq g(x)\leq 1/2$. Thus if $-1/2\leq x_{0}\leq 1/2$ and $x_{n+1}=g(x_{n})$ for $n\geq 0$, the sequence $x_{n}$ tends to a limit $L>0$ with $g(L)=L$; the only such $L$ is $L=1/2$. By continuity, the constant sequence $f(x_{n})$ tends to $f(1/2)$ . In short, $f$ is constant over [-1/2, 1/2].

Similarly, if $x\geq 1/2$, then $1/2\leq g(x)\leq x$, so analogously $f$ is constant on this range. Moreover, the functional equation implies $f(x)=f(-x)$ . We conclude $f$ must be constant.

16. Two unit squares $K_{1}, K_{2}$ with centers $M, N$ are situated in the plane so that $MN=4$. Two sides of $K_{1}$ are parallel to the line $MN$, and one of the diagonals of $K_{2}$ lies on $MN$. Find the locus of the midpoint of $XY$ as $X, Y$ vary over the interior of $K_{1}, K_{2}$, respectively.

Solution: Introduce complex numbers with $M=-2, N=2.$ Then the locus is the set of points of the form $-(w+xi)+(y+zi)$ , where $|w|, |x|<1/2$ and $|x+y|, |x-y|<\sqrt{2}/2$. The result is an octagon with vertices $(1+\sqrt{2})/2+i/2,1/2+(1+\sqrt{2})i/2$, and so on.

17. Find the number of nonempty subsets of $\{$1, 2, . . . , $n\}$ which do not contain two consecutive numbers.

Solution: If $F_{n}$ is this number, then $F_{n}=F_{n-1}+F_{n-2}$: such a subset either contains $n$, in which case its remainder is a subset of $\{1,\ .\ .\ .\ ,\ n-2\}$, or it is a subset of $\{1,\ .\ .\ .\ ,\ n-1\}$. From $F_{1}=1, F_{2}=2,$ we see that $F_{n}$ is the {\it n}-th Fibonacci number.

18. For any natural number $n\geq 2$, consider the polynomial
$$
P_{n}(x)=\left(\begin{array}{l}
n\\
2
\end{array}\right)\ +\ \left(\begin{array}{l}
n\\
5
\end{array}\right)\ x+\ \left(\begin{array}{l}
n\\
8
\end{array}\right)\ x^{2}+\cdots+\ \left(\begin{array}{ll}
 & n\\
3k & +2
\end{array}\right)\ x^{k},
$$
18

where $k=\displaystyle \lfloor\frac{n-2}{3}\rfloor.$

(a) Prove that $P_{n+3}(x)=3P_{n+2}(x)-3P_{n+1}(x)+(x+1)P_{n}(x)$ .

(b) Find all integers $a$ such that $3^{\lfloor(n-1)/2\rfloor}$ divides $P_{n}(a^{3})$ for all
$$
n\geq 3.
$$
Solution:

(a) This is equivalent to the identity (for $0\leq m\leq(n+1)/3$)
$$
\left(\begin{array}{ll}
+n3 & \\
3m & +2
\end{array}\right)=3\ \left(\begin{array}{l}
n+2\\
3m+2
\end{array}\right)-3\ \left(\begin{array}{l}
+n1\\
3m+2
\end{array}\right)+\left(\begin{array}{l}
n\\
3m+2
\end{array}\right)\ +\left(\begin{array}{l}
n\\
3m-1
\end{array}\right),
$$
which follows from repeated use of the identity $\left(\begin{array}{l}
a+1\\
b
\end{array}\right)=\left(\begin{array}{l}
a\\
b
\end{array}\right) + \left(\begin{array}{l}
a\\
b-1
\end{array}\right)$.

(b) If $a$ has the required property, then $P_{5}(a^{3})=10+a^{3}$ is divisible by 9, so $a\equiv-1 (\mathrm{m}\mathrm{o}\mathrm{d}\ 3)$ . Conversely, if $a\equiv-1 (\mathrm{m}\mathrm{o}\mathrm{d}\ 3)$ , then $a^{3}+1\equiv 0 (\mathrm{m}\mathrm{o}\mathrm{d}\ 9)$ . Since $P_{2}(a^{3})=1, P_{3}(a^{3})=3, P_{4}(a^{3})=6$, it follows from (a) that $ 3\lfloor(n-1)/2\rfloor$ divides $P_{n}(a^{3})$ for all $n\geq 3.$

19. Let $M$ be the centroid of triangle {\it ABC}.

(a) Prove that if the line $AB$ is tangent to the circumcircle of the triangle {\it AMC}, then
$$
\sin\angle CAM+\sin\angle CBM\leq\frac{2}{\sqrt{3}}.
$$
(b) Prove the same inequality for an arbitrary triangle {\it ABC}.

Solution:

(a) Let $G$ be the midpoint of {\it AB}, $a, b, c$ the lengths of sides $BC, CA,$ {\it AB}, and $m_{a}, m_{b}, m_{c}$ the lengths of the medians from $A, B, C,$ respectively. We have
$$
(\frac{c}{2})^{2}=GA^{2}=GM\cdot GC=\frac{1}{3}m_{c}^{2}=\frac{1}{12}(2a^{2}+2b^{2}\ - c2)\ ,
$$
29

whence $a^{2}+b^{2}=2c^{2}$ and $m_{a}=\sqrt{3}b/2, m_{b}=\sqrt{3}a/2$. Thus
$$
\sin\angle CAM+\sin\angle CBM=K\frac{1}{bm_{a}}+K\frac{1}{am_{b}}=\frac{(a^{2}+b^{2})\sin C}{\sqrt{3}ab},
$$
where $K$ is the area of the triangle. By the law of cosines, $a^{2}+b^{2}=4ab\cos C$, so the right side is 2 $\sin 2C/\sqrt{3}\leq 2/\sqrt{3}.$

(b) There are two circles through $C$ and $M$ touching $AB$; let $A_{1}, B_{1}$ be the points of tangency, with $A_{1}$ closer to $A$. Since $G$ is the midpoint of $A_{1}B_{1}$ and $CM/MG=2, M$ is also the centroid of triangle $A_{1}B_{1}C$. Moreover, $\angle CAM\leq\angle CA_{1}M$ and $\angle CBM\leq \angle CB_{1}M$. If the angles $\angle CA_{1}M$ and $\angle CB_{1}M$ are acute, we are thus reduced to (a).

It now suffices to suppose $\angle CA_{1}M>90^{\mathrm{o}}, \angle CB_{1}M\leq 90^{\mathrm{o}}.$ Then $CM^{2}>CA_{1}^{2}+A_{1}M^{2}$, that is,
$$
\frac{1}{9}(2b_{1}^{2}+2a_{1}^{2}-c_{1}^{2})>b_{1}^{2}+\frac{1}{9}(2b_{1}^{2}+2c_{1}^{2}-a_{1}^{2})\ ,
$$
where $a_{1}, b_{1}, c_{1}$ are the side lengths of $A_{1}B_{1}C$. From (a), we have $a_{1}^{2}+b_{1}^{2}=c_{1}^{2}$ and the above inequality is equivalent to $a_{1}^{2}>7b_{1}^{2}$. As in (a), we obtain
$$
\sin\angle CB_{1}M=\frac{b_{1}\sin\angle B_{1}CA_{1}}{a_{1}\sqrt{3}}=\frac{b_{1}}{a_{1}\sqrt{3}}\sqrt{1-(\frac{a_{1}^{2}+b_{1}^{2}}{4a_{1}b_{1}})^{2}}.
$$
Setting $b_{1}^{2}/a_{1}^{2}=x$, we get
$$
\sin\angle CB_{1}M=\frac{1}{4\sqrt{3}}\sqrt{14x-x^{2}-1}<\frac{1}{4\sqrt{3}}\sqrt{2-\frac{1}{49}-1}=\frac{1}{7},
$$
since $x<1/7$. Therefore
$$
\sin\angle CAM+\sin\angle CBM<1+\sin\angle CB_{1}M<1+\frac{1}{7}<\frac{2}{\sqrt{3}}.
$$
20. Let $m, n$ be natural numbers and $m+i=a_{i}b_{i}^{2}$ for $i=1,2$, . . . , $n,$ where $a_{i}$ and $b_{i}$ are natural numbers and $a_{i}$ is squarefree. Find all values of $n$ for which there exists $m$ such that $a_{1}+a_{2}+\cdots+a_{n}=12.$

20

Solution: Clearly $n\leq 12$. That means at most three of the $m+i$ are perfect squares, and for the others, $a_{i}\geq 2$, so actually $n\leq 7.$

We claim $a_{i}\neq a_{j}$ for $i=j$. Otherwise, $\mathrm{w}\mathrm{e}' \mathrm{d}$ have $m+i=ab_{i}^{2}$ and $m+j=ab_{j}^{2}$ , so $6\geq n-1\geq(m+j)-(m+i)=a(b_{j}^{2}-b_{i}^{2})$ . This leaves the possibilities $(b_{i},\ b_{j},\ a)=(1,2,2)$ or (2, 3, 1), but both of those force $a_{1}+\cdots+a_{n}>12.$

Thus the $a$'s are a subset of \{1, 2, 3, 5, 6, 7, 10, 11\}. Thus $n\leq 4$, with equality only if $\{a_{1},\ a_{2},\ a_{3},\ a_{4}\}=\{1,2,3,6\}$. But in that case,

$(6b_{1}b_{2}b_{3}b_{4})^{2}=(m+1)(m+2)(m+3)(m+4)=(m^{2}+5m+5)^{2}-1,$

which is impossible. Hence $n=2$ or $n=3$. One checks that the only solutions are then
$$
(m,\ n)=(98,2),\ (3,3)\ .
$$
21. Let $a, b, c$ be positive numbers such that $abc=1$. Prove that
$$
\frac{1}{1+a+b}+\frac{1}{1+b+c}+\frac{1}{1+c+a}\leq\frac{1}{2+a}+\frac{1}{2+b}+\frac{1}{2+c}.
$$
Solution: Brute force! Put $x=a+b+c$ and $y=ab +bc +ca.$ Then the given inequality can be rewritten
$$
\frac{3+4x+y+x^{2}}{2x+y+x^{2}+xy}\leq\frac{12+4x+y}{9+4x+2y},
$$
or
$$
3x^{2}y+xy^{2}+6xy-5x^{2}-y^{2}-24x-3y-27\geq 0,
$$
or

$(3x^{2}y-5x^{2}-12x)+(xy^{2}-y^{2}-3x-3y)+(6xy-9x-27)\geq 0,$

which is true because $x, y\geq 3.$

22. Let {\it ABC} be a triangle and $M, N$ the feet of the angle bisectors of $B, C$, respectively. Let $D$ be the intersection of the ray $MN$ with the circumcircle of {\it ABC}. Prove that
$$
\frac{1}{BD}=\frac{1}{AD}+\frac{1}{CD}.
$$
21

Solution: Let $A_{1}, B_{1}, C_{1}$ be the orthogonal projections of $D$ onto $BC, CA,$ {\it AB}, respectively. Then

$DB_{1}=DA$ sin $\angle DAB_{1}=DA$ sin $\displaystyle \angle DAC=\frac{DA\cdot DC}{2R},$

where $R$ is the circumradius of {\it ABC}. Likewise $DA_{1}=DB\cdot DC/2R$ and $DC_{1}=DA\cdot DB/2R$. Thus it suffices to prove $DB_{1}=DA_{1}+ DC_{1}.$

Let $m$ be the distance from $M$ to $AB$ or $BC$, and $n$ the distance from $N$ to $AC$ or $BC$. Also put $x=DM/MN(x>1)$ . Then
$$
\frac{DB_{1}}{n}=x,\ \frac{DC_{1}}{m}=x-1,\ \frac{DA_{1}-m}{n-m}=x.
$$
Hence $DB_{1}=nx, DC_{1}=m(x-1), DA_{1}=nx-m(x-1)= DB_{1}-DC_{1}$, as desired.

23. Let $X$ be a set of cardinality $n+1(n\geq 2)$ . The ordered {\it n}tuples $(a_{1},\ a_{2},\ .\ .\ .\ ,\ a_{n})$ and $(b_{1},\ b_{2},\ .\ .\ .\ ,\ b_{n})$ of distinct elements of $X$ are called {\it separated} if there exist indices $i\neq j$ such that $a_{i}=b_{j}.$ Find the maximal number of $n$-tuples such that any two of them are separated.

Solution: If $A_{n+1}$ is the maximum number of pairwise separated $n$-tuples, we have $A_{n+1}\leq(n+1)A_{n}$ for $n\geq 4$, since among pairwise separated $n$-tuples, those tuples with a fixed first element are also pairwise separated. Thus $A_{n}\leq n!/2$. To see that this is optimal, take all $n$-tuples $(a_{1},\ .\ .\ .\ ,\ a_{n})$ such that adding the missing member at the end gives an {\it even} permutation of $\{1,\ .\ .\ .\ ,\ n-1\}.$

1.3 Canada

1. How many pairs $(x,\ y)$ of positive integers with $x\leq y$ satisfy $\mathrm{g}\mathrm{c}\mathrm{d}(x,\ y)= 5!$ and $\mathrm{l}\mathrm{c}\mathrm{m}(x,\ y)=50!$?

Solution: First, note that there are 15 primes from 1 to 50:

(2, 3, 5, 7, 11, 13, 17, 19, 23, 29, 31, 37, 41, 43, 47).

To make this easier, let's define $f(a,\ b)$ to be the greatest power of $b$ dividing $a$. (Note $g(50!,\ b)>g(5!,\ b)$ for all $b<50.$)

Therefore, for each prime $p$, we have either $f(x,p)=f(5!,p)$ and $f(y,p)=f(50!,p)$ OR $f(y,p)=f(5!,p)$ and $f(x,p)=f(50!,p)$ . Since we have 15 primes, this gives $2^{15}$ pairs, and clearly $x\neq y$ in any such pair (since the $\mathrm{g}\mathrm{c}\mathrm{d}$ and lcm are different), so there are $2^{14}$ pairs with $x\leq y.$

2. Given a finite number of closed intervals of length 1, whose union is the closed interval $[0,50]$, prove that there exists a subset of the intervals, any two of whose members are disjoint, whose union has total length at least 25. (Two intervals with a common endpoint are not disjoint.)

Solution: Consider

$I1=[1+e,\ 2+e], I2=[3+2e,\ 4+2e]$, . . . $I24=[47+24e,\ 48+24e]$

where $e$ is small enough that $48+24e<50$. To have the union of the intervals include $2k+ke,$ we must have an interval whose smallest element is in {\it Ik}. However, the difference between an element in {\it Ik} and $Ik+1$ is always greater than 1, so these do not overlap.

Taking these intervals and $[0,1]$ (which must exist for the union to be $[0,50])$ we have 25 disjoint intervals, whose total length is, of course, 25.

3. Prove that
$$
\frac{1}{1999}<\frac{1}{2}\cdot\frac{3}{4}\cdot\ \cdot\frac{1997}{1998}<\frac{1}{44}.
$$
Solution: Let $p=1/2\cdot 3/4$. . . . $\cdot$ 1997/1998 and $q=2/3\cdot 4/5$. . . . $\cdot$ 1998/1999. Note $p<q$, so $p^{2}<pq =1/2\cdot 2/3$. . . . $\cdot 1998/1999=$ 1/1999. Therefore, $p<1/1999^{1/2}<1/44$. Also,
$$
p=\frac{19.98!}{(999!2^{999})^{2}}=2^{-1998}\ \left(\begin{array}{l}
1998\\
999
\end{array}\right),
$$
while
$$
2^{1998}=\left(\begin{array}{l}
1998\\
0
\end{array}\right)\ +\cdots+\ \left(\begin{array}{l}
1998\\
1998
\end{array}\right)<1999\ \left(\begin{array}{l}
1998\\
999
\end{array}\right).
$$
Thus $p>$ 1/1999.

4. Let $O$ be a point inside a parallelogram {\it ABCD} such that $\angle AOB+ \angle COD=\pi$. Prove that $\angle OBC=\angle ODC.$

Solution: Translate {\it ABCD} along vector $AD$ so $A^{l}$ and $D$ are the same, and so that $B'$ and $C$ are the same

Now, $\angle COD+\angle CO'D=\angle COD+\angle A'O'D'=180$, so {\it OCO}'{\it D} is cyclic. Therefore, $\angle OO'C=\angle ODC$

Also, vector $BC$ and vector $OO'$ both equal vector $AD$ so {\it OBCO}' is a parallelogram. Therefore, $\angle OBC=\angle OO'C=\angle ODC.$

5. Express the sum
$$
\sum_{k=0}^{n}\frac{(-1)^{k}}{k^{3}+9k^{2}+26k+24}\ \left(\begin{array}{l}
n\\
k
\end{array}\right)
$$
in the form $p(n)/q(n)$ , where $p, q$ are polynomials with integer coefficients.

Solution: We have
$$
\sum_{k=0}^{n}\frac{(-1)^{k}}{k^{3}+9k^{2}+26k+24}\ \left(\begin{array}{l}
n\\
k
\end{array}\right)
$$
$$
=\ \sum_{k=0}^{n}\frac{(-1)^{k}}{(k+2)(k+3)(k+4)}\ \left(\begin{array}{l}
n\\
k
\end{array}\right)
$$
24

$= \displaystyle \sum_{k=0}^{n}(-1)^{k}\frac{k+1}{(n+1)(n+2)(n+3)(n+4)} \left(\begin{array}{ll}
n & +4\\
k & +4
\end{array}\right)$

$= \displaystyle \frac{1}{(n+1)(n+2)(n+3)(n+4)}\sum_{k=4}^{n+4}(-1)^{k}(k-3) \left(\begin{array}{ll}
n & +4\\
 & k
\end{array}\right)$ and
$$
\sum_{k=0}^{n+4}(-1)^{k}(k-3)\ \left(\begin{array}{ll}
n & +4\\
 & k
\end{array}\right)
$$
$$
=\ \sum_{k=0}^{n+4}(-1)^{k}k\ \left(\begin{array}{ll}
n & +4\\
 & k
\end{array}\right)\ -3\sum_{k=0}^{n+4}(-1)^{k}\ \left(\begin{array}{ll}
n & +4\\
 & k
\end{array}\right)
$$
$$
=\ \sum_{k=1}^{n+4}(-1)^{k}k\ \left(\begin{array}{ll}
n & +4\\
 & k
\end{array}\right)\ -3(1-1)^{n+4}
$$
$$
=\ \frac{1}{n+4}\sum_{k=1}^{n+4}(-1)^{k}\ \left(\begin{array}{ll}
n & +3\\
k & -1
\end{array}\right)
$$
$$
=\ \frac{1}{n+4}(1-1)^{n+3}=0.
$$
Therefore

$\displaystyle \sum_{k=4}^{n+4}(-1)^{k}(k-3) \left(\begin{array}{ll}
n & +4\\
 & k
\end{array}\right)$
$$
=\ -\sum_{k=0}^{3}(-1)^{k}(k-3)\ \left(\begin{array}{ll}
n & +4\\
 & k
\end{array}\right)
$$
$=$ 3 $\left(\begin{array}{ll}
n & +4\\
 & 0
\end{array}\right) -2 \left(\begin{array}{ll}
n & +4\\
 & 1
\end{array}\right) + \displaystyle \left(\begin{array}{ll}
n & +4\\
 & 2
\end{array}\right)=\frac{(n+1)(n+2)}{2}$ and the given sum equals $\displaystyle \frac{1}{2(n+3)(n+4)}$ .

1.4 China

1. Let $x_{1}, x_{2}$, . . . , $x_{1997}$ be real numbers satisfying the following conditions:

(a) -$\displaystyle \frac{1}{\sqrt{3}}\leq x_{i}\leq\sqrt{3}$ for $i=1,2$, . . . , 1997; (b) $ x_{1}+x_{2}+\cdot \cdot \cdot+x_{1997}=-318\sqrt{3}.$

Determine the maximum value of $x_{1}^{12}+x_{2}^{12}+\cdots+x_{1997}^{12}.$

Solution: Since $x^{12}$ is a convex function of $x$, the sum of the twelfth powers of the $x_{i}$ is maximized by having all but perhaps one of the $x_{i}$ at the endpoints of the prescribed interval. Suppose $n$ of the $x_{i}$ equal -$\displaystyle \frac{1}{\sqrt{3}}$ , $1996-n$ equal $\sqrt{3}$ and the last one equals
$$
-318\sqrt{3}+\frac{n}{\sqrt{3}}-(1996-n)\sqrt{3}.
$$
This number must be in the range as well, so
$$
-1\leq-318\times 3+n-3(1996-n)\leq 3.
$$
Equivalently $-1\leq 4n-6942\leq 3$. The only such integer is $n=1736,$ the last value is $2/\sqrt{3}$, and the maximum is $1736\times 3^{-6}+260\times 3^{6}+$
$$
(4/3)^{6}.
$$
2. Let $A_{1}B_{1}C_{1}D_{1}$ be a convex quadrilateral and $P$ a point in its interior. Assume that the angles $PA_{1}B_{1}$ and $PA_{1}D_{1}$ are acute, and similarly for the other three vertices. Define $A_{k}, B_{k}, C_{k}, D_{k}$ as the reflections of $P$ across the lines $A_{k-1}B_{k-1}, B_{k-1}C_{k-1}, C_{k-1}D_{k-1},$
$$
D_{k-1}A_{k-1}.
$$
(a) Of the quadrilaterals $A_{k}B_{k}C_{k}D_{k}$ for $k=1$, . . . , 12, which ones are necessarily similar to the 1997th quadrilateral?

(b) Assume that the 1997th quadrilateral is cyclic. Which of the first 12 quadrilaterals must then be cyclic?

Solution: We may equivalently define $A_{k}$ as the foot of the perpendicular from $P$ to $A_{k-1}B_{k-1}$ and so on. Then cyclic quadrilaterals

26

with diameters $PA_{k}, PB_{k}, PC_{k}, PD_{k}$ give that
$$
\angle PA_{k}B_{k}\ =\ \angle PD_{k+1}A_{k+1}=\angle PC_{k+2}D_{k+2}
$$
$$
=\ \angle PB_{k+3}C_{k+3}=\angle PA_{k+4}B_{k+4}.
$$
Likewise, in the other direction we have $\angle PB_{k}A_{k}=PB_{k+1}A_{k+1}$ and so on. Thus quadrilaterals 1, 5, 9 are similar to quadrilateral 1997, but the others need not be. However, if quadrilateral 1997 is cyclic (that is, has supplementary opposite angles), quadrilaterals 3, 7, and 11 are as well.

3. Show that there exist infinitely many positive integers $n$ such that the numbers 1, 2, . . . , $3n$ can be labeled
$$
a_{1},\text{ . . . }a_{n},\ b_{1},\text{ . . . }b_{n},\ c_{1},\text{ . . . }c_{n}
$$
in some order so that the following conditions hold:

(a) $a_{1}+b_{1}+c_{1}=\cdots=a_{n}+b_{n}+c_{n}$ is a multiple of 6;

(b) $a_{1}+\cdots+a_{n}=b_{1}+\cdots+b_{n}=c_{1}+\cdots+c_{n}$ is also a multiple of 6.

Solution: The sum of the integers from 1 to $3n$ is $3n(3n+1)/2,$ which we require to be a multiple both of $6n$ and of 9. Thus $n$ must be a multiple of 3 congruent to 1 modulo 4. We will show that the desired arrangement exists for $n=9^{m}$. For $n=9$, use the arrangement
$$
8\ 1\ 6\ 17\ 10\ 15\ 26\ 19\ 24
$$
$$
21\ 23\ 25\ 3\ 5\ 7\ 12\ 14\ 16
$$
$$
13\ 18\ 11\ 22\ 27\ 20\ 4\ 9\ 2
$$
(in which the first row is $a_{1}, a_{2}$, . . . and so on). It suffices to produce from arrangements for $m$ (without primes) and $n$ (with primes) an arrangement for {\it mn} (with double primes):
$$
a_{i+(j-1)m}''=a_{i}+(m-1)a_{j}'\ (1\leq i\leq m,\ 1\leq j\leq n)
$$
and likewise for the $b_{i}$ and $c_{i}.$

27

4. Let {\it ABCD} be a cyclic quadrilateral. The lines $AB$ and $CD$ meet at $P$, and the lines $AD$ and $BC$ meet at $Q$. Let $E$ and $F$ be the points where the tangents from $Q$ meet the circumcircle of {\it ABCD}. Prove that points $P, E, F$ are collinear.

Solution: Let $X'$ denote the tangent of the circle at a point $X$ on the circle. Now take the polar map through the circumcircle of {\it ABCD}. To show $P, E, F$ are collinear, we show their poles are concurrent. $E$ and $F$ map to $E'$ and $F'$ which meet at $Q$. Since $P=AB\cap CD$, the pole of $P$ is the line through $A'\cap B'$ and $C'\cap D',$ so we must show these points are collinear with $Q.$

However, by Pascal's theorem for the degenerate hexagon {\it AADBBC}, the former is collinear with $Q$ and the intersection of $AC$ and $BD,$ and by Pascal's theorem for the degenerate hexagon {\it ADDBCC}, the latter is as well.

5. [Corrected] Let $A=\{1,2$, . . . , 17$\}$ and for a function $f$ : $A\rightarrow A,$ denote $f^{[1]}(x)=f(x)$ and $f^{[k+1]}(x)=f(f^{[k]}(x))$ for $k\in \mathbb{N}$. Find the largest natural number $M$ such that there exists a bijection $f$ : $A\rightarrow A$ satisfying the following conditions:

(a) If $m<M$ and $1\leq i\leq 17$, then
$$
f^{[m]}(i+1)-f^{[m]}(i)\not\equiv\pm 1(\mathrm{m}\mathrm{o}\mathrm{d}\ 17)\ .
$$
(b) For $1\leq i\leq 17,$
$$
f^{[M]}(i+1)-f^{[M]}(i)\equiv\pm 1(\mathrm{m}\mathrm{o}\mathrm{d}\ 17)\ .
$$
(Here $f^{[k]}(18)$ is defined to equal $f^{[k]}(1).$)

Solution: The map $f(x)=3x (\mathrm{m}\mathrm{o}\mathrm{d}\ 17)$ satisfies the required condition for $M=8$, and we will show this is the maximum. Note that by composing with a cyclic shift, we may assume that $f(17)= 17$. Then $M$ is the first integer such that $f^{[M]}(1)$ equals 1 or 16, and likewise for 16. If 1 and 16 are in the same orbit of the permutation $f$, this orbit has length at most 16, and so either 1 or 16 must map to the other after 8 steps, so $M\leq 8$. If they are in different orbits, one (and thus both) orbits have length at most 8, so again $M\leq 8.$

28

6. [Corrected] Let $a_{1}, a_{2}$, . . . , be nonnegative numbers satisfying
$$
a_{n+m}\leq a_{n}+a_{m}\ (m,\ n\in \mathbb{N})\ .
$$
Prove that
$$
a_{n}\leq ma_{1}+(\frac{n}{m}-1)a_{m}
$$
for all $n\geq m.$

Solution: By induction on $k, a_{n}\leq ka_{m}+a_{n-mk}$ for $k<m/n$. Put $n=mk +r$ with $r\in\{1,\ .\ .\ .\ ,\ m\}$; then
$$
a_{n}\leq ka_{m}+a_{r}=\frac{n-r}{m}a_{m}+a_{r}\leq\frac{n-m}{m}a_{m}+ma_{1}
$$
since $a_{m}\leq ma_{1}$ and $a_{r}\leq ra_{1}.$

1.5 Colombia

1. We are given an $m\times n$ grid and three colors. We wish to color each segment of the grid with one of the three colors so that each unit square has two sides of one color and two sides of a second color. How many such colorings are possible?

Solution: Call the colors $A, B, C$, and let Now let $a_{n}$ be the number of such colorings of a horizontal $1\times n$ board given the colors of the top grid segments. For $n=1$, assume WLOG the top grid segment is colored $A$. Then there are three ways to choose the other $A$-colored segment, and two ways to choose the colors of the remaining two segments for a total of $a_{1}=6$ colorings.

We now find $a_{n+1}$ in terms of $a_{n}$. Given any coloring of a $1\times n$ board, assume WLOG that its rightmost segment is colored $A$. Now imagine adding a unit square onto the right side of the board to make a $1\times(n+1)$ board, where the top color of the new square is known. If the new top segment is colored $A$, then there are two ways to choose the colors of the remaining two segments; otherwise, there are two ways to choose which of the remaining segments is colored $A$. So, $a_{n+1}=2a_{n}$, so $a_{n}=3\cdot 2^{n}.$

As for the original problem, there are $3^{n}$ ways to color the top edges and 3 $2^{n}$ ways to color each successive row, for a total of $3^{m+n}2^{mn}$ colorings.

2. We play the following game with an equilaterial triangle of $n(n+1)/2$ pennies (with $n$ pennies on each side). Initially, all of the pennies are turned heads up. On each turn, we may turn over three pennies which are mutually adjacent; the goal is to make all of the pennies show tails. For which values of $n$ can this be achieved?

Solution: This can be achieved for all $n\equiv 0,2 (\mathrm{m}\mathrm{o}\mathrm{d}\ 3)$ ; we show the positive assertion first. Clearly this is true for $n=2$ and $n=3$ (flip each of the four possible triangles once). For larger $n$, flip each possible set of three pennies once; the corners have been flipped once, and the pennies along the sides of the triangle have each been flipped three times, so all of them become tails. Meanwhile, the interior pennies have each been flipped six times, and they form a triangle of side length $n-3$; thus by induction, all such $n$ work.

30

Now suppose $n\equiv 1 (\mathrm{m}\mathrm{o}\mathrm{d}\ 3)$ . Color the pennies yellow, red and blue so that any three adjacent pennies are different colors; also any three pennies in a row will be different colors. If we make the corners all yellow, then there will be one more yellow penny than red or blue. Thus the parity of the number of yellow heads starts out different than the parity of the number of red heads. Since each move changes the parity of the number of heads of each color, we cannot end up with the parity of yellow heads equal to that of red heads, which would be the case if all coins showed tails. Thus the pennies cannot all be inverted.

3. Let {\it ABCD} be a fixed square, and consider all squares {\it PQRS} such that $P$ and $R$ lie on different sides of {\it ABCD} and {\it Q}lies on a diagonal of {\it ABCD}. Determine all possible positions of the point $S.$

Solution: The possible positions form another square, rotated 45 degrees and dilated by a factor of 2 through the center of the square. To see this, introduce complex numbers such that $A=0, B=1, C= 1+i, D=i.$

First suppose $P$ and $R$ lie on adjacent sides of {\it ABCD}; without loss of generality, suppose $P$ lies on $AB$ and $R$ on $BC$, in which case $Q$ must lie on $AC$. (For any point on $BD$ other than the center of the square, the 90-degree rotation of $AB$ about the point does not meet $DA.$) If $P=x, Q=y+yi,$ then $R=(2y-x)i$ and $S=(x-y)+(y-x)i$, which varies along the specified square.

Now suppose $P$ and $R$ lie on opposite sides of {\it ABCD}; again without loss of generality, we assume $P$ lies on {\it AB}, $R$ on $CD$ and $Q$ on $AC.$ Moreover, we may assume $Q=y+yi$ with $1/2\leq y\leq 1$. The 90-degree rotation of $AB$ about $Q$ meets $CD$ at a unique point, and so $P=2y-1, R=i$, and $S=y-1+(1-y)i$, which again varies along the specified square.

4. Prove that the set of positive integers can be partitioned into an infinite number of (disjoint) infinite sets $A_{1}, A_{2}$, . . . so that if $x, y, z, w$ belong to $A_{k}$ for some $k$, then $x-y$ and $z-w$ belong to the same set $A_{i}$ (where $i$ need not equal {\it k}) if and only if $x/y=z/w.$ Solution: Let $A_{k}$ consist of the numbers of the form $(2k-1)(2^{n})$ ; then this partition meets the desired conditions. To see this, assume

31

$x, y, z, w\in A_{k}$ with $x>y$ and $z>w$. Write

$x=(2k-1)(2^{a+b}), y=(2k-1)(2^{a}), z=(2k-1)(2^{c+d}), w=(2k-1)(2^{c})$ . Then
$$
x-y=(2k-1)(2^{b}-1)(2^{a}),\ z-w=(2k-1)(2^{d}-1)(2^{c})\ .
$$
Also $x/y=2^{b}, z/w=2^{d}$. Now $x/y=z/w$ if and only if $b=d$ if and only if $x-y$ and $z-w$ have the same largest odd divisor.

1.6 Czech and Slovak Republics

1. Let {\it ABC} be a triangle with sides $a, b, c$ and corresponding angles a, $\beta, \gamma$. Prove that the equality $\alpha=3\beta$ implies the inequality $(a^{2}- b^{2})(a-b)=bc^{2}$, and determine whether the converse also holds.

Solution: By the extended law of sines, $a=2R\sin\alpha, b=2R\sin\beta,  c=2R\sin\gamma$, where $R$ is the circumradius of {\it ABC}. Thus,

$(a^{2}-b^{2})(a-b) = 8R^{3}$ (sin2 $\alpha$--sin2 $\beta$) $(\sin\alpha-\sin\beta) = 8R^{3}$ (sin2 $ 3\beta$--sin2 $\beta$) $(\sin 3\beta-\sin\beta)$
$$
=\ 8R^{3}(\sin 3\beta-\sin\beta)^{2}(\sin 3\beta+\sin\beta)
$$
$$
=\ 8R^{3}(8\cos^{2}2\beta\sin^{2}\beta\sin^{2}\beta\cos\beta)
$$
$= 8R^{3}$ (sin2 $(180^{\circ}-4\beta)$) $(\sin\beta)$

$= 8R^{3}$ (sin2 $\gamma$) $(\sin\beta)$
$$
=\ bc^{2}.
$$
The converse is false in general; we can also have $\alpha=3\beta-360^{\circ}$, e.g. for $\alpha=15^{\mathrm{o}}, \beta=125^{\circ}, \gamma=40^{\circ}.$

2. Each side and diagonal of a regular $n$-gon $(n\geq 3)$ is colored red or blue. One may choose a vertex and change the color of all of the segments emanating from that vertex, from red to blue and vice versa. Prove that no matter how the edges were colored initially, it is possible to make the number of blue segments at each vertex even. Prove also that the resulting coloring is uniquely determined by the initial coloring.

Solution: All congruences are taken modulo 2.

First, changing the order in which we choose the vertices does not affect the end coloring. Also, choosing a vertex twice has no net effect on the coloring. Then choosing one set of vertices has the same effect as choosing its �complement�: the latter procedure is equivalent to choosing the first set, then choosing all the vertices. (Here, in a procedure's complement, vertices originally chosen an odd number of times are instead chosen an even number of times, and vice versa.)

33

Label the vertices 1, . . . , $2n+1$. Let $a_{i}$ be the number of blue segments at each vertex, $b_{i}$ be the number of times the vertex is chosen, and $B$ be the sum of all $b_{i}$. When vertex $k$ is chosen, $a_{k}$ becomes $2n-a_{k}\equiv a_{k}$; on the other hand, the segment from vertex $k$ to each other vertex changes color, so the other $a_{i}$ change parity.

Summing the $a_{i}$ gives twice the total number of blue segments; so, there are an even number of vertices with odd $a_{i}$ --say, $2x$ vertices. Choose these vertices. The parity of these $a_{i}$ alternates $2x-1$ times to become even. The parity of the other $a_{i}$ alternates $2x$ times to remain even. Thus, all the vertices end up with an even number of blue segments. We now prove the end coloring is unique.

Consider a procedure with the desired results. At the end, $a_{i}$ becomes $a_{i}+B-b_{i} (\mathrm{m}\mathrm{o}\mathrm{d}\ 2)$ . All the $a_{i}$ equal each other at the end, so $b_{j}\equiv b_{k}$ if and only if $a_{j}\equiv a_{k}$ originally. Thus, either $b_{i}\equiv 1$ if and only if $a_{i}\equiv 1$ -- the presented procedure -- or $b_{i}\equiv 1$ if and only if $a_{i}\equiv 0-$ resuluting in an equivalent coloring from the first pargraph's conclusions. Thus, the resulting coloring is unique.

This completes the proof.

Note: For a regular $2n$-gon, $n\geq 2$, choosing a vertex reverses the parities of all of the $a_{i}$, so it is impossible to have all even $a_{i}$ unless the $a_{i}$ have equal parities to start with. And even if it is possible to have all even $a_{i}$, the resulting coloring is not unique.

3. The tetrahedron {\it ABCD} is divided into five convex polyhedra so that each face of {\it ABCD} is a face of one of the polyhedra (no faces are divided), and the intersection of any two of the five polyhedra is either a common vertex, a common edge, or a common face. What is the smallest possible sum of the number of faces of the five polyhedra?

Solution: The smallest sum is 22. No polyhedron shares two faces with {\it ABCD}; otherwise, its convexity would imply that it is {\it ABCD}. Then exactly one polyhedron $P$ must not share a face with {\it ABCD}, and has its faces in {\it ABCD}'s interior. Each of $P$'s faces must then be shared with another polyhedron, implying that $P$ shares at least 3 vertices with each of the other polyhedra. Also, any polyhedron face not shared with {\it ABCD} must be shared with another polyhedron. This implies that the sum of the number of faces is even. Each polyhedron must have at least four faces for a sum of at least 20. Assume

34

this is the sum. Then each polyhedron is a four-vertex tetrahedron, and $P$ shares at most 2 vertices with {\it ABCD}. Even if it did share 2 vertices with {\it ABCD}, say $A$ and $B$, it would then share at most 2 vertices with the tetrahedron containing {\it ACD}, a contradiction. Therefore, the sum of the faces must be at least 22. This sum can indeed be obtained. Let $P$ and $Q$ be very close to $A$ and $B$, respectively; then the five polyhedra {\it APCD, PQCD, BQCD, ABDPQ}, and {\it ABCPQ} satisfy the requirements.

4. Show that there exists an increasing sequence $\{a_{n}\}_{n=1}^{\infty}$ of natural numbers such that for any $k\geq 0$, the sequence $\{k+a_{n}\}$ contains only finitely many primes.

Solution: Let $p_{k}$ be the {\it k}-th prime number, $k\geq 1$. Set $a_{1}=2.$ For $n\geq 1$, let $a_{n+1}$ be the least integer greater than $a_{n}$ that is congruent to $-\mathrm{k}$ modulo $p_{k+1}$ for all $k\leq n$. Such an integer exists by the Chinese Remainder Theorem. Thus, for all $k\geq 0, k+a_{n}\equiv 0 (\mathrm{m}\mathrm{o}\mathrm{d}\ p_{k+1})$ for $n\geq k+1$. Then at most $k+1$ values in the sequence $\{k+a_{n}\}$ can be prime; from the $k+2$-th term onward, the values are nontrivial multiples of $p_{k+1}$ and must be composite. This completes the proof.

5. For each natural number $n\geq 2$, determine the largest possible value of the expression
\begin{center}
$ V_{n}=\sin x_{1}\cos x_{2}+\sin x$2 $\cos x3+\cdots+\sin x_{n}\cos x_{1},$
\end{center}
where $x_{1}, x_{2}$, . . . , $x_{n}$ are arbitrary real numbers.

Solution: By the inequality $2ab\leq a^{2}+b^{2}$, we get
$$
V_{n}\leq\frac{\sin^{2}x_{1}+\cos^{2}x_{2}}{2}+\ +\frac{\sin^{2}x_{n}+\cos^{2}x_{1}}{2}=\frac{n}{2},
$$
with equality for $x_{1}=\cdots=x_{n}=\pi/4.$

6. A parallelogram {\it ABCD} is given such that triangle {\it ABD} is acute and $\angle BAD=\pi/4$. In the interior of the sides of the parallelogram, points $K$ on {\it AB}, $L$ on $BC, M$ on $CD, N$ on $DA$ can be chosen in various ways so that {\it KLMN} is a cyclic quadrilateral whose circumradius equals those of the triangles {\it ANK} and {\it CLM}. Find the

35

locus of the intersection of the diagonals of all such quadrilaterals

{\it KLMN}.

Solution: Since the arcs subtended by the angles $\angle KLN, \angle KMN, \angle LKM, \angle LNM$ on the circumcircle of {\it KLMN} and the arcs subtended by $\angle KAN$ and $\angle LCM$ on the circumcircles of triangles {\it AKN} and {\it CLM}, respectively, are all congruent, these angles must all be equal to each other, and hence have measure $45^{\circ}$. The triangles {\it SKL} and {\it SMN}, where $S$ is the intersection of $KM$ and $NL,$ are thus right isosceles triangles homothetic through $S$. Under the homothety taking $K$ to $M$ and $L$ to $N,$ {\it AB} is sent to $CD$ and $BC$ to $DA$, so $S$ must lie on the segment $BD.$

1.7 France

1. Each vertex of a regular 1997-gon is labeled with an integer, such that the sum of the integers is 1. Starting at some vertex, we write down the labels of the vertices reading counterclockwise around the polygon. Can we always choose the starting vertex so that the sum of the first $k$ integers written down is positive for $k=1$, . . . , 1997?

Solution: Yes. Let $b_{k}$ be the sum of the first $k$ integers; then $b_{1997}=1$. Let $x$ be the minimum of the $b_{k}$, and find the largest $k$ such that $b_{k-1}=x$; if we start there, the sums will be positive. (Compare Spain 6.)

2. Find the maximum volume of a cylinder contained in the intersection of a sphere with center $O$ and radius $R$ and a cone with vertex $O$ meeting the sphere in a circle of radius $r$, having the same axis as the cone.

Solution: Such a cylinder meets the sphere in a circle of some radius $s<r$. The distance from that circle to the center of the sphere is $\sqrt{R^{2}-s^{2}}$. The cylinder also meets the cone in a circle of radius $s$, whose distance to the center of the sphere is $s\sqrt{R^{2}}/r^{2}-1$ (since the distance from the circle of radius $r$ to the center of the sphere is $\sqrt{R^{2}}$-{\it r}2). Thus the volume of the cylinder is
$$
\pi s^{2}(\sqrt{R^{2}-s^{2}}-s\sqrt{R^{2}/r^{2}-1}.
$$
We maximize this by setting its derivative in $s$ to zero:
$$
0=2s\sqrt{R^{2}-s^{2}}-\frac{s^{3}}{\sqrt{R^{2}-s^{2}}}-3s^{2}\sqrt{R^{2}/r^{2}-1})
$$
or rearranging and squaring,
$$
\frac{s^{4}-4R^{2}s^{2}+4R^{4}}{R^{2}-s^{2}}=\frac{9s^{2}R^{2}-s^{2}r^{2}}{r^{2}}.
$$
Solving,
$$
s^{2}=\frac{3R^{2}+r^{2}+\sqrt{(9R^{2}-r^{2})(R^{2}-r^{2})}}{6}
$$
and one can now plug $s^{2}$ into the volume formula given above to get the minimum volume.

37

3. Find the maximum area of the orthogonal projection of a unit cube onto a plane.

Solution: This projection consists of the projections of three mutually orthogonal faces onto the plane. The area of the projection of a face onto the plane equals the absolute value of the dot product of the unit vectors perpendicular to the face and the plane. If $x, y, z$ are these dot products, then the maximum area is the maximum of $x+y+z$ under the condition $x^{2}+y^{2}+z^{2}=1$. However, by CauchySchwarz, $\sqrt{x^{2}+y^{2}+z^{2}}\geq 3(x+y+z)$ with equality iff $x=y=z.$ Thus the maximum is $\sqrt{3}.$

4. Given a triangle {\it ABC}, let $a, b, c$ denote the lengths of its sides and $m, n,p$ the lengths of its medians. For every positive real $\alpha$, let $\lambda(\alpha)$ be the real number satisfying
\begin{center}
$a^{\alpha} +b\ovalbox{\tt\small REJECT}+c^{\alpha}=\lambda (\alpha)$ a $(m^{\alpha}\ +n\ovalbox{\tt\small REJECT}+p^{\alpha})$ .
\end{center}
(a) Compute $\lambda(2)$ .

(b) Determine the limit of $\lambda(\alpha)$ as $\alpha$ tends to $0.$

(c) For which triangles {\it ABC} is $\lambda(\alpha)$ independent of $\alpha$?

Solution: Say $m, n,p$ are opposite $a, b, c$, respectively, and assume {\it a}$\leq b\leq c.\mathrm{I}\mathrm{t}\mathrm{i}\mathrm{s}\mathrm{e}$asily computed ($\mathrm{e}.\mathrm{g}.,$using vectors)$m^{2}=(2b^{2}+2c^{2}-a^{2})/4\displaystyle \mathrm{a}\mathrm{n}\mathrm{d}\mathrm{s}\mathrm{o}\mathrm{o}\mathrm{o}\lambda(2)=\frac{2}{\sqrt{3}}\mathrm{I}\mathrm{f}x\leq y\leq z\mathrm{t}$henthat as $\ovalbox{\tt\small REJECT}\rightarrow 0$, then
\begin{center}
$x\leq(x^{\alpha} +y\ovalbox{\tt\small REJECT}+z$a$)$ 1/$\alpha \leq 3^{1/\alpha_{X}}$
\end{center}
and so the term in the middle tends to $x$. We conclude that the limit of $\lambda$(a) as $\alpha\rightarrow 0$ is $a/p$. For $\lambda(\alpha)$ to be independent of $\alpha$, we first need $a^{2}/p^{2}=4/3$, which reduces to $a^{2}+c^{2}=2b^{2}$. But under that condition, we have
$$
m=c\sqrt{3}/2,\ n=b\sqrt{3}/2,p=a\sqrt{3}/2
$$
and so $\lambda(\alpha)$ is clearly constant for such triangles.

1.8 Germany

1. Determine all primes $p$ for which the system
$$
p+1\ =\ 2x^{2}
$$
$$
p^{2}+1\ =\ 2y^{2}
$$
has a solution in integers $x, y.$

Solution: The only such prime is $p=7$. Assume without loss of generality that $x, y\geq 0$. Note that $p+1=2x^{2}$ is even, so $p\neq 2.$ Also, $2x^{2}\equiv 1\equiv 2y^{2} (\mathrm{m}\mathrm{o}\mathrm{d}\ p)$ which implies $x\equiv\pm y (\mathrm{m}\mathrm{o}\mathrm{d}\ p)$ since $p$ is odd. Since $x<y<p$, we have $x+y=p$. Then
$$
p^{2}+1=2(p-x)^{2}=2p^{2}-4px+p+1,
$$
so $p=4x-1,2x^{2}=4x, x$ is $0$ or 2 and $p$ is $-1$ or 7. Of course $-1$ is not prime, but for $p=7, (x,\ y)=(2,5)$ is a solution.

2. A square $S_{a}$ is inscribed in an acute triangle {\it ABC} by placing two vertices on side $BC$ and one on each of $AB$ and $AC$. Squares $S_{b}$ and $S_{c}$ are inscribed similarly. For which triangles {\it ABC} will $S_{a}, S_{b}, S_{c}$ all be congruent?

Solution: This occurs for {\it ABC} equilateral (obvious) and in no other cases. Let $R$ be the circumradius of {\it ABC} and let $x_{a}, x_{b}, x_{c}$ be the side lengths of $S_{a}, S_{b}, S_{c}$. Finally, let $\alpha, \beta, \gamma$ denote the angles
$$
\angle BAC,\ \angle CBA,\ \angle ACB.
$$
Suppose $S_{a}$ has vertices $P$ and $Q$ on $BC$, with $P$ closer to $B$. Then

$2R\sin\alpha=BC \ovalbox{\tt\small REJECT}$-
$$
x_{a}\ --
$$
$: BP+PQ+QC$

$ x_{a}\cot\beta+x_{a}+x_{a}\cot\gamma$

$: \displaystyle \frac{2R\sin\alpha}{1+\cot\beta+\cot\gamma}$

$: \displaystyle \frac{2R\sin\alpha\sin\beta\sin\gamma}{\sin\beta\sin\gamma+\cos\beta\sin\gamma+\cos\gamma+\sin\beta}  2R\sin\alpha\sin\beta\sin\gamma$

$\sin\beta\sin\gamma+\sin\alpha$

49

and similarly for $x_{b}$ and $x_{c}$. Now $x_{a}=x_{b}$ implies
\begin{center}
$\sin\beta\sin\gamma+\sin$ a $= \sin\gamma\sin$ a$+ \sin\beta$
$$
0\ =\ (\sin\beta-\sin\alpha)(\sin\gamma-1)\ .
$$
\end{center}
Since {\it ABC} is acute, we have $\sin\beta=\sin\alpha$, which implies $\alpha=\beta$ (the alternative is that $\alpha+\beta=\pi$, which cannot occur in a triangle). Likewise $\beta=\gamma$, so {\it ABC} is equilateral.

3. In a park, 10000 trees have been placed in a square lattice. Determine the maximum number of trees that can be cut down so that from any stump, you cannot see any other stump. (Assume the trees have negligible radius compared to the distance between adjacent trees.)

Solution: The maximum is 2500 trees. In any square of four adjacent trees, at most one can be cut down. Since the $100\times 100$ grid can be divided into 2500 such squares, at most 2500 trees can be cut down.

Identifying the trees with the lattice points $(x,\ y)$ with $0\leq x, y\leq 99,$ we may cut down all trees with even coordinates. To see this, note that if $a, b, c, d$ are all even, and $p/q$ is the expression of $(d-b)/(c-a)$ in lowest terms (where $p, q$ have the same signs as $d-b, c-a$), then one of $a+p$ and $b+q$ is odd, so the tree $(a+p,\ b+q)$ blocks the view from $(a,\ b)$ to $(c,\ d)$ .

4. In the circular segment {\it AMB}, the central angle $\angle AMB$ is less than $90^{\circ}$. FRom an arbitrary point on the arc $AB$ one constructs the perpendiculars $PC$ and $PD$ onto $MA$ and $MB(C\in MA,  D\in MB)$ . Prove that the length of the segment $CD$ does not depend on the position of $P$ on the arc $AB.$

Solution: Since $\angle PCM=\angle PDM=\pi/2$, quadrilateral {\it PCMD} is cyclic. By the Extended Law of Sines, $CD=PM$ sin {\it CMD}, which is constant.

5. In a square {\it ABCD} one constructs the four quarter circles having their respective centers at $A, B, C$ and $D$ and containing the two adjacent vertices. Inside {\it ABCD} lie the four intersection points $E,$

40

$F, G$ and $H$, of these quarter circles, which form a smaller square $\mathcal{S}$. Let $\mathcal{C}$ be the circle tangent to all four quarter circles. Compare the areas of $\mathcal{S}$ and $\mathcal{C}.$

Solution: Circle $\mathcal{C}$ has larger area. Let $[C]$ denotes its area and $[S]$ that of square $\mathcal{S}$. Without loss of generality, let $E$ be the intersection of the circes closest to $AB$, and $G$ the intersection closest to $CD.$ Drop perpendiculars $EE'$ to $AB$ and $GG'$ to $CD$. By symmetry, $E', E, G, G'$ are collinear.

Now since $AB=BG=AG$, {\it ABG} is equilateral and $E'G$ is the altitude $\sqrt{3}AB/2$. Likewise $G'E=\sqrt{3}AB/2$. Then $\sqrt{3}AB=E'G+ G'E=AB+GE$, so $GE=(\sqrt{3}-1)AB$ and $[S]=EG^{2}/2= (2-\sqrt{3})AB^{2}.$

Let $I$ and $K$ be the points of tangency of $\mathcal{C}$ with the circles centered at $C$ and $A$, respectively. By symmetry again, $A, I, K, C$ are collinear. Then $2AB=AK+CI=AC+IK=\sqrt{2}AB+IK$, and $IK=(2-\sqrt{2})AB$. Thus
$$
[C]=\frac{\pi}{4}IK^{2}=\frac{(3-2\sqrt{2})\pi}{2}AB^{2}>(2-\sqrt{3})AB^{2}.
$$
6. Denote by $u(k)$ the largest odd number that divides the natural number $k$. Prove that
$$
\frac{1}{2^{n}}.\sum_{k=1}^{2^{n}}\frac{u(k)}{k}\geq\frac{2}{3}.
$$
Solution: Let $v(k)$ be the greatest power of 2 dividing $k$, so $u(k)v(k)=k$. Among $\{$1, . . . , $2^{n}\}$, there are $2^{n-i-1}$ values of $k$ such that $v(k)=2^{i}$ for $i\leq n-1$, and one value such that $v(k)=2^{n}.$ Thus the left side equals
$$
\frac{1}{2^{n}}\sum_{k=1}^{2^{n}}\frac{1}{v(k)}=\frac{1}{4^{n}}+\sum_{i=0}^{n-1}\frac{2^{n-1-i}}{2^{n+i}}.
$$
Summing the geometric series gives
$$
4^{-n}+\frac{2}{3}(1-4^{-n})\geq\frac{2}{3}.
$$
41

7. Find all real solutions of the system of equations

$x^{3} = y^{3} = z^{3} =$

$2y-1 2z-1 2x-1$

Solution: The solutions are
$$
x=y=z=t,\ t\in\{1,\ \frac{-1+\sqrt{5}}{2},\ \frac{-1-\sqrt{5}}{2}\}.
$$
Clearly these are all solutions with $x=y=z$. Assume on the contrary that $x\neq y$. If $x>y$, then $y=(x^{3}+1)/2>(y^{3}+1)/2=z,$ so $y>z$, and likewise $z>x$, contradiction. Similarly if $x<y$, then $y<z$ and $z<x$, contradiction.

8. Define the functions
$$
f(x)\ =\ x^{5}+5x^{4}+5x^{3}+5x^{2}+1
$$
$$
g(x)\ =\ x^{5}+5x^{4}+3x^{3}-5x^{2}-1.
$$
Find all prime numbers $p$ for which there exists a natural number $ 0\leq x<p$, such that both $f(x)$ and $g(x)$ are divisible by $p$, and for each such find all such

$p$, find all such $x.$ Solution: The only such primes are $p=5,17$. Note that
$$
f(x)+g(x)=2x^{3}(x+1)(x+4)\ .
$$
Thus if $p$ divides $f(x)$ and $g(x)$ , it divides either 2, $x, x+1$ or $x+4$ as well. Since $f(0)=1$ and $f(1)=17$, we can't have $p=2$. If $p$ divides $x$ then $f(x)\equiv 1 (\mathrm{m}\mathrm{o}\mathrm{d}\ p)$ , also impossible. If $p$ divides $x+1$ then $f(x)\equiv 5 (\mathrm{m}\mathrm{o}\mathrm{d}\ p)$ , so $p$ divides 5, and $x=4$ works. If $p$ divides $x+4$ then $f(x)\equiv 17 (\mathrm{m}\mathrm{o}\mathrm{d}\ p)$ , so $p$ divides 17, and $x=13$ works.

1.9 Greece

1. Let $P$ be a point inside or on the sides of a square {\it ABCD}. Determine the minimum and maximum possible values of
$$
f(P)=\angle ABP+\angle BCP+\angle CDP+\angle DAP.
$$
Solution: Put the corners of the square at 1, $i, -1, -i$ of the complex plane and put $P$ at $z$; then $f(P)$ is the argument of
$$
\frac{z-1}{i+1}\frac{z-i}{-1-i}\frac{z+1}{-i+1}\frac{z+i}{1+i}=\frac{z^{4}-1}{4}.
$$
Since $|P|\leq 1, (z^{4}-1)/4$ runs over a compact subset of the complex plane bounded by a circle of radius 1/4 centered at $-1/4$. Hence the extreme angles must occur at the boundary of the region, and it suffices to consider $P$ on a side of the square. By symmetry any side, say $AB$, will do. As $P$ moves from $A$ to $B, \angle CDP$ decreases from $\pi/2$ to $\pi/4, \angle BCP$ decreases from $\pi/4$ to $0$, and the other two remain fixed at $\pi/2$ and $0$. Hence the supremum and infimum of $f(P)$ are $5\pi/4$ and $3\pi/4$ respectively.

2. Let $f$ : $(0,\ \infty)\rightarrow \mathbb{R}$ be a function such that

(a) $f$ is strictly increasing;

(b) $f(x)>-1/x$ for all $x>0$;

(c) $f(x)f(f(x)+1/x)=1$ for all $x>0.$

Find $f(1)$ .

Solution: Let $k=f(x)+1/x$. Then $k>0$, so
$$
f(k)f(f(k)+1/k)=1.
$$
But also $f(x)f(k)=1$, hence
$$
f(x)=f(f(k)+1/k)=f(1/f(x)+1/(f(x)+1/x))\ .
$$
Since $f$ is strictly increasing, $f$ is injective, so $x=1/f(x)+1/(f(x)+ 1/x)$ . Solving for $f(x)\mathrm{w}\mathrm{e}$ get $f(x)=(1+-\sqrt{(}5)$) $/(2x)$ , and it's easy to check that only $(1-\sqrt{(}5)$) $/(2x)$ satisfies all three conditions. Hence $f(1)=(1-J(5))-/2.$

43

3. Find all integer solutions of
$$
\frac{13}{x^{2}}+\frac{1996}{y^{2}}=\frac{z}{1997}.
$$
Solution: Let $d=\mathrm{g}\mathrm{c}\mathrm{d}(x,\ y)$ so that $x=dx_{1}, y=dy_{1}$. Then the equation is equivalent to $1997(13)y_{1}^{2}+1997(1996)x_{1}^{2}=d^{2}zx_{1}^{2}y_{1}^{2}.$ Since $x_{1}$ and $y_{1}$ are coprime we must have
$$
x_{1}^{2}|1997\times 13,\ y_{1}^{2}|1997\times 1996.
$$
$\mathrm{I}\mathrm{t}$'s easy to check that 1997 is square-free, and clearly is coprime to 13 and to 1996. Moreover, $1996=2^{2}\cdot 499$, and it's easy to check that 499 is square-free. Therefore $(x_{1},\ y_{1})=(1,1)$ or (1, 2). Consider them as separate cases:

Case 1: $(x_{1},\ y_{1})=(1,1)$ . Then $d^{2}z=(13+1996)1997=1997\cdot 7^{2}\cdot 41.$ Since 1997 is coprime to 7 and 41, $d=1,7$. These give respectively the solutions
\begin{center}
$(x,\ y,\ z)=(1,1$, 4011973$)$ , (7, 7, 81877).
\end{center}
Case 2: $(x_{1},\ y_{1})=(1,2)$ . Then $d^{2}z=(13+499)1997=1997\cdot 2^{9}$. So $d=1,2,4,8,16$. These give respectively the solutions
\begin{center}
$(x,\ y,\ z) =$ (1, 2, 1022464), (2, 4, 255616), (4, 8, 63904),
\end{center}
(8, 16, 15976), (16, 32, 3994).

There are also solutions obtained from these by negating $x$ and $y.$

4. Let $P$ be a polynomial with integer coefficients having at least 13 distinct integer roots. Show that if $n\in \mathbb{Z}$ is not a root of $P$, then $|P(n)|\geq$ 7(6!)2, and give an example where equality is achieved.

Solution: If we factor out a linear factor from a polynomial with integer coefficients, then by the division algorithm the remaining depressed polynomial also has integer coefficients. Hence $P(x)$ can be written as $(x-r_{1})(x-r_{2})\ldots(x-r_{13})Q(x)$ , where the $r$'s are 13 of its distinct integer roots. Therefore for all integers $x, P(x)$ is the product of 13 distinct integers times another integer.

44

Clearly the minimum nonzero absolute value of such a product is $|$(1)(-1)(2)(-2)...(6)(-6)(7)(1)$| =$ 7(6!)2, as desired. Equality is satisfied, for example, when $x=0$ and $P(x)=(x+1)(x-1)(x+$ 2) $(x-2)\ldots(x+7)$ .

1.10 Hungary

1. Each member of a committee ranks applicants $A, B, C$ in some order. It is given that the majority of the committee ranks $A$ higher than $B$, and also that the majority of the commitee ranks $B$ higher than $C$. Does it follow that the majority of the committee ranks $A$ higher than $C$?

Solution: No. Suppose the committee has three members, one who ranks $A>B>C$, one who ranks $B>C>A$, and one who ranks $C>A>B$. Then the first and third both prefer $A$ to $B$, and the first and second both prefer $B$ to $C$, but only the first prefers $A$ to $C.$

2. Let $a, b, c$ be the sides, $m_{a}, m_{b}, m_{c}$ the lengths of the altitudes, and $d_{a}, d_{b}, d_{c}$ the distances from the vertices to the orthocenter in an acute triangle. Prove that
$$
m_{a}d_{a}+m_{b}d_{b}+m_{c}d_{c}=\frac{a^{2}+b^{2}+c^{2}}{2}.
$$
Solution: Let $D, E, F$ be the feet of the altitudes from $A, B, C$ respectively, and let $H$ be the orthocenter of triangle {\it ABC}. Then triangle {\it ACD} is similar to triangle {\it AHE}, so $m_{a}d_{a}=AD\cdot AH= AC\cdot AE=AE\cdot b$. Similarly triangle {\it ABD} is similar to triangle {\it AHF}, so $m_{a}d_{a}=AD\cdot AH=AB\cdot AF=AB\cdot c$. Therefore
$$
m_{a}d_{a}=\frac{AE\cdot b+AF\cdot c}{2}.
$$
Similarly

$m_{b}d_{b}=\displaystyle \frac{BF\cdot c+BD\cdot a}{2}$ and $m_{c}d_{c}=\displaystyle \frac{CD\cdot a+CE\cdot b}{2}.$ Therefore
$$
m_{a}d_{a}+m_{b}d_{b}+m_{c}d_{c}
$$
$$
=\ \frac{1}{2}(AE\cdot b+AF\cdot c+BF\cdot c+BD\cdot a+CD\cdot a+CE\cdot b)
$$
$$
=\ \frac{1}{2}((BD+CD)\cdot a+(CE+AE)\cdot b+(AF+BF)\cdot c)
$$
$$
a^{2}+b^{2}+c^{2}
$$
2

46

3. Let $R$ be the circumradius of triangle {\it ABC}, and let $G$ and $H$ be its centroid and orthocenter, respectively. Let $F$ be the midpoint of $GH$. Show that $AF^{2}+BF^{2}+CF^{2}=3R^{2}.$

Solution: We use vectors with the origin at the circumcenter of triangle {\it ABC}. Then we have the well-known formulas $H=A+B+C$ and $G=H/3$, so $F=(G+H)/2=2H/3$, and $2(A+B+C)=3F.$ Therefore

$AF^{2}+BF^{2}+CF^{2}$

$= (A-F)\cdot(A-F)+(B-F)\cdot(B-F)+(C-F)\cdot(C-F)$
$$
=\ A\cdot A+B\cdot B+C\cdot C-2(A+B+C)\cdot F+3F\cdot F
$$
$$
=\ 3R^{2}-F\cdot(2(A+B+C)-3F)=3R^{2}.
$$
4. A box contains 4 white balls and 4 red balls, which we draw from the box in some order without replacement. Before each draw, we guess the color of the ball being drawn, always guessing the color more likely to occur (if one is more likely than the other). What is the expected number of correct guesses?

Solution: The expected number of correct guesses is 373/70. For $i,j\geq 0$, let $a_{ij}$ denote the expected number of correct guesses when there are $i$ white balls and $j$ red balls. Suppose $i>j\geq 1$; then our guess is correct with probability $i/(i+j)$ , giving an expected number of correct guesses of $1+a_{i-1,j}$, and wrong with probability $j/(i+j)$ , giving an expected number of $a_{i,j-1;}$ so
\begin{center}
$a_{ij}=\displaystyle \frac{\dot{i}}{i+j}(1+a_{i-1,j})+\frac{j}{i+j}a_{i,j-1}$ if $i>j.$
\end{center}
Also, we clearly have $a_{ij}=a_{ji}$ for $i,j\geq 0$. If $i=j\geq 1$, then our guess is correct with probability 1/2, and
$$
a_{ii}=\frac{1}{2}(1+a_{i-1,1})+\frac{1}{2}a_{i,i-1}=\frac{1}{2}+a_{i,i-1}
$$
as $a_{i,i-1}=a_{i-1,1}$. Finally, the initial conditions are

$a_{i0}=a_{0i}=i$ for $i\geq 0.$

We can use these equations to compute $a_{4,4}=$ 373/70.

47

5. Find all solutions in integers of the equation
$$
x^{3}+(x+1)^{3}+(x+2)^{3}+\cdots+(x+7)^{3}=y^{3}.
$$
Solution: The solutions are $(-2,6), (-3,4), (-4,\ -4), (-5,\ -6)$ . Let $P(x)=x^{3}+(x+1)^{3}+(x+2)^{3}+\cdots+(x+7)^{3}=8x^{3}+84x^{2}+ 420x+784$. If $x\geq 0$, then
$$
(2x+7)^{3}\ =\ 8x^{3}+84x^{2}+294x+343
$$
$$
<\ P(x)<8x^{3}+120x^{2}+600x+1000=(2x+10)^{3},
$$
so $2x+7<y<2x+10$; therefore $y$ is $2x+8$ or $2x+9$. But neither of the equations
$$
P(x)-(2x+8)^{3}=-12x^{2}+36x+272=0
$$
$$
P(x)-(2x+9)^{3}=-24x^{2}-66x+55=0
$$
have any integer roots, so there are no solutions with $x\geq 0$. Next, note that $P$ satisfies $P(-x-7)=-P(x)$ , so $(x,\ y)$ is a solution iff $(-x-7,\ -y)$ is a solution. Therefore there are no solutions with $x\leq-7$. So for $(x,\ y)$ to be a solution, we must have $-6\leq x\leq-1.$ For $-3\leq x\leq-1$, we have $P(-1)=440$, not a cube, $P(-2)= 216=6^{3}$, and $P(-3)=64=4^{3}$, so $(-2,6)$ and $(-3,4)$ are the only solutions with $-3\leq x\leq-1$. Therefore $(-4,\ -4)$ and $(-5,\ -6)$ are the only solutions with $-6\leq x\leq-4$. So the only solutions are $(-2,6), (-3,4), (-4,\ -4)$ , and $(-5,\ -6)$ .

6. We are given 1997 distinct positive integers, any 10 of which have the same least common multiple. Find the maximum possible number of pairwise coprime numbers among them.

Solution: The maximum number of pairwise coprime numbers in this set is 9.

First, suppose there were 10 pairwise coprime numbers $n_{1}, n_{2}$, . . . , $n_{10}$. Then the least common multiple of any 10 members of this set is $\mathrm{l}\mathrm{c}\mathrm{m} (n_{1},\ n_{2},\ .\ .\ .\ ,\ n_{10})=n_{1}n_{2}\cdots n_{10}$. In particular, for any other $N$ in this set, $\mathrm{l}\mathrm{c}\mathrm{m} (N,\ n_{2},\ \cdots\ ,\ n_{10})=n_{1}n_{2}\cdots n_{10}$ is divisible by $n_{1}$; as $n_{1}$ is relatively prime to $n_{j}$ for $2\leq j\leq 10, n_{1}$ divides $N$. Similarly $n_{i}$

48

divides $N$ for each $i\in\{2$, . . . , 10$\}$, so as the $n_{i}$ are relatively prime, $n_{1}n_{2}\cdots n_{10}$ divides $N$. But $N\leq \mathrm{l}\mathrm{c}\mathrm{m} (N,\ n_{2},\ \cdots\ ,\ n_{10})=n_{1}n_{2}\cdots n_{10},$ so we must have $N=n_{1}n_{2}\cdots n_{10}$. Since this holds for every element of our set other than $n_{1}$, . . . , $n_{10}$, our set can only contain 11 elements, a contradiction.

Now we construct an example where there are 9 pairwise coprime numbers. Let $p_{n}$ denote the {\it n}th prime, and let
$$
S=\{\frac{p_{1}p_{2}\cdots p_{1988}}{p_{j}}\ 1\leq j\leq 1988\}\cup\{n_{1},\ n_{2},\ .\ .\ .\ ,\ n9\}
$$
where

$n_{i}=p_{i}$ for $1\leq i\leq 8, n_{9}=p_{9}p_{10}\cdots p_{1988}.$

Clearly any two elements of $\{n_{1},\ .\ .\ .\ ,\ n_{9}\}$ are coprime, so it suffices to show that any 10 elements of $S$ have the same least common multiple. Let $K=p_{1}p_{2}\cdots p_{1988;}$ then $n$ divides $K$ for every $n\in S,$ so the least common multiple of any 10 elements of $S$ is at most $K.$ Also note that each prime $p_{i}(1\leq i\leq 1988)$ divides all but 9 of the elements of $S$, so any collection of 10 members of $S$ contains at least one element divisible by each prime $p_{i}$. So the least common multiple of any 10 members is divisible by $p_{i}$ for every $1\leq i\leq 1988.$ Therefore the least common multiple of any 10 members of $S$ is $K.$

7. [Corrected] Let $AB$ and $CD$ be nonintersecting chords of a circle, and let $K$ be a point on $CD$. Construct (with straightedge and compass) a point $P$ on the circle such that $K$ is the midpoint of the part of segment $CD$ lying inside triangle {\it ABP}.

Solution: Construct $A'$ on line $AK$ so that $AK=KA'$, and construct the point $E$ such that triangles {\it ABC} and $A'BE$ are directly similar. Construct the circumcircle $\omega$ of $A'BE$; it intersects the segment $CD$ at a point $N$ such that $\angle BNA'=\pi-\angle A'EB=\pi-\angle ACB.$ Let $P$ be the second intersection of $BN$ with the original circle, and let $M=AP\cap CD$. Then $\angle PNA'=180-\angle BNA'=\angle ACB= \angle APB=\angle APN$, so $AP$ is parallel to $NA'$. Therefore triangles {\it AKM} and $A'KN$ are similar; but $AK=KA'$, so $MK=KN.$ Thus $P$ has the desired property.

59

8. We are given 111 unit vectors in the plane whose sum is zero. Show that there exist 55 of the vectors whose sum has length less than 1.

Solution: We will first show that given a collection of $k$ vectors whose sum has length at most 1, we can find either $k+1$ or $k+2$ of the vectors whose sum also has length at most 1. Let $v$ be the sum of the $k$ given vectors. If $v=0$, we can add any vector and the length will become 1. If $v\neq 0$, assume without loss of generality that $v$ points horizontally to the right. If any of the remaining vectors has horizontal component at least $-1/2$, then we can add that vector to our collection and the total length will still be at most 1. So suppose no such vector exists. Since the sum of the vectors is $0$, but $v$ points strictly to the right, there must be at least one vector with a leftward horizontal component. Let $w$ be the vector with the maximum leftward component, which is greater than $-1/2$ by assumption. Draw the line on which $w$ lies; since $v$ lies on the right of this line, there must be a vector $x$ which points to the left. Then $w+x$ lies on the arc of the circle of radius 1 centered at $w$ which lies below the $x$-axis. This arc is contained in the circle around $($-1/2, $0)$ of radius 1/2, as the horizontal component of $w$ is greater than $-1/2$. So $w+x$ lies in this circle; therefore $v+w+x$ has length at most 1, and the claim is proven.

Now starting with the empty collection, which has sum $0$, we repeatedly apply this argument, and we end up with either 55 or 56 vectors whose sum has length less than 1. If 55, take those; if 56, take the remaining vectors.

1.11 Iran

1. Suppose $w_{1}$, . . . , $w_{k}$ are distinct real numbers with nonzero sum. Prove that there exist integers $n_{1}$, . . . , $n_{k}$ such that $n_{1}w_{1}+\cdots+ n_{k}w_{k}>0$ and that for any permutation $\pi$ of $\{$1, . . . , $k\}$ not equal to the identity, we have $n_{1}w_{\pi(1)}+\cdots+n_{k}w_{\pi(k)}<0.$

Solution: First recall the following �rearrangement� inequality: if $a_{1}<\cdots<a_{n}, b_{1}<\cdots<b_{n}$ are real numbers,
$$
\alpha=\min\{a_{i+1}-a_{i}\},\ \beta=\min\{b_{i+1}-b_{i}\},
$$
then for any nontrivial permutation $\pi$ of $\{$1, . . . , $n\},$
$$
\sum b_{i}a_{\pi(i)}\leq\sum b_{i}a_{i}-\alpha\beta.
$$
This holds because if $i<j$ but $\pi(i)>\pi(j)$ , then replacing $\pi$ by its composition with the transposition of $i$ and $j$ increases the sum by
$$
(a_{j}-a_{i})(b_{j}-b_{i})\ .
$$
Assume that $w_{1}<\cdots<w_{k}$, and let $s=|\displaystyle \sum w_{i}|$. Let $\displaystyle \alpha=\min\{w_{i+1}- w_{i}\}$ and pick a natural number $ N>s/\alpha$. Now set
\begin{center}
$(n_{1},\ n_{2},\ .\ .\ .\ ,\ n_{k})=(N,\ 2N,\ \cdots\ ,\ kN)+p(1$, . . . , 1$)$ ,
\end{center}
where $p$ is the unique integer such that $\displaystyle \sum n_{i}w_{i}\in(0,\ s$]. Now the theorem implies that for $\pi\neq 1,$
$$
\sum n_{i}w_{\pi(i)}\leq\sum n_{i}w_{i}-N\alpha\leq s-N\alpha<0.
$$
2. Suppose the point $P$ varies along the arc $BC$ of the circumcircle of triangle {\it ABC}, and let $I_{1}, I_{2}$ be the respective incenters of the triangles {\it PAB, PAC}. Prove that

(a) the circumcircle of $PI_{1}I_{2}$ passes through a fixed point;

(b) the circle with diameter $I_{1}I_{2}$ passes through a fixed point; (c) the midpoint of $I_{1}I_{2}$ lies on a fixed circle.

Solution: Let $B_{1}, C_{1}$ be the midpoint of the arcs $AC,$ {\it AB}. Since $I_{1}, I_{2}$ are the incenters of the triangles {\it ABP, ACP}, we have
$$
C_{1}A=C_{1}B=C_{1}I_{2},\ B_{1}A=B_{1}C=B_{1}I_{2}.
$$
51

Let $O$ be the circumcenter of {\it ABC}, and let $Q$ be the second intersection of the circumcircles of {\it ABC} and $PI_{1}I_{2}$. Since $C_{1}I_{1}$ and $B_{1}I_{2}$ pass through $P$, the triangles $QI_{1}C_{1}$ and $QI_{2}B_{1}$ are similar, so
$$
\frac{QC_{1}}{QB_{1}}=\frac{C_{1}I_{1}}{B_{1}I_{2}}=\frac{C_{1}A}{B_{1}A}
$$
which is constant. Hence $Q$ is the intersection of the circumcircle of {\it ABC} with a fixed circle of Apollonius, so is constant and (a) is complete.

Since
$$
\angle I_{1}QI_{2}=I_{1}PI_{2}=C_{1}PB_{1}=(B+C)/2,
$$
the triangles $QI_{1}I_{2}$ for various $P$ are all similar. Thus if $M$ is the midpoint of $I_{1}I_{2}$, the triangles $QI_{1}M$ are also all similar. If $k= QM/QI_{1}, \alpha=\angle MQI_{1}$, this means $M$ is the image of $I_{1}$ under a spiral similarity about $Q$ with angle $\alpha$ and ratio $k$. Since $C_{1}I_{1}=C_{1}A$ is constant, $I_{1}$ moves on an arc of a circle and $M$ is the image of said arc under the spiral similarity, and (c) is complete.

To finish, we compute that $\angle I_{1}II_{2}=\pi/2$. Thus the circle with diameter $I_{1}I_{2}$ passes through $I$, and (b) is complete.

3. Suppose $f$ : $\mathbb{R}^{+}\rightarrow \mathbb{R}^{+}$ is a decreasing continuous function such that for all $x, y\in \mathbb{R}^{+},$
$$
f(x+y)+f(f(x)+f(y))=f(f(x+f(y)))+f(y+f(x))\ .
$$
Prove that $f(f(x))=x.$

Solution: Putting $y=x$ gives
$$
f(2x)+f(2f(x))=f(2f(x+f(x)))\ .
$$
Replacing $x$ with $f(x)$ gives
$$
f(2f(x))+f(2f(f(x)))=f(2f(f(x)+f(f(x))))\ .
$$
Subtracting these two equations gives

$f(2f(f(x)))-f(2x)=f(2f(f(x)+f(f(x))))-f(2f(x+f(x)))$ .

52

If $f(f(x))>x$, the left side of this equation is negative, so
$$
f(f(x)+f(f(x))>f(x+f(x))
$$
and $f(x)+f(f(x))<x+f(x)$ , a contradiction. A similar contradiction occurs if $f(f(x))<x$. Thus $f(f(x))=x$ as desired. (Continuity is not needed.)

4. Let $A$ be a matrix of zeroes and ones which is symmetric $(A_{ij}=A_{ji}$ for all $i,j$) such that $A_{ii}=1$ for all $i$. Show that there exists a subset of the rows whose sum is a vector all of whose components are odd.

Solution: If the claim does not hold, there exists a vector $(v_{1},\ .\ .\ .\ ,\ v_{n})$ such that $\displaystyle \sum_{i}v_{i}w_{i}=0$ for any row $(w_{1},\ .\ .\ .\ ,\ w_{n})$ but $\displaystyle \sum v_{i}\neq 0$. (All arithmetic here is modulo 2.) Summing over all of the rows, we get
$$
\sum_{j}\sum_{i}v_{i}A_{ij}v_{j}=0.
$$
By symmetry, this reduces to $\displaystyle \sum_{i}v_{i}^{2}A_{ii}=0$, or $\displaystyle \sum_{i}v_{i}=0$ (since $v_{i}\in\{0,1\})$ , a contradiction.

1.12 Ireland

1. Find all pairs $(x,\ y)$ of integers such that $1+1996x+1998y=xy.$

Solution: We have
$$
(x-1998)(y-1996)=xy\ -1998y-1996x+1996\cdot 1998=1997^{2}.
$$
Since 1997 is prime, we have $x-1998=\pm 1, \pm 1997, \pm 1997^{2}$, yielding the six solutions

$(x,\ y) = ($1999, 19972 $+1996), (1997,$ -19972$+$1996$)$ , (3995, 3993),
\begin{center}
$(1,\ -1)$ , $($19972 $+1998$, 1997$)$ , $($-19972 $+1998$, 1995$)$ .
\end{center}
2. Let {\it ABC} be an equilateral triangle. For $M$ inside the triangle, let $D, E, F$ be the feet of the perpendiculars from $M$ to $BC, CA,$ {\it AB}, respectively. Find the locus of points $M$ such that $\angle FDE=\pi/2.$

Solution: From the cyclic quadrilaterals {\it MDBF} and {\it MDCE}, $\angle MDE=\angle MCE$ and $\angle MDF=\angle MBF$. Thus $\angle FDE=\pi/2$ if and only if $\angle MCB+\angle MBC=\pi/6$, or equivalently, if $\angle BMC= 5\pi/6$. This holds for $M$ on an arc of the circle through $B$ and $C.$

3. [Corrected] Find all polynomials $p(x)$ such that for all $x,$
$$
(x-16)p(2x)=16(x-1)p(x)\ .
$$
Solution: If $d=\deg P$ and $a$ is the leading coefficient of $p(x)$ , then the leading coefficient of the left side is $2^{d}a$, which must equal $16a$. Thus $d=4$. Now the right side is divisible by $x-1$, as must be the left side. But in that case, the right side is divisible by $x-2,$ and likewise by $x-4$ and $x-8$. Thus $P$ must be a multiple of $(x-1)(x-2)(x-4)(x-8)$ , and all such polynomials satisfy the equation.

4. Let $a, b, c$ be nonnegative real numbers such that $a+b+c\geq abc.$ Prove that $a^{2}+b^{2}+c^{2}\geq abc.$

Solution: We may assume $a, b, c>0$. Suppose by way of contradiction that $a^{2}+b^{2}+c^{2}<abc;$ then $abc>a^{2}$ and so $a<bc,$ and likewise $b<ca, c<ab.$ Then
$$
abc\geq a^{2}+b^{2}+c^{2}\geq ab\ +bc+ca
$$
by AM-GM, and the right side exceeds $a+b+c$, contradiction.

5. Let $S=\{3,5,7,\ .\ .\ .\}$. For $x\in S$, let $\delta(x)$ be the unique integer such that $2^{\delta(x)}<x<2^{\delta(x)+1}$. For $a, b\in S$, define
$$
a*b=2^{\delta(a)-1}(b-3)+a.
$$
(a) Prove that if $a, b\in S$, then $a*b\in S.$

(b) Prove that if $a, b, c\in S$, then $(a*b)*c=a*(b*c)$ .

Solution: (a) is obvious, so we focus on (b). If $2^{m}<a<2^{m+1}, 2^{n}<b<2^{n+1}$, then

$a*b=2^{m-1}(b-3)+a\geq 2^{m-1}(2^{n}-2)+2^{m}+1=2^{m+n-1}+1$

and
$$
a*b\leq 2^{m-1}(2^{n+1}-4)+2^{m+1}-1=2^{m+n}-1
$$
so $\delta(a*b)=m+n-1$. If also $2^{p}<c<2^{p+1}$, then

$(a*b)*c=(2^{m-1}(b-3)+a)*c=2^{m+n-2}(c-3)+2^{m-1}(b-3)+a$

and

$a*(b*c)=a*(2^{n-1}(c-3)+b)=2^{m-1}(2^{n-1}(c-3)+b-3)+a=(a*b)*c.$

6. Let {\it ABCD} be a convex quadrilateral with an inscribed circle. If $\angle A=\angle B=2\pi/3, \angle D=\pi/2$ and $BC=1$, find the length of $AD.$

Solution: Let $I$ be the center of the inscribed circle. Then {\it ABI} is an equilateral triangle, $\angle BIC=105^{\circ}, \angle ICB=15^{\circ}, \angle AID=75^{\circ}, \angle IDA=45^{\circ}$, so
$$
AD=\frac{BI}{BC}\frac{AD}{AI}=\frac{\sin 15^{\mathrm{o}}}{\sin 105^{\circ}}\frac{\sin 75^{\circ}}{\sin 45^{\circ}}=\sqrt{2}\sin 15^{\circ}.
$$
55

7. Let $A$ be a subset of $\{0,1$, . . . , 1997$\}$ containing more than 1000 elements. Prove that $A$ contains either a power of 2, or two distinct integers whose sum is a power of 2.

Solution: Suppose $A$ did not verify the conclusion. Then $A$ would contain at most half of the integers from 51 to 1997, since they can be divided into pairs whose sum is 2048 (with 1024 left over); likewise, $A$ contains at most half of the integers from 14 to 50, at most half of the integers from 3 to 13, and possibly $0$, for a total of
$$
973+18+5+1=997
$$
integers.

8. Determine the number of natural numbers $n$ satisfying the following conditions:

(a) The decimal expansion of $n$ contains 1000 digits.

(b) All of the digits of $n$ are odd.

(c) The absolute value of the difference between any two adjacent digits of $n$ is 2.

Solution: Let $a_{n}, b_{n}, c_{n}, d_{n}, e_{n}$ be the number of $n$-digit numbers whose digits are odd, any two consecutive digits differing by 2, and ending in 1,3,5,7,9,respectively. Then
$$
(00001\ 00011\ 00011\ 00011\ 00001\ \left(\begin{array}{l}
a_{n}\\
b_{n}\\
c_{n}\\
d_{n}\\
e_{n}
\end{array}\right)=\left(\begin{array}{l}
a_{n+1}\\
b_{n+1}\\
c_{n+1}\\
d_{n+1}\\
e_{n+1}
\end{array}\right)
$$
Let $A$ be the square matrix in this expression. We wish to compute the eigenvalues of $A$, so suppose {\it Av} $=\lambda v$ for some vector $v= (v_{1},\ v_{2},\ v_{3},\ v_{4},\ v_{5})$ . Then
$$
v_{2}\ =\ \lambda v_{1}
$$
$$
v_{3}\ =\ \lambda v_{2}-v_{1}=(\lambda^{2}-1)v_{1}
$$
$$
v_{4}\ =\ \lambda v_{3}-v_{2}=(\lambda^{3}-2\lambda)v_{1}
$$
$$
v_{5}\ =\ \lambda v_{4}-v_{3}=(\lambda^{4}-3\lambda^{2}+1)v_{1}
$$
56

and $v_{4}=\lambda v_{5}$, so $\lambda^{5}-3\lambda^{3}+\lambda=\lambda^{3}-2\lambda$. Solving this polynomial gives $\lambda=0, \pm 1, \pm\sqrt{3}$. The corresponding eigenvectors $x_{1}$, {\it x}2, $x_{3}, x_{4}, x_{5}$ are are

$(1,0,\ -1,0,1), (1,1,0,\ -1,\ -1), (1,\ -1,0,1,\ -1), (1,\ \pm\sqrt{3},2,\ \pm\sqrt{3},1)$

and
$$
(1,\ 1,\ 1,\ 1,\ 1)=\frac{1}{3}x_{1}\frac{2+\sqrt{3}}{6}x_{4}+\frac{2-\sqrt{3}}{6}x_{5},
$$
so

$(a_{1000},\ b_{1000},\ c_{1000},\ d_{1000},\ e_{1000})$

$= 3^{999/2}\displaystyle \frac{2+\sqrt{3}}{6}(1,\ \sqrt{3},2,\ \sqrt{3},1)-\frac{2-\sqrt{3}}{6}(1,\ -\sqrt{3},2,\ -\sqrt{3},1) = (3^{499},2\cdot 3^{499},2\cdot 3^{499},2\cdot 3^{499},3^{499})$ .

Thus the answer is 8 $3^{499}.$

1.13 Italy

1. [Corrected] A rectangular strip of paper 3 centimeters wide and of infinite length is folded exactly once. What is the least possible area of the region where the paper covers itself?

Solution: Label the vertices of the triangle which covers itself $A, B, C$, where $AB$ is the folding axis and $\angle BAC$ is acute. Drop altitudes from $A$ and $B$ and label the points across the strip from them $A'$ and $B'$, respectively. Note that $\angle BAB$' folds onto $\angle BAC,$ so these angles have the same measure, which we call $x$. We consider two cases.

First, suppose $0<x\leq\pi/4$. Then $C$ is between $A'$ and $B, \angle ACA'= 2x$ and $\angle ABA'=x$. Then

[{\it ABC}] $=$ [{\it ABA}'] $-[ACA']$
$$
=\ \frac{1}{2}(3)(3\cot x)-\frac{1}{2}(3)(3\cot 2x)
$$
$$
=\ \frac{9}{2}(\cot x-\cot 2x)=\frac{9}{2}\csc 2x.
$$
Second, suppose $\pi/4\leq x<\pi/2$. Then $A'$ is between $B$ and $C, \angle ACA'=\pi-2x$ and $\angle ABA'=x$. Then again

[{\it ABC}] $=$ [{\it ABA}'] $+[ACA']$
$$
=\ \frac{1}{2}(3)(3\cot x)+\frac{1}{2}(3)(3\cot\pi-2x)
$$
$$
=\ \frac{9}{2}(\cot x-\cot 2x)=\frac{9}{2}\csc 2x.
$$
The minimum value of $\csc 2x$ is of course 1 (for $x=\pi/4$), so the minimum area is 9/2.

2. Let $f$ be a real-valued function such that for any real $x,$

(a) $f(10+x)=f(10-x)$ ; (b) $f(20+x)=-f(20-x)$ .

Prove that $f$ is odd $(f(-x)=-f(x))$ and periodic (there exists $T>0$ such that $f(x+T)=f(x)$).

58

Solution: Putting $x=n-10$ in (a), we get $f(n)=f(20-n)$ ; putting $x=n$ in (b), we get $f(20-n)=-f(n+20)$ . We conclude $f(n)=-f(n+20)$ , and likewise $f(n+20)=-f(n+40)$ . Thus $f(n+40)=f(n)$ , so $f$ is periodic with period 40. Also $-f(n)=f(20+n)=-f(20-n)=-f(n)$ , so $f$ is odd.

3. The positive quadrant of a coordinate plane is divided into unit squares by lattice lines. Is it possible to color some of the unit squares so as to satisfy the following conditions:

(a) each square with one vertex at the origin and sides parallel to the axes contains more colored than uncolored squares;

(b) each line parallel to the angle bisector of the quadrant at the origin passes through only finitely many colored squares?

Solution: It is possible as follows: on each diagonal line $y=x+D,$ color the $|D|+1$ squares closest to the axes.

Consider the line $y=x+D$, where $D\geq 0$. Along this line, the first colored square is in the first column and the $(D+1)-\mathrm{s}\mathrm{t}$ row, while the last is in the $(D+1)-\mathrm{s}\mathrm{t}$ column and the $(2D+1)-\mathrm{s}\mathrm{t}$ row. Since the squares to the right of this square (and above the line $y=x$) are part of diagonals which start lower and have fewer colored squares, none of them is colored. If we write $(i,j)$ for the square in row $i$ and column $j$, then $(i,j)$ is colored if and only if
$$
j\geq i,\ i\leq D+1\Rightarrow i\leq(j-i+1)\Rightarrow i\leq(j+1)/2
$$
or $j\leq(i+1)/2$. The total number of colored squares in an $n\times n$ square is then
$$
C_{n}=2(\sum_{k=1}^{n}\lfloor\frac{k+1}{2}\rfloor)-1.
$$
When $n$ is even, we find $C_{n}=\displaystyle \frac{1}{2}n^{2}+n-1$; when $n$ is odd, $C_{n}= \displaystyle \frac{1}{2}n^{2}+n-\frac{1}{2}$ . Thus $C_{n}>\displaystyle \frac{1}{2}n^{2}$ for all $n$ and the conditions are satisfied.

4. Let {\it ABCD} be a tetrahedron. Let $a$ be the length of $AB$ and let $S$ be the area of the projection of the tetrahedron onto a plane perpendicular to $AB$. Determine the volume of the tetrahedron in terms of $a$ and $S.$

69

Solution: Assign coordinates as follows:
$$
A=(0,0,0)\ ,\ B=(0,0,\ n)\ ,\ C=(0,\ b,\ c)\ ,\ D=(i,j,\ k)\ .
$$
The plane $z=0$ is perpendicular to $AB$, and the projection of the tetrahedron onto this plane is a triangle with vertices $A'=B'= (0,0,0), C'=(0,\ b,\ 0), D'=(i,j,\ 0)$ . This triangle has base $b$ and altitutde $i$, so $S=bi/2$ and $a=AB=n.$

To find the desired volume, consider the tetrahedron as a pyramid with base {\it ABC}. The plane of the base is $x=0$, and the altitude from $D$ has length $i$. The area of triangle {\it ABC} is $bn/2$, so the tetrahedron has volume $bin/6=Sa/3.$

5. Let $X$ be the set of natural numbers whose decimal representations have no repeated digits. For $n\in X$, let $A_{n}$ be the set of numbers whose digits are a permutation of the digits of $n$, and let $d_{n}$ be the greatest common divisor of the numbers in $A_{n}$. Find the largest possible value of $d_{n}.$

Solution: Suppose $n$ has 3 or more digits; let $AB$ be the last two digits. If these digits are transposed, the resulting number ends in $BA$ and also belongs to $A_{n}$; if both numbers are multiples of $d_{n},$ so is their difference $|AB-BA|=9|A-B|\leq 81.$

If $n$ has two digits, both nonzero, the above reasoning again applies; however, if the second digit is zero, $A_{n}$ only contains $n$ itself, and so $d_{n}=n$. The largest such number is 90, which is greater than 81 and so is the largest possible $d_{n}.$

1.14 Japan

1. Prove that among any ten points located in a circle of diameter 5, there exist two at distance less than 2 from each other.

Solution: Divide the circle into nine pieces: a circle of radius 1 concentric with the given circle, and the intersection of the remainder with each of eight equal sectors. Then one checks that two points within one piece have distance at most 2.

2. Let $a, b, c$ be positive real numbers. Prove the inequality
$$
\frac{(b+c-a)^{2}}{(b+c)^{2}+a^{2}}+\frac{(c+a-b)^{2}}{(c+a)^{2}+b^{2}}+\frac{(a+b-c)^{2}}{(a+b)^{2}+c^{2}}\geq\frac{3}{5},
$$
and determine when equality holds.

Solution:

slightly:

When all else fails , try brute force! First simplify
$$
\sum\frac{2ab+2ac}{a^{2}+b^{2}+c^{2}+2bc}\leq\frac{12}{5}\ .
$$
cyclic

Writing $s=a^{2}+b^{2}+c^{2}$, and clearing denominators, this becomes
$$
5s^{2}\sum\ ab+10s\sum a^{2}\ bc+20\sum a^{3}b^{2}c
$$
sym sym sym
$$
\leq\ 6s^{3}+6s^{2}\sum\ ab+12s\sum a^{2}bc+48a^{2}b^{2}c^{2}
$$
sym sym

which simplifies a bit to
$$
6s^{3}+s^{2}\sum\ ab+2s\sum a^{2}\ bc+8\sum a^{2}b^{2}c^{2}
$$
sym sym sym
$$
\geq\ 10s\sum a^{2}bc+20\sum a^{3}b^{2}c.
$$
sym sym

Now we multiply out the powers of $s$:
$$
\sum 3a^{6}+2a^{5}b-2a^{4}b^{2}+3a^{4}\ bc+2a^{3}b^{3}-12a^{3}b^{2}c+4a^{2}b^{2}c^{2}\geq 0.
$$
sym

61

The trouble with proving this is the $a^{2}b^{2}c^{2}$ with a {\it positive} coefficient, since it is the term with the most evenly distributed exponents. We save face using Schur's inequality (multiplied by $4abc:$)
$$
\sum 4a^{4}bc-8a^{3}b^{2}c+4a^{2}b^{2}c^{2}\geq 0,
$$
sym

which reduces our claim to
$$
\sum 3a^{6}+2a^{5}b-2a^{4}b^{2}-a^{4}bc+2a^{3}b^{3}-4a^{3}b^{2}c\geq 0.
$$
sym

Fortunately, this is a sum of four expressions which are nonnegative by weighted AM-GM:
$$
0\ \leq\ 2\sum(2a^{6}+b^{6})/3-a^{4}b^{2}
$$
sym
$$
0\ \leq\ \sum(4a^{6}+b^{6}+c^{6})/6-a^{4}bc
$$
sym
$$
0\ \leq\ 2\sum(2a^{3}b^{3}+c^{3}a^{3})/3-a^{3}b^{2}c
$$
sym
$$
0\ \leq\ 2\sum(2a^{5}b+a^{5}c+ab^{5}+ac^{5})/6-a^{3}b^{2}c.
$$
sym

Equality holds in each case if and only if $a=b=c.$

3. Let $G$ be a graph with 9 vertices. Suppose given any five points of $G$, there exist at least 2 edges with both endpoints among the five points. What is the minimum possible number of edges in $G$?

Solution: The minimum is 9, achieved by three disjoint 3-cycles. Let $a_{n}$ be the minimum number of edges in a graph on $n$ vertices satisfying the given condition. We show that $a_{n+1}\displaystyle \geq\frac{n+1}{n-1}a_{n}$. Indeed, given such a graph on $n+1$ vertices, let $l_{i}$ be the number of edges of the graph obtained by removing vertex $i$ and all edges incident to it. Then $l_{i}\geq a_{n}$; on the other hand, $l_{1}+\cdots+l_{n+1}=(n-1)a_{n+1}$ since every edge is counted for every vertex except its endpoints. The desired inequality follows.

Since $a_{5}=2$, we get $a_{6}\geq 3, a_{7}\geq 5, a_{8}\geq 7, a_{9}\geq 9.$

62

4. Let $A, B, C, D$ be four points in space not lying in a plane. Suppose $AX+BX+CX+DX$ is minimized at a point $X=X_{0}$ distinct from $A, B, C, D$. Prove that $\angle AX_{0}B=\angle CX_{0}D.$

Solution: Let $A, B, C, D$ and $P$ have coordinates $(x_{1},\ y_{1},\ z_{1})$ , . . ., $(x_{4},\ y_{4},\ z_{4})$ and $(x,\ y,\ z)$ . Then the function being minimized is
$$
f(P)=\sum_{i}\sqrt{(x-x_{i})^{2}+(y-y_{i})^{2}+(z-z_{i})^{2}}.
$$
At its minimum, its three partial derivatives are zero, but these are precisely the three coordinates of $u_{a}+u_{b}+u_{c}+u_{d}$, where $u_{a}$ is the unit vector $(P-A)/||P-A||$ and so on. Thus this sum is zero, and so $u_{a}\cdot u_{b}=u_{c}\cdot u_{d}$ for $P=X_{0}$, which proves the claim.

5. Let $n$ be a positive integer. Show that one can assign to each vertex of a $2^{n}$-gon one of the letters $A$ or $B$ such that the sequences of $n$ letters obtained by starting at a vertex and reading counterclockwise are all distinct.

Solution: Draw a directed graph whose vertices are the sequences of length $n-1$, with an edge between two sequences if the last $n-2$ letters of the first vertex match the first $n-2$ letters of the second. (Note: this is actually a graph with two loops, one from each of the words of all $A$'s or all $B$'s to themselves.) This graph has two edges into and out of each vertex, so there exists $\mathrm{a}$ (directed) path traversing each edge exactly once. We convert this into a cycle of the desired form by starting at any vertex, writing down the sequence corresponding to it, then appending in turn the last letter of each sequence we encounter along the path.

1.15 Korea

1. Show that among any four points contained in a unit circle, there exist two whose distance is at most $\sqrt{2}.$

Solution: If one of the four points lies at the center $O$ of the circle, the statement is trivial. Otherwise, label the points $P_{1}, P_{2}, P_{3}, P_{4}$ so that quadrilateral $Q_{1}Q_{2}Q_{3}Q_{4}$ is convex, where $Q_{i}$ is the intersection of the circle with $OP_{i}$. Then $\angle P_{1}OP_{2}+\angle P_{2}OP_{3}+\angle P_{3}OP_{4}+ \angle P_{4}OP_{1}\leq 2\pi$, so $\angle P_{i}OP_{i+1}\leq\pi/2$ for some $i$. Now segment $P_{i}P_{i+1}$ is contained in triangle $OQ_{i}Q_{i+1}$, so
$$
P_{i}P_{i+1}\ \leq\ \max(OQ_{i},\ Q_{i}Q_{i+1},\ Q_{i+1}O)
$$
$$
=\ \max(1,2\sin\angle Q_{i}OQ_{i+1})\leq\sqrt{2}.
$$
2. Let $f$ : $\mathbb{N}\rightarrow \mathbb{N}$ be a function satisfying

(a) For every $n\in \mathbb{N}, f(n+f(n))=f(n)$ .

(b) For some $n_{0}\in \mathbb{N}, f(n_{0})=1.$

Show that $f(n)=1$ for all $n\in \mathbb{N}.$

Solution: First, note that if $n\in \mathbb{N}$ and $f(n)=1$, then $f(n+1)= f(n+f(n))=f(n)=1$; so as $f(n_{0})=1, f(n)=1$ for all $n\geq n_{0}$ by induction. Let $S= \{\ n\in \mathbb{N}|f(n)\neq 1\}$; then $S$ has finitely many elements. If $ S\neq\emptyset$, let $N=\displaystyle \max S$; then $f(N+f(N))=f(N)\neq 1,$ so $N+f(N)\in S$, but $N+f(N)>N$, a contradiction. So $ S=\emptyset$ and $f(n)=1$ for every $n\in \mathbb{N}.$

3. Express $\displaystyle \sum_{k=1}^{n}\lfloor\sqrt{k}\rfloor$ in terms of $n$ and $a=\lfloor\sqrt{n}\rfloor.$

Solution: The closed form is $(n+1)a-a(a+1)(2a+1)/6.$

We will use Iverson's bracket convention: if $P$ is a statement, $[P]$ is 1 if $P$ is true, $0$ if $P$ is false. Note that $\lfloor\sqrt{k}\rfloor$ is equal to the number of positive integers whose squares are at most $k$, so $\displaystyle \lfloor\sqrt{k}\rfloor=\sum_{j=1}^{a}[j^{2}\leq k]$ if $ a\geq\lfloor\sqrt{k}\rfloor$. Therefore
$$
\sum\lfloor\sqrt{k}\rfloor n=\sum\sum^{n}[j^{2}a\leq k]=\sum\sum^{a}[j^{2}n\leq k].
$$
$$
k=1\ k=1j=1\ j=1k=1
$$
64

Now $\displaystyle \sum_{k=1}^{n}[j^{2}\leq k]$ counts the number of $k\in\{1,\ .\ .\ .\ ,\ n\}$ such that $k\geq j^{2}$; when $j\leq a, j^{2}\leq n$, so this number is $n+1-j^{2}$. So
$$
\sum_{k=1}^{n}\lfloor\sqrt{k}\rfloor=\sum_{j=1}^{a}n+1-j^{2}=(n+1)a-a(a+1)(2a+1)/6.
$$
4. [Corrected] Let $C$ be a circle touching the edges of an angle $\angle XOY\ ,$ and let $C_{1}$ be a circle touching the same edges and passing through the center of $C$. Let $A$ be the second endpoint of the diameter of $C_{1}$ passing through the center of $C$, and let $B$ be the intersection of this diameter with $C$. Prove that the circle centered at $A$ passing through $B$ touches the edges of $\angle XOY|.$

Solution: Let $T$ and $T_{1}$ be the centers of $C$ and $C_{1}$, and let $r$ and $r_{1}$ be their radii. Drop perpendiculars $TT', T_{1}T_{1}$, and $AA'$ to $OX$; then $TT'=r$ and $T_{1}T_{1}=r_{1}$. But $T_{1}$ is the midpoint of $AT,$ so $T_{1}T_{1}=(AA'+TT')/2$; therefore $AA'=2T_{1}T_{1}-TT'=2r_{1}-r.$ Also $AB=AT-BT=2r_{1}-r$, so the circle centered at $A$ with radius $AB$ touches $OX$ at $A'$. Similarly, this circle touches $OY$ .

5. Find all integers $x, y, z$ satisfying $x^{2}+y^{2}+z^{2}-2xyz=0.$

Solution: The only solution of this equation is $x=y=z=0.$

First, note that $x, y$, and $z$ cannot all be odd, as then $x^{2}+y^{2}+z^{2}- 2xyz$ would be odd and therefore non-zero. Therefore 2 divides {\it xyz}. But then $x^{2}+y^{2}+z^{2}=2xyz$ is divisible by 4; since all squares are $0$ or 1 $(\mathrm{m}\mathrm{o}\mathrm{d}\ 4), x, y$, and $z$ must all be even. Write $x=2x_{1}, y=2y_{1}, z=2z_{1}$; then we have $4x_{1}^{2}+4y_{1}^{2}+4z_{1}^{2}=16x_{1}y_{1}z_{1}$, or $x_{1}^{2}+y_{1}^{2}+z_{1}^{2}= 4x_{1}y_{1}z_{1}$. Since the right-hand side is divisible by 4, $x_{1}, y_{1}$, and $z_{1}$ must again be even, so we can write $x_{1}=2x_{2}, y_{1}=2y_{2}, z_{1}=2z_{2}$; plugging this in and manipulating we obtain $x_{2}^{2}+y_{2}^{2}+z_{2}^{2}=8x_{2}y_{2}z_{2}.$ In general, if $n\geq 1, x_{n}^{2}+y_{n}^{2}+z_{n}^{2}=2^{n+1}x_{n}y_{n}z_{n}$ implies that $x_{n}, y_{n}, z_{n}$ are all even, so we can write $x_{n}=2x_{n+1}, y_{n}=2y_{n+1}, z_{n}=2z_{n+1},$ which satisfy $x_{n+1}^{2}+y_{n+1}^{2}+z_{n+1}^{2}=2^{n+2}x_{n+1}y_{n+1}z_{n+1}$; repeating this argument gives us an infinite sequence of integers $(x_{1},\ x_{2},\ .\ .\ .)$ in which $x_{i}=2x_{i+1}$. But then $x=2^{n}x_{n}$, so $2^{n}$ divides $x$ for every $n\geq 1$; therefore we must have $x=0$. Similarly $y=z=0.$

65

Note: The substitution $x=yz -w$ reduces this problem to USAMO 76/3.

6. Find the smallest integer $k$ such that there exist two sequences $\{a_{i}\}, \{b_{i}\} (i=1,\ .\ .\ .\ ,\ k)$ such that

(a) For $i=1$, . . . , $k, a_{i}, b_{i}\in\{1$, 1996, 19962, . . .$\}.$

(b) For $i=1$, . . . , $k, a_{i}\neq b_{i}.$

(c) For $i=1$, . . . , $k-1, a_{i}\leq a_{i+1}$ and $b_{i}\leq b_{i+1}.$

(d) $\displaystyle \sum_{i=1}^{k}a_{i}=\sum_{i=1}^{k}b_{i}.$

Solution: The smallest such integer is 1997.

Suppose $\{a_{i}\}, \{b_{i}\}$ are two sequences satisfying these four conditions with $k\leq 1996$. The second condition tells us that $a_{1}\neq b_{1}$, so assume without loss of generality that $a_{1}<b_{1}$. By the first condition, there are $0\leq m<n$ such that $a_{1}=1996^{m}, b_{1}=1996^{n}$. Since $b_{i}\geq b_{1}$ for all $1\leq i\leq k$ (by the third condition) and each $b_{i}$ is a power of 1996, $\displaystyle \sum_{i=1}^{k}b_{i}$ is divisible by $1996^{n}$. Therefore by the fourth condition $\displaystyle \sum_{i=1}^{k}a_{i}$ is divisible by $1996^{n}$; letting $t$ denote the number of $j$'s for which $a_{j}=1996^{m}$, we have $t\displaystyle \cdot 1996^{m}\equiv\sum_{i=1}^{k}a_{i}\equiv 0 (\mathrm{m}\mathrm{o}\mathrm{d}\ 1996^{m+1})$ , so 1996 divides $t$ and $t\geq 1996$. But $ t\leq k\leq$ 1996, so we must have $t=k=1996$. Then $1996^{m+1}=\displaystyle \sum_{i=1}^{k}a_{i}= \displaystyle \sum_{i=1}^{k}b_{i}\geq\sum_{i=1}^{k}b_{1}=1996\cdot 1996^{n}=1996^{n+1}$, a contradiction as $m<n$. So we must have $k\geq 1997$. For $k=1997$, we have the example

$a_{1}=\cdots=a_{1996}=1, a_{1997}=1996^{2}, b_{1}= =b_{1997}=1996.$

7. Let $A_{n}$ be the set of all real numbers of the form $1+\displaystyle \frac{\ovalbox{\tt\small REJECT}_{1}}{\sqrt{2}}+\frac{\alpha_{2}}{(\sqrt{2})^{2}}+\cdots+ \displaystyle \frac{\ovalbox{\tt\small REJECT}_{n}}{(\sqrt{2})^{n}}$ ' where $\alpha_{j}\in\{-1,1\}$ for each $j$. Find the number of elements of $A_{n}$, and find the sum of all products of two distinct elements of $A_{n}.$

Solution: First we prove a lemma: for any $n\geq 1,$

$\displaystyle \{\frac{\beta_{1}}{2}+\frac{\beta_{2}}{4}+\cdots+\frac{\beta_{n}}{2^{n}}|\beta_{i}\in\{-1,1\}\}=$\{$\displaystyle \frac{j}{2^{n}}|j$ odd, $|j|<2^{n}$\}.

66

The proof is by induction on $n$; if $n=1$, both sides are the set $\displaystyle \{-\frac{1}{2}\ ,\ \frac{1}{2}\}$. If $n\geq 1, \beta_{i}\in\{-1,1\}$, let $j=2^{n-1}\beta_{1}+\cdots+2^{0}\beta_{n}$; then $j$ is odd and $\beta_{1}/2+\beta_{2}/4+\cdots+\beta_{n}/2^{n}=j/2^{n}$, and since

$|\displaystyle \frac{\beta_{1}}{2}+\frac{\beta_{2}}{4}+\cdots+\frac{\beta_{n}}{2^{n}}|\leq|\frac{\beta_{1}}{2}|+|\frac{\beta_{2}}{4}|+\cdots+|\frac{\beta_{n}}{2^{n}}|=\frac{1}{2}+\frac{1}{4}+\cdots+\frac{1}{2^{n}}<1,$

we must have $|j|<2^{n}$. Therefore the set on the left is contained in the set on the right. Now if $j$ is odd and $|j|<2^{n}$, either $(j-1)/2$ or $(j+1)/2$ is odd, since these are consecutive integers; let $j_{0}$ denote the odd one. Then $|j_{0}|\leq(|j|+1)/2\leq 2^{n-1}$, as $|j|\leq 2^{n}-1$; since $j_{0}$ is odd, $|j_{0}|<2^{n-1}$. So by the inductive hypothesis, there exist $\beta_{1},$ . . . , $\beta_{n-1}$ such that
$$
\frac{\beta_{1}}{2}+\frac{\beta_{2}}{4}+\ +\frac{\beta_{n-1}}{2^{n-1}}=\frac{j_{0}}{2^{n-1}}\ .
$$
Let $\beta_{n}=j-2j_{0}\in\{-1,1\}$; then
$$
\frac{\beta_{1}}{2}+\frac{\beta_{2}}{4}+\ +\frac{\beta_{n-1}}{2^{n-1}}+\frac{\beta_{n}}{2^{n}}=\frac{j_{0}}{2^{n-1}}+\frac{j-2j_{0}}{2^{n}}=\frac{j}{2^{n}}
$$
and the proof of the lemma is complete.

We can now conclude from this lemma that

$A_{n}=$\{$1+\displaystyle \frac{j}{2\lfloor n/2\rfloor}+\frac{k\sqrt{2}}{2\lceil n/2\rceil} j, k$ odd, $|j|<2^{\lfloor n/2\rfloor}, |k|<2^{\lceil n/2\rceil}$\},

so $A_{n}$ contains $2^{\lfloor n/2\rfloor}2^{\lceil n/2\rceil}=2^{n}$ elements. To compute the sum of all products of two distinct elements of $A_{n}$, we use the formula
$$
a,b\in A_{n}\sum_{a<b}\ ab=\frac{1}{2}((\sum_{a\in A_{n}}a)^{2}-\sum_{a\in A_{n}}a^{2})\ .
$$
Now we can pair off the elements $1+j/2^{\lfloor n/2\rfloor}+k\sqrt{2}/(2^{\lceil n/2\rceil})$ and $1-j/2^{\lfloor n/2\rfloor}-k\sqrt{2}/2^{\lceil n/2\rceil}$, so the average of $A_{n}$'s elements is 1; therefore

67
$$
\sum_{a\in A_{n}}\ a=|A_{n}|=2^{n}.
$$
Now if $X, Y$ are any finite sets of real numbers with $\displaystyle \sum_{x\in X}x= \displaystyle \sum_{y\in Y}y=0$, we have $\displaystyle \sum_{x\in X}\sum_{y\in Y}(1+x+y)^{2}=|X|\cdot|Y|+|Y|. \displaystyle \sum_{x\in X}x^{2}+|X|\cdot\sum_{y\in Y}y^{2}$ since the other three terms are all $0$ by assumption. Also the formula
$$
|j|<2m\sum_{jodd}j^{2}=\frac{1}{3}2m((2m)^{2}-1)
$$
can easily be proven by induction on $m$. So

$\displaystyle \sum_{a\in A_{n}}a^{2} = \displaystyle \sum_{jodd} \displaystyle \sum_{kodd} (1+\displaystyle \frac{j}{2\lfloor n/2\rfloor}+\frac{k\sqrt{2}}{2\lceil n/2\rceil})^{2}$
$$
|j|<2^{\lfloor n/2\rfloor}\ |k|<2^{\lceil n/2\rceil}
$$
$$
\ovalbox{\tt\small REJECT} 2^{\lfloor n/2\rfloor+\lceil n/2\rceil}+\ \sum\ \frac{\dot{j}^{2}2^{\lceil n/2\rceil}}{2^{2\lfloor n/2\rfloor}}+\ \sum\ \frac{2k^{2}2^{\lfloor n/2\rfloor}}{2^{2\lceil n/2\rceil}}
$$
$j$ {\it odd} $k$ {\it odd}
$$
|j|<2^{\lfloor n/2\rfloor}\ |k|<2^{\lceil n/2\rceil}
$$
$$
=\ 2^{n}+\frac{1}{3}(2^{\lceil n/2\rceil^{2(2^{2\lfloor n/2\rfloor}}}\lfloor n/2\rfloor_{2^{2\lfloor n/2\rfloor}}-1)_{+}
$$
$$
2\ \lfloor n/2\rfloor^{2^{\lceil n/2\rceil}(2^{2\lceil n/2\rceil}-}2^{2\lceil n/2\rceil-1}1))
$$
$$
=\ 2^{n}+\frac{1}{3}2^{n}(3-\frac{1}{2^{2\lfloor n/2\rfloor}}-\frac{1}{2^{2\lceil n/2\rceil-1}})
$$
$$
=\ 2^{n+1}-\frac{1}{3}(2^{n-2\lfloor n/2\rfloor}+2^{n-2\lceil n/2\rceil+1})
$$
$$
=\ 2^{n+1}-1
$$
by considering even $n$ and odd $n$ separately in the last step. So
$$
a,b\in A_{n}\sum_{a<b}\ ab=\frac{1}{2}((\sum_{a\in A_{n}}a)^{2}-\sum_{a\in A_{n}}a^{2})=\frac{1}{2}(2^{2n}-2^{n+1}+1)\ .
$$
68

8. In an acute triangle {\it ABC} with $AB\neq AC$, let $V$ be the intersection of the angle bisector of $A$ with $BC$, and let $D$ be the foot of the perpendicular from $A$ to $BC$. If $E$ and $F$ are the intersections of the circumcircle of $AVD$ with $CA$ and $AB$, respectively, show that the lines $AD,$ {\it BE}, $CF$ concur.

Solution: Since $\angle ADV=\pi/2$ and $A, D, V$ , $E, F$ are concyclic, $\angle BFV=\angle CEV=\pi/2$. Therefore triangles {\it BFV} and {\it BDA} are similar and triangles {\it CEV} and {\it CDA} are similar, so $BD/BF= AB/VB$ and $CD/CE=AC/VC$. But $AB/VB=AC/VC$ by the Angle Bisector theorem, so $BD/BF=CD/CE$. Also, since $\angle FAV=\angle VAE, AE=AF$. Therefore
$$
\frac{BD}{DC}\frac{CE}{EA}\frac{AF}{FB}=\frac{BD}{BF}/\frac{CD}{CE}=1
$$
and $AD,$ {\it BE}, $CF$ concur by Ceva's Theorem.

9. [Corrected] A word is a sequence of 8 digits, each equal to $0$ or 1. Let $x$ and $y$ be two words differing in exactly three places. Show that the number of words differing from each of $x$ and $y$ in at least five places is 38.

Solution: Assume without loss of generality that $x=00000000, y=00000111$. Then a word $z$ differs from each of $x$ and $y$ in at least five places if and only if $a+b\geq 5$ and $a+(3-b)\geq 5$, where $a$ is the number of $1$'s among the first five digits of $z$ and $b$ is the number of $1$'s among the last three digits of $z$. Adding these inequalities gives $2a\geq 7$, so we must have $a\geq 4$; the solutions are (4, 1), (4, 2), and (5, {\it b}) for $b\in\{0,1,2,3\}$ . The first two solutions give $\left(\begin{array}{l}
5\\
4
\end{array}\right) (\left(\begin{array}{l}
3\\
1
\end{array}\right)+\ \left(\begin{array}{l}
3\\
2
\end{array}\right))=30$ words for $z$, and the others give $2^{3}=8,$ so there are 38 words differing from each of $x$ and $y$ in at least five places.

10. Find all pairs of functions $f, g:\mathbb{R}\rightarrow \mathbb{R}$ such that

(a) if $x<y$, then $f(x)<f(y)$ ;

(b) for all $x, y\in \mathbb{R}, f(xy)=g(y)f(x)+f(y)$ .

Solution: The pairs $(f,\ g)$ given by

$f(x)=t(1-g(x)), g(x)=\left\{\begin{array}{ll}
x^{m}, & x\geq 0\\
-|x|^{m}, & x<0'
\end{array}\right.$

where $t<0, m>0$ are the only solutions.

Letting $x=0$ in item (b) gives $f(0)=f(0)g(y)+f(y)$ , so $f(y)= f(0)(1-g(y))$ . Let $t=f(0)$ ; then $f(y)=t(1-g(y))$ . Since $f$ is increasing, we cannot have $t=0$. Substituting this formula for $f$ in item (b) gives $t(1-g(xy))=g(y)t(1-g(x))+t(1-g(y))$ ; rearranging we get $1-g(xy)=g(y)(1-g(x))+1-g(y)=1-g(x)g(y)$ , or

$g(xy)=g(x)g(y)$ for all $x, y\in \mathbb{R}.$

Since $g=1-f/t, g$ is strictly monotone, so $g(1)\neq 0$; but $g(1)= g(1)^{2}$, so we must have $g(1)=1$. Therefore $g$ is increasing, so as $f$ is increasing, we must have $t<0$. So $g(x)>0$ for $x> 0$; define $h$ : $\mathbb{R}\rightarrow \mathbb{R}$ by $h=$ logo $g\mathrm{o}\mathrm{e}\mathrm{x}\mathrm{p}$. Then $h(x+y)= \log g(e^{x+y})=\log(g(e^{x})g(e^{y}))=\log g(e^{x})+\log g(e^{y})=h(x)+h(y)$ , $h(0)=\log g(e^{0})=0$, and $h$ is strictly increasing. Also $h(x+y)= h(x)+h(y)$ implies $h(nx)=nh(x)$ for $n\in \mathbb{N}$ and $h(-x)=-h(x)$ , so $h(\alpha x)=$ a{\it h}({\it x}) for $\alpha\in \mathbb{Q}$. Taking sequences of rationals $x_{i}$ approaching $x$ from below and $y_{i}$ approaching $x$ from above, and using monotonicity, shows that $h(x)=xh(1)$ for all $x$; let $m=h(1)$ . Then we must have $m>0$, as $h$ is increasing. So $g(x)=x^{m}$ for all $x>0.$ Now $g(-1)<0$, but $(g(-1))^{2}=g(1)=1$, so $g(-1)=-1$; therefore $g(-x)=-g(x)$ , so

$g(x)=\left\{\begin{array}{ll}
x^{m}, & x\geq 0\\
-|x|^{m}, & x<0
\end{array}\right.$

Also we have $f(x)=t(1-g(x))$ . It is easy to check that this pair $(f,\ g)$ is a solution for any $m>0>t.$

11. [Corrected] Let $a_{1}$, . . . , $a_{n}$ be positive numbers, and define
$$
a_{1}+\cdots+a_{n}
$$
$$
A\ =
$$
$$
n
$$
$$
G\ =\ (a_{1}\cdots a_{n})^{1/n}
$$
$$
H\ =\ \frac{n}{a_{1}^{-1}+\cdots+a_{n}^{-1}}\ .
$$
70

(a) If $n$ is even, show that $\displaystyle \frac{A}{H}\leq-1+2(\frac{A}{G})^{n}.$

(b) If $n$ is odd, show that $\displaystyle \frac{A}{H}\leq-\frac{n-2}{n}+\frac{2(n-1)}{n} (\displaystyle \frac{A}{G})^{n}.$

Solution: Note that
$$
\frac{G^{n}}{H}\ =\ a_{1}\cdots a_{n}(a_{1}^{-1}+\cdots a_{n}^{-1})n
$$
$$
=\ \frac{1}{n}\sum_{j=1}^{n}\frac{a_{1}\cdots a_{n}}{a_{j}}\leq(\frac{1}{n}\sum_{j=1}^{n}a_{j})^{n-1}=A^{n-1}
$$
by Maclaurin's inequality, so $\displaystyle \frac{A}{H}\leq (\displaystyle \frac{A}{G})^{n}$. Since $A\geq G, (\displaystyle \frac{A}{G})^{n}\geq 1,$ so $\displaystyle \frac{A}{H}\leq (\displaystyle \frac{A}{G})^{n}\leq-1+2(\frac{A}{G})^{n}$, proving part (a). For part (b), $\displaystyle \frac{A}{H}\leq (\displaystyle \frac{A}{G})^{n}\leq(\frac{A}{G})^{n}+\frac{n-2}{n}((\frac{A}{G})^{n}-1)=-\frac{n-2}{n}+\frac{2(n-1)}{n}(\frac{A}{G})^{n}.$

1.16 Poland

1. The positive integers $x_{1}$, . . . , $x_{7}$ satisfy the conditions
$$
x_{6}=144,\ x_{n+3}=x_{n+2}(x_{n+1}+x_{n})\ n=1,2,3,4.
$$
Compute $x_{7}.$

Solution: Multiplying the given equation for $n=1,2,3$ and canceling terms we get:
\begin{center}
$144=x_{3}(x_{1}+x_{2})(x_{2}+x_{3})(x_{3}+x_{4})$ . [1]
\end{center}
Also from the given equality we get the following inequalities:
$$
x_{4}\ =\ x_{3}(x_{2}+x_{1})\geq 2x_{3}
$$
$$
x_{5}\ =\ x_{4}(x_{3}+x_{2})\geq 2x_{3}^{2}
$$
$$
144\ =\ x_{6}\geq x_{5}(x_{4}+x_{3})\geq 2x_{3}^{2}(3x_{3})\Rightarrow 144\geq 6x_{3}^{3}\Rightarrow x_{3}=1,2.
$$
Case 1: $x_{3}=1.$

By [1] , $144=(x_{1}+x_{2})(x_{2}+1)(x_{1}+x_{2}+1)$ . The pairs of factors of 144 that are consecutive integers are (1,2) (2, 3), (3, 4), and (8, 9). Since $x_{1}+x_{2}$ and $x_{1}+x_{2}+1$ are factors of 144 that are consecutive integers and since $x_{1}+x_{2}\geq 2$, we have 3 subcases:

Subcase 1a) $x_{1}+x_{2}=2$ implies $6(x_{2}+1)=144$ and so $x_{2}=23$; $x_{1}=-21$; however, this is not a valid solution since the $x_{i}$ are positive integers.

Subcase $1\mathrm{b}$) $x_{1}+x_{2}=3$ implies $12(x_{2}+1)=144;x_{2}=11;x_{1}= -8$; again, this is invalid.

Subcase $1\mathrm{c}$) $x_{1}+x_{2}=8$ implies $72(x_{2}+1)=144;x_{2}=1;x_{1}= 7$. Testing this possible $\mathrm{s}\mathrm{o}\mathrm{l}\mathrm{u}\mathrm{t}\mathrm{i}\mathrm{o}\mathrm{n}:x_{4}=8, x_{5}=16, x_{6}=144$, so $(x_{1},\ x_{2},\ x_{3})=(7,1,1)$ works. The value of $x_{7}=3456.$

Case2: $x_{3}=2$

$144=2(x_{1}+x_{2})(x_{2}+2)(2x_{1}+2x_{2}+2)$ implies $36=(x_{1}+x_{2})(x_{2}+ 2)(x_{1}+x_{2}+1)$ . The pairs of factors of 36 that are consecutive integers are (1, 2), (2, 3), (3, 4). We then analyze as above:

Subcase $2\mathrm{a}$) $x_{1}+x_{2}=2$ implies $x_{2}=4;x_{1}=-2$; invalid solution.

72

Subcase $2\mathrm{b}$) $x_{1}+x_{2}=3$ implies $x_{2}=1, x_{1}=2$. Testing this solution:

$x_{4}=6, x_{5}=18, x_{6}=144$, so $(x_{1},\ x_{2},\ x_{3})=(2,1,2)$ also works. The value of $x_{7}=3456.$

The value of $x_{7}$ is thus 3456.

2. Solve the following system of equations in real numbers $x, y, z$:
$$
3(x^{2}+y^{2}+z^{2})\ =\ 1
$$
$$
x^{2}y^{2}+y^{2}z^{2}+z^{2}x^{2}\ =\ xyz(x+y+z)^{3}.
$$
Solution: First note that none of $x, y, z,\mathrm{o}\mathrm{r}(x+y+z)$ can equal $0$, for otherwise, by the second equation, $x=y=z=0$, which does not satisfy the first equation. Also note that $xyz(x+y+z)= \displaystyle \frac{x^{2}y^{2}+y^{2}z^{2}+z^{2}x^{2}}{(x+y+z)^{2}}\geq 0.$

For three real numbers $a, b, c$, we have $(a-b)^{2}+(a-c)^{2}+(b-c)^{2}\geq 0,$ or equivalently $a^{2}+b^{2}+c^{2}\geq ab +ac+bc$, with equality if and only if $a=b=c$. So
$$
1=3(x^{2}+y^{2}+z^{2})\ \geq\ (x+y+z)^{2}=\frac{x^{2}y^{2}+y^{2}z^{2}+z^{2}x^{2}}{xyx(x+y+z)}
$$
$$
\geq\ \frac{xy^{2}z+x^{2}yz+xyz^{2}}{xyz(x+y+z)}=1.
$$
Hence all the inequalities in the above expression must be equalitites, so we must have $x=y=z$. So from the first equation we find that the possible triples $(x,\ y,\ z)$ are $(\displaystyle \frac{1}{3},\ \frac{1}{3},\ \frac{1}{3})$ and $(\displaystyle \frac{-1}{3},\ \frac{-1}{3},\ \frac{-1}{3})$ . Both satisfy the given equations, so they are the solutions.

3. [Corrected] In a tetrahedron {\it ABCD}, the medians of the faces {\it ABD, ACD, BCD} from $D$ make equal angles with the corresponding edges {\it AB}, $AC, BC$. Prove that each of these faces has area less than or equal to the sum of the areas of the other two faces.

Solution: Let a $\leq 90^{\circ}$ be the common angle; first suppose $\alpha\neq 90^{\circ}$. If $a, b, c$ are the lengths of $BC, CA,$ {\it AB} and $m_{a}, m_{b}, m_{c}$ are the lengths of the medians from $BC, CA,$ {\it AB}, respectively, then

73

the area of triangle {\it DAB} is $\displaystyle \frac{1}{2}m_{c}c\sin\alpha$. Note that the absolute value of the dot product between the vectors $(D-(A+B)/2)$ and $A-B$ is $m_{c}c\cos\alpha=2\cot\alpha[DAB]$, but also equals $|DA^{2}-DB^{2}|$. Thus the three areas [{\it ABD}], [{\it ACD}], [{\it BCD}] are proportional to $|DA^{2}- DB^{2}|, |DA^{2}-DC^{2}|, |DB^{2}-DC^{2}|$, and the desired inequality is evident in this case.

Now suppose $\alpha=90^{\circ}$. Let $x=\angle ADB, y=\angle BDC, z=\angle ADC (0<x,\ y,\ z<180)$ . Note that since $x, y, z$ are the angles of a trihedral angle, we have $x+y>z, x+y>y, y+z>x$ and $x+y+z\leq 360.$ Since the area of triangle {\it ADB} is $(AD)(BD)(\sin x)/2$ (and similar relations hold for the areas of {\it BDC} and {\it ADC}) and $AD=BD=CD$ (since $\alpha=90$), we need to prove that $\sin x+\sin y>\sin z$ (and the analogous inequalities).

Using trigonometric identities, we have
$$
\sin x+\sin y=2\sin(\frac{x+y}{2})\cos(\frac{x-y}{2})\text{ , }\sin z=2\sin\frac{z}{2}\cos\frac{z}{2},
$$
so we need to prove
$$
\sin(\frac{x+y}{2})\cos(\frac{x-y}{2})>\sin\frac{z}{2}\cos\frac{z}{2}.
$$
Since $x+y+z\displaystyle \leq 360\Rightarrow\frac{x+y}{2}\leq 180-\frac{z}{2}$ . Note that since $0< \displaystyle \frac{z}{2}\leq\frac{x+y}{2}\leq 180-\frac{z}{2}$, we have $\displaystyle \sin(\frac{x+y}{2})>\sin\frac{z}{2}$. Also, since $\displaystyle \frac{x-y}{2}< \displaystyle \frac{z}{2}$, we have $\displaystyle \cos(\frac{x-y}{2})>\cos\frac{z}{2}$ since the cosine function is strictly decreasing on $[0,180]$ . Multiplying the two inequalities together (which is legal since the expressions involved are positive), we get $\displaystyle \sin(\frac{x+y}{2})\cos(\frac{x-y}{2})>\sin\frac{z}{2}\cos\frac{z}{2}$, as desired.

4. The sequence $a_{1}, a_{2}$, . . . is defined by
$$
a_{1}=0,\ a_{n}=a_{\lfloor n/2\rfloor}+(-1)^{n(n+1)/2}\ n>1.
$$
For every integer $k\geq 0$, find the number of $n$ such that

$2^{k}\leq n<2^{k+1}$ and $a_{n}=0.$

Solution: Let $B_{n}$denote the base 2 representation of $\mathrm{n}.$

First, we will prove by induction that $a_{n}$ is the number of 00 or 11 strings in $B_{n}$ minus the number of 01 or 10 strings in the $B_{n}$. For

74

the base case $\mathrm{n}=1$, we have $B_{1}=1$ so $a_{1}=0-0=0$. Assume that for $k=1,2, \ldots n-1, a_{k}$ is the number of 00 or 11 strings in $B_{k}$ minus the number of 01 or 10 strings in the $B_{k}$. First consider the cases when $n\equiv 0,3 (\mathrm{m}\mathrm{o}\mathrm{d}\ 4)$ . Then $B_{n}$ ends in 00 or 11, so $a_{n}$ equals one plus the number of 00 or 11 strings in all but the digit of $B_{n}$ minus the number of 01 or 10 strings in the all but the last digit of $B_{n}$. This latter number is given by $a_{\lfloor\frac{n}{2}\rfloor}$. Thus
$$
a_{n}=a_{\lfloor\frac{n}{2}\rfloor}+1=a_{\lfloor\frac{n}{2}\rfloor}+(-1)^{n(n+1)/2},
$$
as desired. Similarly, for $n\equiv 1,2 (\mathrm{m}\mathrm{o}\mathrm{d}\ 4)$ , we get (since $B_{n}$ ends in 01 or 10 and we are subtracting these)
$$
a_{\lfloor\frac{n}{2}\rfloor^{-1=a}\lfloor\frac{n}{2}\rfloor}+(-1)^{n(n+1)/2},
$$
completing the induction.

So for a given integer $k$, we need to find the number of integers $n$ such that $2^{k}\leq n<2^{k+1}$ and the number of 00 and 11 strings equals the number of 01 and 10 strings. Note that $B_{n}$ has $k+1$ digits. For each $B_{n}$, we construct a new sequence $C_{n}$ of $0$'s and $1$'s as follows: starting at the leftmost digit of $B_{n}$ and working our way to the penultimate digit of $B_{n}$, we add to our sequence $C_{n}$ the absolute value of the difference between that digit and the digit to its right. For example, $B_{11}=1011$ and $C_{11}=110$. Since a 00 or 11 string in $B_{n}$ yields a $0$ in $C_{n}$ and a 01 or 10 string in $B_{n}$ yields a 1 in $C_{n}$, we need to find the number of integers $n$ such that $2^{k}\leq n<2^{k+1}$ and the number of zeroes in $C_{n}$ equals the number of ones. Since $C_{n}$ has $k$ digits (one less than $B_{n}$), an equal number of zeroes and ones is impossible if $k$ is odd. If $k$ is even, then we can select $\displaystyle \frac{k}{2}$ of the $k$ digits equal to one; the rest of the $\displaystyle \frac{k}{2}$ places we can set equal to zero. From this sequence $C_{n}$, we can reconstruct $B_{n}$; it is easy to see that each sequence $C_{n}$ corresponds to a unique value of $B_{n}$ and hence a unique value of $n$. There are $\left\{\begin{array}{l}
k\\
\frac{k}{2}
\end{array}\right\}$ ways to select the placement of the ones. So the answer is $0$ if $k$ is odd, $\left\{\begin{array}{l}
k\\
\frac{k}{2}
\end{array}\right\}$otherwise.

5. Given a convex pentagon {\it ABCDE} with $DC=DE$ and $\angle BCD= \angle DEA=\pi/2$, let $F$ be the point on segment $AB$ such that $AF/BF= AE/BC$. Show that

$\angle FCE=\angle FDE$ and $\angle FEC=\angle BDC.$

75

Solution: Let $P$ be the intersection of $AE$ and $BC$, and note that $C, D, E, P$ are concyclic. Let $Q$ and $R$ denote the intersections of $DA$ and $DB$, respectively, with the circumcircle of {\it CDEP}. Let $G=QC\cap RE$. We have $\angle GCE=\angle ADE$ and $\angle GEC=\angle BDC.$ Also, by Pascal's Theorem for the hexagon {\it PCQDRE}, $A, G, B$ are collinear. So all we must do is show $AG/GB=AE/BC$ and then we conclude $F=G$. We do this via the law of sines:

$\displaystyle \frac{AG}{GB}$

$\ovalbox{\tt\small REJECT}\ovalbox{\tt\small REJECT} \underline{\sin\angle DCQ}\underline{QC}\underline{\sin\angle RBG} \sin\angle ERDRG\sin\angle GAQ \ovalbox{\tt\small REJECT}\ovalbox{\tt\small REJECT} \underline{CD}\underline{\sin\angle QRG}\underline{\sin\angle DBA}$ {\it DE} $\sin\angle GQR\sin\angle BAD \ovalbox{\tt\small REJECT}\ovalbox{\tt\small REJECT} \displaystyle \frac{\sin\angle ADE}{\sin\angle CDB}\frac{AD}{BD}=\frac{AE}{BC},$

where the last step follows from the fact that {\it ADE} and {\it BDC} are right triangles.

6. Consider $n$ points $(n\geq 2)$ on a unit circle. Show that at most $n^{2}/3$ of the segments with endpoints among the $n$ chosen points have length greater than $\sqrt{2}.$

Solution: Construct a graph on the given vertices by connecting every pair of points whose distance is greater than $\sqrt{2}$. We will show that no $K_{4}$ (the complete graph on 4 vertices) exists. Suppose a $K_{4}$ exists; call its vertices (in cyclic order) {\it ABCD}.

Edges of length greater than $\sqrt{2}$ subtend a minor arc of greater than $\pi/2$ radians. So {\it AB}, $BC, CD, DA$ each subtend minor arcs of greater than p/2 radians, and together they subtend more than $ 2\pi$ radians. This is a contradiction, since there are only $ 2\pi$ radians in a circle. That the graph has at most $\displaystyle \frac{n^{2}}{3}$ edges now follows from Turan's theorem.

1.17 Romania

1. In the plane are given a line $\Delta$ and three circles tangent to $\Delta$ and externally tangent to each other. Prove that the triangle formed by the centers of the circles is obtuse, and find all possible measures of the obtuse angle.

Solution: Let $a, b, c$ be the radii of the circles, $A, B, C$ the centers of the circles and $A', B', C'$ the projections of $A, B, C$, respectively, onto $\Delta$. Assume that $c\leq a, b$. Then
$$
A'B'=\sqrt{(a+b)^{2}-(a-b)^{2}}=2\sqrt{ab},\ B'C'=2\sqrt{bc},\ C'A'=2\sqrt{ca}.
$$
Thus $\sqrt{ab}=\sqrt{ac}+\sqrt{bc}$, and
$$
c=\frac{ab}{(\sqrt{a}+\sqrt{b})^{2}}\ .
$$
Put $x=\sqrt{c}/a, y=\sqrt{c}/b$, so that $x+y=1$. By the law of cosines in triangle {\it ABC}, we have

$\displaystyle \cos C=1-\frac{2ab}{(a+c)(b+c)}, \displaystyle \sin^{2}\frac{C}{2}=\frac{ab}{(a+c)(b+c)}=\frac{1}{(1+x^{2})(1+y^{2})}$ .

We see that $C$ is obtuse because $(1+x^{2})(1+y^{2})\leq 2$, or more precisely,
$$
(2x^{2}+2xy+y^{2})(x^{2}+2xy+2y^{2})\leq 2(x+y)^{4},
$$
which follows by expanding and applying AM-GM.

To find the possible measures of $C$, we must maximize $\sin^{2}C/2.$ Given $x+y=1$, we need to minimize $(1+x^{2})(1+y^{2})=2-2xy+$ ({\it xy})2. Now $xy\leq(x+y)^{2}/4=1/4$ and the function $f(z)=z^{2}-2z+2$ is decreasing on $[0,1/4]$. Thus the $C$ is maximized for $x=y=1/2,$ in which case $C=2\arcsin 16/25.$

2. Determine all sets $A$ of nine positive integers such that for any $ n\geq$ 500, there exists a subset $B$ of $A$, the sum of whose elements is $n.$

Solution: Note that if $A$ contains $x, y, z$ with $x+y=z$, then the sets $C\cup\{x,\ y\}$ and $C\cup\{z\}$ give the same sum for any $ C\subseteq$

77

$A-\{x,\ y,\ z\}$, giving at most $511-2^{6}=447$ nonzero sums. Therefore no such $x, y, z$ can exist. Similarly there do not exist $x, y, z, w$ in $A$ with $x+y+z=w.$

In order for 1 and 2 to occur as sums, 1 and 2 must belong to $A.$ Since then 3 does not belong, 4 must belong. Now 5,6,7 are excluded, so 8 belongs, which in turn excludes the numbers from 9 to 15, which in turn forces 16 to belong, which excludes the numbers from 17 to 30.

If we write the next element of $A$ as $32-a$, with $0\leq a\leq 1$, we can now write the numbers from 1 to $63-a$ as subset sums. Thus the next number must be $64-a-b$ for some $b\geq 0$. Likewise, we find that

$A=\{1,2,4,8,16,32-a,\ 64-a-b,\ 128-2a-b-c,\ 256-4a-2b-c-d\}$

for some $a, b, c, d\geq 0$, as long as the maximum sum $511-8a-4b- 2c-d$ is at least 500, that is, $8a+4b+2c+d\leq 11$. One now easily enumerates the 74 possible solutions.

3. Let $n\geq 4$ be an integer and $M$ a set of $n$ points in the plane, no three collinear and not all lying on a circle. Find all functions $f$ : $M\rightarrow \mathbb{R}$ such that for any circle $C$ containing at least three points of $M,$
$$
\sum_{P\in M\cap C}f(P)=0.
$$
Solution: For each two points $A, B$, the number $k$ of distinct circles through $A, B$ and another point of $M$ is at least 2. Summing the given condition over these circles, we get
$$
0=\sum_{P\in M}f(P)+(k-1)(f(A)+f(B))\ .
$$
Thus if $\displaystyle \sum f(P)$ is nonzero, it has the opposite sign as $f(A)+f(B)$ for any $A, B$, but the sum of $f(A)+f(B)$ over all $A, B$ is $(n-1)\displaystyle \sum f(P)$ , a contradiction Thus $\displaystyle \sum f(P)=0$, and also $f(A)=-f(B)$ for any two points $A, B$. Since $n\geq 3$, we conclude $f(P)=0$ for all $P\in M.$

4. Let {\it ABC} be a triangle, $D$ a point on side $BC$ and $\omega$ the circumcircle of {\it ABC}. Show that the circles tangent to $\omega, AD, BD$ and

78

to $\omega, AD, DC$, respectively, are tangent to each other if and only if $\angle BAD=\angle CAD.$

Solution: We focus first on the circle $\omega_{1}$ tangent to $\omega, AD, BD.$ Let this circle touch $AD$ at $M$, {\it BD} at $N$, let $r$ be its radius and let $K$ be its center. Also let $O$ be the center of $\omega$ and $R$ its radius; since $\omega_{1}$ is tangent to $\omega, OK=R-r$. If we put $\angle ADB=2\ovalbox{\tt\small REJECT}, \angle KDO=\beta,$ and $x=DN$, this becomes $ OK=R-x\tan\alpha$. Applying the law of cosines to triangle {\it ODK}, one extracts that
$$
x^{2}+2\lambda x+DO^{2}-R^{2}=0,\ \lambda=\frac{R\sin\ovalbox{\tt\small REJECT}-DO\cos\beta}{\cos\ovalbox{\tt\small REJECT}}.
$$
Likewise, if $y$ is the length of the tangent from $D$ to the other circle, we get that
$$
y^{2}+2\mu y+DO^{2}-R^{2}=0,\ \mu=\frac{R\cos\ovalbox{\tt\small REJECT}-DO\sin\beta}{\sin\alpha}.
$$
The circles are tangent if and only if these circles have a common root, which occurs if and only if $\lambda=\mu$. This in turn is equivalent to $ R\cos 2\ovalbox{\tt\small REJECT}=DO$ sin($\beta$--a) $=DO$ sin $\angle ADO$. By the Law of Sines in triangle {\it ADO}, this in turn is equivalent to $R\cos 2\alpha=$ {\it R}sin{\it DAO}. Now let $H$ be the foot of the altitude from $A$ to $BC$; then the condition becomes $\cos 2\alpha=\sin$ {\it HAD}, which is equivalent to $\angle HAD= \angle DAO$. Since $H$ and $O$ are isogonal conjugates, this occurs if and only if $AD$ is the angle bisector of $A.$

5. Let $VA_{1}\cdots A_{n}$ be a pyramid with $n\geq 4$. A plane $\Pi$ intersects the edges $VA_{1}$, . . . , $VA_{n}$ at $B_{1}$, . . . , $B_{n}$, respectively. Suppose that the polygons $A_{1}\cdots A_{n}$ and $B_{1}\cdots B_{n}$ are similar. Prove that $\Pi$ is parallel to the base of the pyramid.

Solution: By projecting to a pyramid with perpendicular lateral edges, one sees that if $A', B', C'$ are points on the lateral edges of triangular pyramid $V$ {\it ABC} (with vertex $V$), then
$$
\frac{[VA'B'C']}{[VABC]}=\frac{VA'}{VA}\frac{VB'}{VB}\frac{VC'}{VC}\ .
$$
Now let $x=VB_{i}/VA_{i}$, and let $S$ be the constant of similitude from $A_{1}\cdots A_{n}$ to $B_{1}\cdots B_{n}$. On the one hand, by the previous assertion,

89

$[V\ B_{i}B_{j}B_{k}]/[VA_{i}A_{j}A_{k}]=x_{i}x_{j}x_{k}$. On the other hand, if $h$ and $H$ are the lengths of the altitudes from $V$ to $B_{1}\cdots B_{n}$ and $A_{1}\cdots A_{n},$ respectively, then the ratio of volumes is also $S^{2}h/H$. We deduce from this (using that $n\geq 4$) that $x_{1}=\cdots=x_{n}=s^{2/3}(h/H)^{1/3}.$

6. Let $A$ be the set of positive integers representable in the form $a^{2}+2b^{2}$ for integers $a, b$ with $b\neq 0$. Show that if $p^{2}\in A$ for a prime $p$, then $p\in A.$

Solution: The case $p=2$ is easy, so assume $p>2$. Note that if $p^{2}=a^{2}+2b^{2}$, then $2b^{2}=(p-a)(p+a)$ . In particular, $a$ is odd, and since $a$ cannot be divisible by $p, \mathrm{g}\mathrm{c}\mathrm{d}(p-a,p+a)=\mathrm{g}\mathrm{c}\mathrm{d}(p-a,\ 2p)=2.$ By changing the sign of $a$, we may assume $p-a$ is not divisible by 4, and so
$$
|p+a|=m^{2},\ |p-a|=2n^{2}
$$
Since $|a|<|p|$, both $p+a$ and $p-a$ are actually positive, so we have $2p=m^{2}+2n^{2}$, so $p=n^{2}+2(m/2)^{2}.$

7. Let $p\geq 5$ be a prime and choose $k\in\{0,\ .\ .\ .\ ,\ p-1\}$. Find the maximum length of an arithmetic progression, none of whose elements contain the digit $k$ when written in base $p.$

Solution: We show that the maximum length is $p-1$ if $k\neq 0$ and $p$ if $k=0$. In a $p$-term arithmetic progression, the lowest nonconstant digit takes all values from $0$ to $p-1$. This proves the upper bound for $k\neq 0$, which is also a lower bound because of the sequence 1, . . . , $p-1$. However, for $k=0$, it is possible that when $0$ occurs, it is not actually a digit in the expansion but rather a leading zero. This can only occur for the first term in the progression, so extending the progression to $p+1$ terms would cause an honest zero to appear. Thus the upper bound for $k=0$ is $p$, and the sequence 1, $p+1$, . . . , $(p-1)p+1$ shows that it is also a lower bound. (Compare with USAMO 1995/1.)

8. Let $p, q, r$ be distinct prime numbers and let $A$ be the set
$$
A=\{p^{a}q^{b}r^{c}\ :\ 0\leq a,\ b,\ c\leq 5\}.
$$
Find the smallest integer $n$ such that any $n$-element subset of $A$ contains two distinct elements $x, y$ such that $x$ divides $y.$

80

Solution: Of course $n$ is one more than the size of the largest subset of $A$ no two of whose elements divide one another. The set of $p^{a}q^{b}r^{c}$ has this property and contains 27 elements. On the other hand, we can partition $A$ into 27 sequences such that each element of a sequence divides the next element. To see this, identify $A$ with the points $(a,\ b,\ c)$ in 3-space. We partition the points in the plane $c=0$ into six chains, running from $(a,\ 0,0)$ to $(a,\ 5,0)$ , and we might as well continue to $(a,\ 5,5)$ . This leaves a $5\times 6$ rectangle in the plane $c=1$, which we split into five chains which continue upward. Now $c=2$ is left with a $5\times 5$ rectangle which splits into five chains, and so on. We end up with $6+5+5+4+4+3=27$ chains, as desired. Thus $n=28.$

9. Let {\it ABCDEF} be a convex hexagon. Let $P, Q, R$ be the intersections of the lines $AB$ and $EF, EF$ and $CD, CD$ and $AB$, respectively. Let $S, T, U$ be the intersections of the lines $BC$ and {\it DE}, $DE$ and $FA, FA$ and $BC$, respectively. Show that if $AB/PR=CD/RQ= EF/QP$, then $BC/US=DE/ST=FA/TU.$

Solution: Both sets of equalities are equivalent to the vector equality
$$
A-B+C-D+E-F=0.
$$
10. Let $P$ be the set of points in the plane and $D$ the set of lines in the plane. Determine whether there exists a bijective function $f$ : $P\rightarrow D$ such that for any three collinear points $A, B, C$, the lines $f(A), f(B), f(C)$ are either parallel or concurrent.

Solution: No such function exists. We first note that the inverse images $A, B, C$ of three concurrent or parallel lines $l_{1}, l_{2}, l_{3}$ must be collinear; otherwise, any point on the lines {\it AB}, $BC, CA$ would map to a line concurrent with or parallel to these, as then would any point on a line through two such points, i.e. any point in the plane, a contradiction.

In particular, given two pencils $P_{1}, P_{2}$ of parallel lines, the inverse images of the lines in $P_{i}$ are the points on a line $l_{i}$. Note that $l_{1}$ and $l_{2}$ must be parallel, since no line lies in both $P_{1}$ and $P_{2}$. In other

81

words, the lines $l_{i}$ for pencils $P_{i}$ of parallel lines are all parallel to each other.

Pick a pencil $Q$ of concurrent lines, which corresponds to a line $m$ necessarily not parallel to $l_{1}$. Any line $m'$ parallel to $m$ also corresponds to a pencil of lines through a different point (if $m'$ corresponded to parallel lines, so would $m$). However, there is a line through the points corresponding to $m$ and $m'$, whose inverse image would be a point on both $m$ and $m'$, a contradiction.

11. [Corrected] Find all continuous functions $f$ : $\mathbb{R}\rightarrow[0,\ \infty$) such that for all $x, y\in \mathbb{R},$
$$
f(x^{2}+y^{2})=f(x^{2}-y^{2})+f(2xy)\ .
$$
Solution: From $x=y=0$ we get $f(0)=0$; then putting $x=0$ we get $f(t)=f(-t)$ for all $t\in \mathbb{R}$. Note that for any $a, b>0$, the equations $x^{2}-y^{2}=a$ and $2xy=b$ have a real solution, and so
$$
f(a)+f(b)=f(\sqrt{a^{2}+b^{2}})\ .
$$
Putting $g(x)=f(\sqrt{x})$ , we find that $g(a+b)=g(a)+g(b)$ for all $a, b\geq 0$. Since $g$ is continuous, a standard argument shows that $g(a)=ca$ for some constant $c\geq 0$, so $f(x)=cx^{2}$

12. Let $n\geq 2$ be an integer and $P(x)=x^{n}+a_{n-1}x^{n-1}+\cdots+a_{1}x+1$ be a polynomial with positive integer coefficients. Suppose that $a_{k}= a_{n-k}$ for $k=1,2$, . . . , $n-1$. Prove that there exist infinitely many pairs $x, y$ of positive integers such that $x|P(y)$ and $y|P(x)$ .

Solution: There exists at least one such pair, namely $(1,\ P(1))$ . If there were only finitely many such pairs, we could choose a pair $(x,\ y)$ with $y$ maximal. However, we claim that for any pair $(x,\ y)$ with $x|P(y)$ and $y|P(x)$ , the pair $(y,\ P(y)/x)$ also has this property, or equivalently, that $y|P(P(y)/x)$ . Indeed, the given conditions imply $P(P(y)/x)=(x/P(y))^{n}P(x/P(y))$ . Since $P(y)\equiv 1 (\mathrm{m}\mathrm{o}\mathrm{d}\ y)$ , we conclude that $P(P(y)/x)\equiv x^{n}P(x)\equiv 0 (\mathrm{m}\mathrm{o}\mathrm{d}\ y)$ .

Moreover, $P(y)/x>y^{2}/x>y$, so the pair $(y,\ P(y)/x)$ has larger second member than does $(x,\ y)$ , a contradiction. Thus infinitely many pairs exist.

82

13. Let $P(x), Q(x)$ be monic irreducible polynomials over the rational numbers. Suppose $P$ and $Q$ have respective roots a and $\beta$ such that $\alpha+\beta$ is rational. Prove that the polynomial $P(x)^{2}-Q(x)^{2}$ has a rational root.

Solution: Let $\alpha+\beta=q$. Then $P(x)$ and $Q(q-x)$ are irreducible polynomials with rational coefficients sharing a root; it follows that $P(x)=cQ(q-x)$ for some constant $c$, and it is evident that in fact $c=(-1)^{\mathbb{d}\mathbb{e}\mathbb{g}Q}$. In particular, $P(x)^{2}=Q(q-x)^{2}$, and so $q/2$ is a root of $P(x)^{2}-Q(x)^{2}.$

14. Let $a>1$ be an integer. Show that the set
$$
\{a^{2}+a-1,\ a^{3}+a^{2}-1,\ .\ .\ .\}
$$
contains an infinite subset, any two members of which are relatively prime.

Solution: We show that any set of $n$ elements of the set which are pairwise coprime can be extended to a set of $n+1$ elements. For $n=1$, note that any two consecutive terms in the sequence are relatively prime. For $n>1$, let $N$ be the product of the numbers in the set so far; then {\it a}f({\it N})$+$1 $+${\it a}f({\it N}) $-1\equiv a (\mathrm{m}\mathrm{o}\mathrm{d}\ N)$ , and so can be added (since every element of the sequence is coprime to $a, N$ is as well).

15. Find the number of ways to color the vertices of a regular dodecagon in two colors so that no set of vertices of a single color form a regular polygon.

Solution: Call the colors red and blue. Obviously we need only keep track of equilateral triangles and squares. The vertices of the dodecagon form four triangles, each of which can be colored in 6 nonmonochromatic ways. Thus we have $6^{4}=1296$ colorings with no monochromatic triangles.

We now consider how many of these colorings have one, two or three monochromatic squares . The number with a given square monochromatic is $2\times 3^{4}=162$, since if the square is colored red, the other two vertices in each triangles can be red-blue, blue-red, or

83

blue-blue. For two given squares, we have one choice in each triangle if the squares are the same color, and two otherwise, for a total of $2+2\times 2^{4}=34$. For all three squares, we need that the squares themselves are not all the same color, for a total of 6 colorings.

By inclusion-exclusion, the number of colorings with no monochromatic triangles or squares is
$$
1296-3\times 162+3\times 34-6=906.
$$
16. Let $\Gamma$ be a circle and $AB$ a line not meeting $\Gamma$. For any point $P\in\Gamma,$ let $P'$ be the second intersection of the line $AP$ with $\Gamma$ and let $f(P)$ be the second intersection of the line $BP'$ with $\Gamma$. Given a point $P_{0},$ define the sequence $P_{n+1}=f(P_{n})$ for $n\geq 0$. Show that if a positive integer $k$ satisfies $P_{0}=P_{k}$ for a single choice of $P_{0}$, then $P_{0}=P_{k}$ for all choices of $P_{0}.$

Solution: Perform a projective transformation taking $AB$ to infinity. This takes $\Gamma$ to an ellipse, and a suitable affine transformation takes this ellipse to a circle while keeping $AB$ at infinity. Now the map $P\rightarrow P'$ is a reflection across the diameter through the point $A$ (at infinity), while $P'\rightarrow f(P)$ is a reflection across the diameter through $B$. Thus $P\rightarrow f(P)$ is a rotation, so if $P_{0}=P_{k}$ holds for a single $P_{0}$, it holds for all $P_{0}.$

1.18 Russia

1. Show that the numbers from 1 to 16 can be written in a line, but not in a circle, so that the sum of any two adjacent numbers is a perfect square.

Solution: If the numbers were in a circle with 16 next to $x$ and $y,$ then $16+1\leq 16+x,  16+y\leq$ 16$+$15, forcing $16+x=16+y=25,$ a contradiction. They may be arranged in a line as follows:
$$
16,\ 9,\ 7,\ 2,\ 14,\ 11,\ 5,\ 4,\ 12,\ 13,\ 3,\ 6,\ 10,\ 15,\ 1,\ 8.
$$
2. On equal sides $AB$ and $BC$ of an equilateral triangle {\it ABC} are chosen points $D$ and $K$, and on side $AC$ are chosen points $E$ and $M$, so that $DA+AE=KC+CM=AB$. Show that the angle between the lines $DM$ and $KE$ equals $\pi/3.$

Solution: We have $CE=AC-AE=AD$ and likewise $CK= AM$. Thus a $2\pi/3$ rotation about the center of the triangle takes $K$ to $M$ and $E$ to $D$, proving the claim.

3. A company has 50000 employees. For each employee, the sum of the numbers of his immediate superiors and of his immediate inferiors is 7. On Monday, each worker issues an order and gives copies of it to each of his immediate inferiors (if he has any). Each day thereafter, each worker takes all of the orders he received on the previous day and either gives copies of them to all of his immediate inferiors if he has any, or otherwise carries them out himself. It turns out that on Friday, no orders are given. Show that there are at least 97 employees who have no immediate superiors.

Solution: Let $k$ be the number of �top� employees, those with no superiors. Their orders reach at most $7k$ employees on Monday, at most 6 $7k$ on Tuesday, and at most 36 $7k$ on Wednesday. On Thursday, each employee receiving an order has no inferiors, so each has 7 superiors, each giving at most 6 orders, and so there are at most 216 $7k/7$ employees receiving orders. We conclude that
$$
50000\leq k+7k+42k+252k+216k=518k
$$
and $k\geq 97.$

85

4. The sides of the acute triangle {\it ABC} are diagonals of the squares $K_{1}, K_{2}, K_{3}$. Prove that the area of {\it ABC} is covered by the three squares.

Solution: Let $I$ be the incenter of {\it ABC}. Since the triangle is acute, $\angle IAB, \angle IBA<45^{\circ}$, so the triangle {\it IAB} is covered by the square with diagonal $AB$, and likewise for {\it IBC} and {\it ICA}.

5. The numbers from 1 to 37 are written in a line so that each number divides the sum of the previous numbers. If the first number is 37 and the second number is 1, what is the third number?

Solution: The last number $x$ must divide the sum of all of the numbers, which is $37\times 19$; thus $x=19$ and the third number, being a divisor of 38 other than 1 or 19, must be 2.

6. Find all pairs of prime numbers $p, q$ such that $p^{3}-q^{5}=(p+q)^{2}.$

Solution: The only solution is (7, 3). First suppose neither $p$ nor $q$ equals 3. If they are congruent modulo 3, the left side is divisible by 3 while the right is not; if they are not congruent modulo 3, the right side is divisible by 3 while the left side is not, so there are no such solutions. If $p=3$, we have $q^{5}<27$, which is impossible. Thus $q=3$ and $p^{3}-243=(p+3)^{2}$, whose only integer root is $p=7.$

7. (a) In Mexico City, to restrict traffic flow, for each private car are designated two days of the week on which that car cannot be driven on the streets of the city. A family needs to have use of at least 10 cars each day. What is the smallest number of cars they must possess, if they may choose the restricted days for each car?

(b) [Corrected] The law is changed to restrict each car only one day per week, but the police get to choose the days. The family bribes the police so that the family names two days for each car in succession, and the police immediately restrict the car for one of those days. Now what is the smallest number of cars the family needs to have access to 10 cars each day?

Solution:

86

(a) If $n$ cars are used, the number $5n$ of free days per car must be at least $7\times 10$ so $n\geq 14$. In fact 14 cars suffice: restrict four on Monday and Tuesday, four on Wednesday and Thursday, two on Friday and Saturday, two on Saturday and Sunday, and two on Friday and Sunday.

(b) 12 cars are needed. First let us show $n\leq 11$ cars do not suffice. As there are $n$ restricted days, on some day at most $6n/7$ cars will be available, but for $n\leq 11,6n/7<10$. For $n=12$, the family simply needs to offer for each car in succession two days which have not yet been restricted for two cars, which they can certainly do. (I don't know whether 12 still works if the family must offer their options all at once.)

8. A regular 1997-gon is divided by nonintersecting diagonals into triangles. Prove that at least one of the triangles is acute.

Solution: The circumcircle of the 1997-gon is also the circumcircle of each triangle; since the center of the circle does not lie on any of the diagonals, it must lie inside one of the triangles, which then must be acute.

9. On a chalkboard are written the numbers from 1 to 1000. Two players take turns erasing a number from the board. The game ends when two numbers remain: the first player wins if the sum of these numbers is divisible by 3, the second player wins otherwise. Which player has a winning strategy?

Solution: The second player has a winning strategy: if the first player erases $x$, the second erases $1001-x$. Thus the last two numbers will add up to 1001.

10. 300 apples are given, no one of which weighs more than 3 times any other. Show that the apples may be divided into groups of 4 such that no group weighs more than 11/2 times any other group.

Solution: Sort the apples into increasing order by weight, and pair off the heaviest and lightest apples, then the next heaviest and next lightest, and so on. Note that no pair weighs more than twice any other; if $a, d$ and $b, c$ are two groups with $a\leq b\leq c\leq d,$

87

then $a+d\leq 4a\leq 2b+2c, b+c\leq 3a+d\leq 2a+2d$. Now pairing the heaviest and lightest pairs gives foursomes, none weighing more than 3/2 times any other; if $e\leq f\leq g\leq h$ are pairs, then $e+h\leq 3e\leq(3/2)(f+g), f+g\leq 2e+h\leq(3/2)(e+h)$ .

11. In Robotland, a finite number of (finite) sequences of digits are forbidden. It is known that there exists an infinite decimal fraction, not containing any forbidden sequences. Show that there exists an infinite periodic decimal fraction, not containing any forbidden sequences.

Solution: Let $N$ be the maximum length of a forbidden sequence. We are given an infinite decimal containing no forbidden sequence; it must contain two identical copies of some subsequence of length $N+1$. Repeating the part of the decimal from the digit after the first copy to the end of the second copy gives an infinite periodic decimal containing no forbidden sequences.

12. (a) [Corrected] A collection of 1997 distinct numbers has the property that if each number is subtracted from the sum of all of the numbers, the same collection of numbers is obtained. Prove that the product of the numbers is $0.$

(b) A collection of 100 distinct numbers has the same property. Prove that the product of the numbers is positive.

Solution:

(a) Let $M$ be the sum of the numbers. Replacing each number $a$ by $M-a$ preserves the collection, so $\displaystyle \sum a=\sum(M-a)$ , so $2M=1997M$ and so $M=0$. Thus the numbers must divide into pairs $a, -a$, and so there must be a $0$ left over.

(b) None of the numbers can be $0$, since otherwise two would be and the numbers are assumed distinct. Thus the product is of 50 positive numbers and 50 negative numbers, which is positive.

13. Given triangle {\it ABC}, let $A_{1}, B_{1}, C_{1}$ be the midpoints of the broken lines {\it CAB, ABC, BCA}, respectively. Let $l_{A}, l_{B}, l_{C}$ be the respective lines through $A_{1}, B_{1}, C_{1}$ parallel to the angle bisectors of $A, B, C.$ Show that $l_{A}, l_{B}, l_{C}$ are concurrent.

88

Solution: Let $BC$ meet the bisector of $\angle A$ at $A_{2}$ and $l_{A}$ at $A_{3}.$ Define $B_{2}, B_{3}, C_{2}, C_{3}$ similarly. If $A_{1}$ is on $CA$ (the other case is similar), then

$A_{3}B=CB-CA_{3}=CA_{2}+BA_{2}-CA_{3}=CA_{2}(1+\displaystyle \frac{AB}{AC})-CA_{3}$

by the angle bisector theorem. Now
$$
A_{3}B\ =\ CA_{3}(\frac{CA_{2}}{CA_{3}}(1+\frac{AB}{AC})-1)
$$
$$
=\ CA_{3}(\frac{AC}{CA_{1}}(1+\frac{AB}{AC})-1)=CA_{3},
$$
since $AC+AB=2CA_{1}$ by assumption. Thus $A_{3}, B_{3}, C_{3}$ are the midpoints of the sides of {\it ABC}, and so the given lines are concurrent (at the image of the incenter of {\it ABC} under the homothety taking $A$ to $A_{3}, B$ to $B_{3}, C$ to $C_{3})$ .

14. The MK-97 calculator can perform the following three operations on numbers in its memory:

(a) Determine whether two chosen numbers are equal.

(b) Add two chosen numbers together.

(c) For chosen numbers $a$ and $b$, find the real roots of $x^{2}+ax+b,$ or announce that no real roots exist.

The results of each operation are accumulated in memory. Initially the memory contains a single number $x$. How can one determine, using the MK-97, whether $x$ is equal to 1?

Solution: First compute $2x$ and compare it to $x$. If they are different and $x\neq 1$, the roots of the polynomial $y^{2}+2xy+x=0$ are $-x\pm\sqrt{x^{2}-x}$, which are either distinct or not real. Thus comparing these roots if they exist tells us whether $x=1.$

15. The circles $S_{1}$ and $S_{2}$ intersect at $M$ and $N$. Show that if vertices $A$ and $C$ of a rectangle {\it ABCD} lie on $S_{1}$ while vertices $B$ and $D$ lie on $S_{2}$, then the intersection of the diagonals of the rectangle lies on the line $MN.$

99

Solution: By the radical axis theorem, the radical axes of $S_{1}, S_{2}$ and the circumcircle of {\it ABCD} intersect. But these lines are none other than $AC, BD, MN$, respectively.

16. [Corrected] For natural numbers $m, n$, show that $2^{n}-1$ is divisible by $(2^{m}-1)^{2}$ if and only if $n$ is divisible by $m(2^{m}-1)$ .

Solution: Since
$$
2^{kn+d}-1\equiv 2^{d}-1\ (\mathrm{m}\mathrm{o}\mathrm{d}\ 2^{n}-1)\ ,
$$
we have $2^{m}-1$ divides $2^{n}-1$ if and only if $m$ divides $n$. Thus in either case, we must have $n=km,$ in which case
$$
\frac{2^{km}-1}{2^{m}-1}=1+2^{m}+\ +2^{m(k-1)}\equiv k\ (\mathrm{m}\mathrm{o}\mathrm{d}\ 2^{m}-1)\ .
$$
The two conditions are now that $k$ is divisible by $2^{m}-1$ and that $n$ is divisible by $m(2^{m}-1)$ , which are equivalent.

17. [Corrected] Can three mutually adjacent faces of a cube of side length 4 be covered with 16 $1\times 3$ rectangles?

Solution: No. Color each face black and white as below, with the isolated black square on the unit cube adjacent to all three faces. Then of the 27 black unit squares, an even number are covered by each rectangle, so they cannot be exactly covered by the rectangles.
$$
B\ B\ W\ B
$$
$$
W\ W\ W\ W
$$
$$
B\ B\ W\ B
$$
$$
B\ B\ W\ B
$$
18. The vertices of triangle {\it ABC} lie inside a square $K$. Show that if the triangle is rotated $180^{\circ}$ about its centroid, at least one vertex remains inside the square.

Solution: Suppose the square has vertices as $(0,0), (0,1), (1,1)$ , $($1, $0)$ , and without loss of generality suppose the centroid is at $(x,\ y)$

90

with $x, y\leq 1/2$. There must be at least one vertex of the triangle on the same side as $(0,0)$ of the line through $(2x,\ 0)$ and $(0,2y)$ ; the rotation of this vertex remains inside the square.

19. Let $S(N)$ denote the sum of the digits of the natural number $N.$ Show that there exist infinitely many natural numbers $n$ such that $S(3^{n})\geq S(3^{n+1})$ .

Solution: If $S(3^{n})<S(3^{n+1})$ for large $n$, we have (since powers of 3 are divisible by 9, as are their digit sums) $S(3^{n})\leq S(3^{n+1})-9.$ Thus $S(3^{n})\geq 9(n-c)$ for some $c$, which is eventually a contradiction since for large $n, 3^{n}<10^{n-c}.$

20. The members of Congress form various overlapping factions such that given any two (not necessarily distinct) factions $A$ and $B$, the complement of $A\cup B$ is also a faction. Show that for any two factions $A$ and $B, A\cup B$ is also a faction.

Solution: By putting $A=B$, we see the complement of any faction is a faction. Thus for any factions $A$ and $B$, the complement of $A\cup B$ is a faction, so $A\cup B$ is also.

21. Show that if $1<a<b<c$, then
$$
\log_{a}(\log_{a}b)+\log_{b}(\log_{b}c)+\log_{c}(\log_{c}a)>0.
$$
Solution: Since $\log_{a}b>1, \log_{a}\log_{a}b>\log_{b}\log_{a}b$. Since $\log_{c} a< 1, \log_{c}\log_{c} a>\log_{b}\log_{c}a$. Thus the left side of the given inequality exceeds
$$
\log_{b}(\log_{a}b\log_{b}c\log_{c}a)=0.
$$
22. Do there exist pyramids, one with a triangular base and one with a convex $n$-sided base $(n\geq 4)$ , such that the solid angles of the triangular pyramid are congruent to four of the solid angles of the $n$-sided pyramid?

Solution: The pyramids do not exist. Suppose {\it ABCD} and $SA_{1}A_{2}\cdots A_{n}$ are triangular and $n$-sided pyramids, such that the

91

solid angles at $A, B, C, D$ are congruent to those at $A_{i}, A_{j}, A_{k}, A_{l}.$ The face angles at the latter vertices must then add up to 4 $180^{\circ}= 720^{\circ}$. By the spherical triangle inequality,
$$
\angle A_{m-1}A_{m}A_{m+1}<\angle A_{m-1}A_{m}S+\angle A_{m+1}A_{m}S,
$$
so

$\angle A_{i-1}A_{i}A_{i+1}+\angle A_{j-1}A_{j}A_{j+1}+\angle A_{k-1}A_{k}A_{k+1}+\angle A_{l-1}A_{l}A_{l+1}<360^{\circ}.$

But the sum of the angles of the polygon $A_{1}A_{2}\cdots A_{n}$ is $180^{\circ}(n-2)$ , so the sum of the remaining angles is at least $180^{\circ}(n-4)$ , contradicting convexity of the base.

23. For which $\alpha$ does there exist a nonconstant function $f$ : $\mathbb{R}\rightarrow \mathbb{R}$ such that
\begin{center}
$f(\alpha(x+y))=f(x)+f(y)$ ?
\end{center}
Solution: For $\alpha=1$, we may set $f(x)=x$. For any other $\alpha,$ putting $y=\alpha x/(1-\alpha)$ gives $f(y)=f(x)+f(y)$ , so $f(x)=0$ for all $x$, which is not allowed. So only $\ovalbox{\tt\small REJECT}=1$ works.

24. Let $P(x)$ be a quadratic polynomial with nonnegative coefficients. Show that for any real numbers $x$ and $y$, we have the inequality
$$
P(xy)^{2}\leq P(x^{2})P(y^{2})\ .
$$
Solution: This actually holds for any polynomial with nonnegative coefficients. If $P(x)=a_{0}x^{n}+\cdots+a_{n}x^{0}$, then
$$
(a_{0}x^{2n}+\cdots+a_{0})(a_{0}y^{2n}+\cdots+a_{0})\geq(a_{0}x^{n}y^{n}+\cdots+a_{0})^{2}
$$
by the Cauchy-Schwarz inequality.

25. Given a convex polygon $M$ invariant under a $90^{\circ}$ rotation, show that there exist two circles, the ratio of whose radii is $\sqrt{2}$, one containing $M$ and the other contained in $M.$

Solution: Let $O$ be the center of the rotation and $A_{1}$ a vertex at maximum distance $R$ from $O$. If $A_{1}$ goes to $A_{2}$ under the rotation, $A_{2}$ to $A_{3}, A_{3}$ to $A_{4}$, and $A_{4}$ to $A_{1}$, then $A_{1}A_{2}A_{3}A_{4}$ is a square

92

with center $O$ contained entirely in $M$. Thus the circle with radius $R/\sqrt{2}$ is contained in the square and thus in $M$, and the circle with radius $R$ contains $M.$

26. (a) The Judgment of the Council of Sages proceeds as follows: the king arranges the sages in a line and places either a white hat or a black hat on each sage's head. Each sage can see the color of the hats of the sages in front of him, but not of his own hat or of the hats of the sages behind him. Then one by one (in an order of their choosing), each sage guesses a color. Afterward, the king executes those sages who did not correctly guess the color of their own hat.

The day before, the Council meets and decides to minimize the number of executions. What is the smallest number of sages guaranteed to survive in this case?

(b) The king decides to use three colors of hats: white, black and red. Now what is the smallest number of sages guaranteed to survive?

Solution: All but one sage can be saved, for any number $n$ of hat colors. Represent each hat color by a different integer among $\{$1, . . . , $n\}$. If the rearmost sage goes first and announces the hat color congruent modulo $n$ to the sum of the colors of the other hats, and the remaining sages answer from back to front, each can deduce his hat color by subtracting the hats he sees and the (correct) guesses before him from the total.

27. [Corrected] The lateral sides of a box with base $a\times b$ and height $c$ (where $a, b, c$ are natural numbers) are completely covered without overlap by rectangles whose edges are parallel to the edges of the box, each containing an even number of unit squares. (Rectangles may cross the lateral edges of the box.) Prove that if $c$ is odd, then the number of possible coverings is even.

Solution: We may replace the box by a cylinder of circumference $2(a+b)$ and height $c$. Select a $1\times c$ rectangle consisting of $c$ unit squares, and draw a plane through the axis of symmetry bisecting this column. In any covering, there must be a rectangle covering this column and not symmetric across the plane (any symmetric

93

rectangle has odd width and thus even height, and the height of the column is odd). Thus reflecting across this plane pairs off the possible coverings, and their total number is even.

28. Do there exist real numbers $b$ and $c$ such that each of the equations $x^{2}+bx+c=0$ and $2x^{2}+(b+1)x+c+1=0$ have two integer roots?

Solution: No. Suppose they exist. Then $b+1$ and $c+1$ are even integers, so $b$ and $c$ are odd and $b^{2}-4c\equiv 5 (\mathrm{m}\mathrm{o}\mathrm{d}\ 8)$ is not a square, a contradiction.

29. A class consists of 33 students. Each student is asked how many other students in the class have his first name, and how many have his last name. It turns out that each number from $0$ to 10 occurs among the answers. Show that there are two students in the class with the same first and last name.

Solution: Consider groups of students with the same first name, and groups of students with the same last name. Each student belongs to two groups, and by assumption there are groups of size 1, . . . , 11; but these numbers add up to $66=2\cdot 33$, so there is one group of each size from 1 to 11 and no other groups.

Suppose the group of 11 is a group of students with the same first name. There are at most 10 groups by last name, so two students in the group of 11 must also have the same last name.

30. The incircle of triangle {\it ABC} touches sides {\it AB}, $BC, CA$ at $M, N, K,$ respectively. The line through $A$ parallel to $NK$ meets $MN$ at $D.$ The line through $A$ parallel to $MN$ meets $NK$ at $E$. Show that the line $DE$ bisects sides $AB$ and $AC$ of triangle {\it ABC}.

Solution: Let the lines $AD$ and $AE$ meet $BC$ at $F$ and $H$, respectively. It suffices to show that $D$ and $E$ are the midpoints of $AF$ and $AH$, respectively. Since $BN=BM$ and $MN||AH$, the trapezoid {\it AMNH} is isosceles, so $NH=AM$. Likewise $NF=AK.$ Since $AK=AM, N$ is the midpoint of $FH$. Since $NE$ is parallel to $AF, E$ is the midpoint of $AH$, and likewise $D$ is the midpoint of $AF.$

94

31. [Corrected] The numbers from 1 to 100 are arranged in a $10\times 10$ table so that no two adjacent numbers have sum less than $S$. Find the smallest value of $S$ for which this is possible.

Solution: The minimum is $S=106$. Suppose the numbers from 100 to 1 (in decreasing order) are placed in the grid so as to ensure $S\leq 105$. Divide the grid into 5 $2\times 10$ horizontal strips and 5 $10\times 2$ vertical strips, and let $n_{0}$ be the number whose insertion first causes there to be at least one number in each horizontal strip, or in each vertical strip. If $n_{0}<68$, then the 33 numbers 68, . . . , 100 fit into 64 squares (in the 16 intersections of strips) which can be divided into $1\times 2$ rectangles; thus two numbers must be adjacent, but $68+69>105$, so this is impossible. Hence $n_{0}\geq 68.$

We next note that at the moment after $n_{0}$ is inserted, no strip contains 10 numbers. Suppose on the contrary that some horizontal strip contains 10 numbers. Before $n_{0}$ was inserted, the strip contained at least 9 numbers, none next to another, and so each of its intersections with the vertical strips contains a number, contradicting the choice of $n_{0}.$

Now we observe that $n\leq 9$ nonadjacent squares in a strip always have at least $n+1$ distinct neighbors in the strip. Indeed, if the strip is horizontal, each square is adjacent to the other square in its vertical $1\times 2$ rectangle, making $n$ neighbors. Also, there is at least one empty vertical $1\times 2$ rectangle in the strip, and there is at least one more neighbor in an empty rectangle.

We conclude that the $101-n_{0}$ numbers from $n_{0}$ to 100 are adjacent to at least $(101-n_{0})+5=106-n_{0}$ others. So some number from $n_{0}$ to 100 is adjacent to a number at least as big as $106-n_{0}$, so two adjacent numbers sum to at least 106, contradiction.

Hence $S\geq 106$, and the following example shows that $S=106$ is possible.
$$
100\ 1\ 99\ 2\ 98\ 3
$$
$$
6\ 95\ 7\ 94\ 8\ 93
$$
$$
90\ 11\ 89\ 12\ 88\ 13
$$
32. Find all integer solutions of the equation
$$
(x^{2}-y^{2})^{2}=1+16y.
$$
Solution: The solutions are
$$
(\pm 1,0),\ (\pm 4,3),\ (\pm 4,5)\ .
$$
We must have $y\geq 0$. As the right side is nonzero, so then must be the left side, hence $|x|\geq|y|+1$ or $|x|\leq|y|-1$. In either case, $(x^{2}-y^{2})^{2}\geq(2y-1)^{2}$, so $(2y-1)^{2}\leq 1+16y$, and hence $y\leq 5.$ Trying all such values of $y$ yields the above solutions.

33. An $n\times n$ square grid $(n\geq 3)$ is rolled into a cylinder. Some of the cells are then colored black. Show that there exist two parallel lines (horizontal, vertical or diagonal) of cells containing the same number of black cells.

Solution: We may as well join the edges of the cylinder to form a torus. Assume the contrary. Then one horizontal line contains either $0$ or $n$ black cells; without loss of generality (by interchanging black with nonblack) we may assume the latter. Then no vertical or diagonal line contains $0$ black cells so one line of each type contains $n$ black cells, and no horizontal line contains $0$ black cells either. Thus the lines of each type contain 1, 2, . . . , $n$ black cells.

Number the rows and columns 1, . . . , $n$ in order, where in each case 1 is the row/column containing $n$ black cells. Then every diagonal except the two passing through (1, 1) contains 2 black cells, so those two diagonals must contain no other black cells. If $n$ is odd, this precludes any row from containing $n-1$ black cells, a contradiction. If $n$ is even, the diagonals passing through (1, 1) meet again at $(n/2+ 1, n/2+1)$ , so all of the other cells in row and column $n/2+1$ must be black. But now no row or column can contain one black cell, $\mathrm{a}$ contradiction.

34. Two circles intersect at $A$ and $B$. A line through $A$ meets the first circle again at $C$ and the second circle again at $D$. Let $M$ and $N$ be the midpoints of the arcs $BC$ and $BD$ not containing $A$, and let

96

$K$ be the midpoint of the segment $CD$. Show that $\angle MKN=\pi/2.$ (You may assume that $C$ and $D$ lie on opposite sides of $A.$)

Solution: Let $N_{1}$ be the reflection of $N$ about $K$. Then triangles $KCN_{1}$ and {\it KDN} are congruent, so $CN_{1}=ND$ and $\angle N_{1}CK= \angle NDK=\pi-\angle ABN$. From this we note that
$$
\angle MCN_{1}=(\pi-\angle ABN)+(\pi-\angle ABM)=\angle MBN.
$$
Also, $CN_{1}=DN=BN$ and $CM=BM$. Hence the triangles $MCN_{1}$ and {\it MBN} are congruent, so $MN=MN_{1}$ and the median $MK$ is also an altitude. Thus $\angle MKN=\pi/2.$

35. A polygon can be divided into 100 rectangles, but not into 99. Prove that it cannot be divided into 100 triangles.

Solution: We first prove by induction on $k$ that a $2k$-gon which can be divided into rectangles can always be divided into at most $k-1$ rectangles. First note that such a polygon has all angles equal to $90^{\circ}$ or $270^{\circ}$. If all angles equal $90^{\circ}$, we have a rectangle, which certainly satisfies the claim. Otherwise, pick a $270^{\mathrm{o}}$ angle, and extend one of its sides to meet the polygon again. This divides the polygon into a $2m$-gon and $\mathrm{a}(2k-2m+2)$-gon, which can be divided into $m-1$ and $k-m$ rectangles, respectively.

From the lemma, we find our given polygon has more than 200 vertices. If the polygon is divided into $m$ triangles, these triangles have total angular measure $m\cdot 180^{\circ}$. However, each angle of the polygon must be filled with angles of the triangles whose measures add up to $90^{\circ}$ (a $270^{\circ}$ angle can be reduced to two angles adding up to $90^{\circ}$ if the side of a triangle passes through it). Thus $m\cdot 180>200\cdot 90,$ whence $m>100.$

36. Do there exist two quadratic trinomials $ax^{2}+bx+c$ and $(a+1)x^{2}+ (b+1)x+(c+1)$ with integer coefficients, both of which have two integer roots?

Solution: No. Without loss of generality, assume $a$ is even (or else replace $a$ by $-1-a$). If $ax^{2}+bx+c$ has integer roots, then $a$ must divide $b$ and $c$, so $b$ and $c$ are even. But now the polynomial

98
\end{document}
